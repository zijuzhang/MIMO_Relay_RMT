\documentclass[12pt,a4paper]{report}
\usepackage[utf8]{inputenc}
\usepackage{amsmath}
\usepackage{amsfonts}
\usepackage{amssymb}
\usepackage[colorlinks=true,linkcolor=blue]{hyperref}
\usepackage{graphicx,tikz}
\usepackage{float}
\usetikzlibrary{positioning}
\input defs.tex
\bibliographystyle{ieeetr}
\graphicspath{ {./figures/} }

\title{Progress Report}
\author{Peter Hartig}

\begin{document}
\maketitle
\tableofcontents

\section{Asymptotic Vs. Numerical Results}
First I confirm that the methods I use to obtain the AED given a closed-form S-Transform work for basic cases.
The numerical results  used for comparison are generated by averaging 100 realizations of the system for size $N=100 \times 100$.
As seen in the figures below, the solution becomes incorrect if there is too much correlation in the channel. 
\begin{itemize}
\item
	Channel 
	\begin{equation}
	\mH  \sim \CNO
	\end{equation}
	For equal power allocation 
	\\
	Numeric Capacity = $  1.4286$ 
	\\
	Asymptotic Capacity = $ 1.4129 $
	
\item
	Channel 
	\begin{equation}
	\mH_2 \phase \mH_1  
	\end{equation}
	with $\mH \sim \CNO$.
	For equal power allocation 
	\\
	Numeric Capacity = $ 1.4227$ 
	\\
	Asymptotic Capacity = $ 1.4060 $	
	
	* This difference between numeric and asymptotic  can be reduced by increasing the AED sampling density.
	
\item
	The single Rayleigh channel with receiver correlation given by 
	\begin{equation}
	\mC\mH  
	\end{equation}
	with $\mC$ as the exponential correlation matrix  with elements $c_{i,j} = \rho^{i-j}$. 
	For equal power allocation the capacities are plotted below. Because the s-transform for the correlation has a quadratic form, I show the solutions for both the $+/-$ forms of the quadratic expression.
	
	\begin{figure}[H]
	\includegraphics[width= 15cm,height = 10cm]{figures/correlation_comparison}
	  \caption{Equal power capacity with correlation form 1.}
	\end{figure}
	\begin{figure}[H]
	\includegraphics[width= 15cm,height = 10cm]{figures/correlation_comparison2}
	  \caption{Equal power capacity with correlation form 2.}
\end{figure}

\item
	The IRS channel without LOS and correlation only at receiver
	\begin{equation}
	\mC\mH_2 \phase\mH_1
	\end{equation}
	with $\mC$ as the exponential correlation matrix. 
	For equal power allocation the capacities are plotted below. 
		\begin{figure}[H]
	\includegraphics[width= 15cm,height = 10cm]{figures/equation1_Rx_only}
	  \caption{Equal power capacity for IRS channel without LOS and correlation only at receiver.}
	\end{figure}
\item
	The IRS channel without LOS and correlation at receiver and IRS elements
	\begin{equation}
	\mC_2 \mH_2\phase\mC_1 \mH_1
	\end{equation}
	with $\mC_i$ as an exponential correlation matrix. 
	For equal power allocation the capacities are plotted below. 
		\begin{figure}[H]
	\includegraphics[width= 15cm,height = 10cm]{figures/Rx_IRS_Correlation}
	  \caption{Equal power capacity for IRS channel without LOS and correlation at receiver and IRS.}
	\end{figure}	
	
\item
	The IRS channel without LOS and correlation at receiver, IRS elements and transmitter
	\begin{equation}
	\mC_3 \mH_2\mC_2 \phase\mC_1 \mH_1
	\end{equation}
	with $\mC_i$ as the exponential correlation matrix. 
	For equal power allocation the capacities are plotted below. 
		\begin{figure}[H]
	\includegraphics[width= 15cm,height = 10cm]{figures/Rx_IRS_Tx_Correlation}
	  \caption{Equal power capacity for IRS channel without LOS and correlation at receiver, IRS and transmitter.}
	\end{figure}	
	
\item
	These are the steps I use to move from the closed-form S-Transform of the channel to the AED.
	The S-Tranform of the exponential correlation model is derived in \cite{skupch2005free}
	\begin{enumerate}
	\item
		Beginning with the S-Transform, $S(z)$, I find $\gamma(s)$ using the fixed-point equation
		\begin{equation}
			S(\gamma(z))\gamma(z) - z - z\gamma(z) = 0 
		\end{equation}
	\item
		I find the Stieltjes transform using
		\begin{equation}
			G(s) = \frac{1}{s} (1+\gamma(\frac{1}{s}))
		\end{equation}			
	\item 
		I recover the AED using 
\begin{equation}
p(x) = \frac{1}{\pi} \underset{y \rightarrow \infty}{\text{lim}} \; \text{Img}(G(x+jy))
\end{equation}
	\end{enumerate}

\item
	I checked the equation provided for the gamma transform of just the the exponential correlation model and observed the following AEDs.
		\begin{figure}[H]
	\includegraphics[width= 15cm,height = 10cm]{figures/exponential_correlation_aed}
	  \caption{Comparison of AED for the exponential correlation model using results from \cite{skupch2005free}}
	\end{figure}

\end{itemize}
\section{Capacity Convergence}
As there is not a method for evaluating the rate of convergence of finite systems towards the asymptotic behavior I investigate this numerically in the following figures by looking at the variance of the capacity of numerical simulations for the IRS channel given by 
	\begin{equation}
	\mC_3 \mH_2\mC_2 \phase\mC_1 \mH_1
	\end{equation}
Figures \ref{equal_power} and \ref{water_filling} show the average capacity and capacity variance over different, random, values of  $\phase$ with all channels held constant for equal power allocation and optimal power allocation respectively. Note that \emph{neither} uses any optimization of the phases at the IRS. 
These results reflect the intuition that smaller finite system sizes with significant correlation are not well represented by the asymptotic result that shows the phase matrix at the IRS canceling. It might be interesting to see if an optimization of the IRS phases reduces this variance in the high correlation case. 

		\begin{figure}[H]
	\includegraphics[width= 15cm,height = 10cm]{figures/correlation_equal_power_capacity}
	  \caption{Capacity average and variance for system size $100$ equal power allocation over $N=1000$ channel realizations}
	  \label{equal_power}
	\end{figure}
	
		\begin{figure}[H]
	\includegraphics[width= 15cm,height = 10cm]{figures/correlation_water_filling_capacity}
	  \caption{Capacity average and variance for system size $100$ equal power allocation over $N=100$ channel realizations}
	  	  \label{water_filling}
	\end{figure}

\section{Questions}
\begin{enumerate}
\item
	Because we can show that the phases cancel for the asymptotic analysis, it seems that any further look at this is just extending capacity results without 
	IRS to a more complicated channel model. Can I focus the remainder of the work on just looking at the when the asymptotic assumption is valid?
\item
	To start considering the "zero-tapping" that was suggested in our last meeting.I began by using the S-Transform of the "projector" matrix (essentially resulting in the deformed quarter circle law for $\mH \in \complex^{N \times K}$ with $N \leq K$). There are two points, I was hoping to discuss on this.
	\begin{enumerate}
	\item
		First, one problem is that this is also changing the power of the received signal to $\frac{N}{K}$ 
		 because the deformed quarter circle assumes elements with variance $\frac{1}{K}$ rather than $\frac{1}{N}$.
\item
		One other strange thing I noticed is that when I perform numerical simulations with the deformed quarter circle the distributions I have, as shown in Figure \ref{AED} 
		appear to not integrate to 1 but nevertheless  the resulting capacity they agree with numerical results as shown in Figure \ref{projection}.
				\begin{figure}[H]
	\includegraphics[width= 12cm,height = 8cm]{figures/projection}
	  \caption{Comparison of AED for the exponential correlation model using results from \cite{skupch2005free}}
	  \label{projection}
	\end{figure}
		\begin{figure}[H]
	\includegraphics[width= 12cm,height = 8cm]{figures/AED}
	  \caption{Comparison of AED for the exponential correlation model using results from \cite{skupch2005free}}
	  	  \label{AED}
	\end{figure}

	\end{enumerate}
\item
I was also preparing some slides for the mid-term presentation and a question on the background.
\begin{itemize}
\item
	Will it be okay to just assume flat-fading using OFDM rather than including a model with multiple delays? Will OFDM still be possible at very high frequencies?
\end{itemize}
\end{enumerate}
\bibliography{bibliography}

\end{document}
