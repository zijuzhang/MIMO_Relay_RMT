\documentclass[12pt,a4paper]{report}
\usepackage[utf8]{inputenc}
\usepackage{amsmath}
\usepackage{amsfonts}
\usepackage{amssymb}
\usepackage[colorlinks=true,linkcolor=blue]{hyperref}
\usepackage{graphicx,tikz}
\usepackage{float}
\usetikzlibrary{positioning}
\input defs.tex
\bibliographystyle{ieeetr}
\graphicspath{ {./figures/} }

\title{Progress Report}
\author{Peter Hartig}

\begin{document}
\maketitle
\tableofcontents

\section{Status}
\begin{itemize}
\item
From 5/29/20 Update (yikes almost a year ago...)
\item
Known S-Transform for the exponential correlation model comes from \cite{skupch2005free}
\item
Ideally, the outline should go like:
\begin{enumerate}
\item 
 Show phases cancel and the capacity for exponential correlation (for high correlations). Note that the S-Transform of the correlation matrix is only valid for infinite size.
\item 
Look at ways of increasing capacity (and rate) for the small cases. Starting with $N_T = N_R = 1$.s
\item 
Conclude by looking at system sizes for which the asymptotic values represent the system well.
\item
For proving freeness, skupch also says that for H being gaussian, it is a wishart matrix and it is free w.r.t the correlation... Need to show that it is okay as long as the correlation matrices are not next to oneanother?
\end{enumerate}
\end{itemize}
\bibliography{bibliography}

\end{document}
