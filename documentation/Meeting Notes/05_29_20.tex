\documentclass[12pt,a4paper]{report}
\usepackage[utf8]{inputenc}
\usepackage{amsmath}
\usepackage{amsfonts}
\usepackage{amssymb}
\usepackage[colorlinks=true,linkcolor=blue]{hyperref}
\usepackage{graphicx,tikz}
\usepackage{float}
\usetikzlibrary{positioning}
\input defs.tex
\bibliographystyle{ieeetr}
\graphicspath{ {./figures/} }

\title{Progress Report}
\author{Peter Hartig}

\begin{document}
\maketitle
\tableofcontents

\section{Channel Model}\label{section:channel}
In this update I detail the channel model I am using as well as the current issue I am having in finding the resulting channel AED. 
\par
The channel considered includes correlation at the transmitter, receiver and IRS, as well as a LOS component with a non-zero mean. First, I describe the general channel. Then I will begin with a simple channel in order to investigate the calculation of the AED as each component (correlation, IRS, LOS) is introduced. 
\par
The general channel is given by 
	\begin{equation}
	\mathbf{H}_{Total} = \mathbf{R}_{R}(\underbrace{\mathbf{H}_{2}\boldsymbol{\Phi}\mathbf{R}_{S}\mathbf{H}_{1}}_{\text{IRS}} + \underbrace{\mathbf{G}}_{\text{LOS}})\mathbf{R}_{T}.
	\end{equation}
	 $\mathbf{R}_{R}$, $\mathbf{R}_{T}$ and $\mathbf{R}_{S}$ represent deterministic correlation matrices at the receiver, transmitter and IRS. $\mathbf{H}_{1}$ and $\mathbf{H}_{2}$ have i.i.d elements in $\mathcal{NC}(0,\frac{1}{N})$, $\mathbf{G}$ has i.i.d elements in $\mathcal{NC}(\mu,\frac{1}{N})$ and $\boldsymbol{\Phi}$ represent the phase shifts at the IRS elements.
	This simplify notation, I will consider all matrices to have dimension $N \times N$. 

\subsection{Correlation Normalization}

In order to evaluate the impact of correlation on the system capacity, the power of the correlation matrices, $\mathbf{R}_{R}$, $\mathbf{R}_{T}$ and $\mathbf{R}_{S},$ must be normalized to prevent adding power to the received signal. For $\mathbf{H}$ with $h_{i,j} \in \mathcal{NC}(0,\frac{1}{N})$ and received signal $\mathbf{y} = \mathbf{H} \mathbf{x}$ with $E[\mathbf{x}\mathbf{x}^H] = \mathbf{I}$, the expected received power is given by 
$E[\text{trace}\left( \mathbf{y} \mathbf{y}^H \right) ]= E[\text{trace}\left(\mathbf{H} \mathbf{H}^H \right)] = N$.
For the correlated channel given by $\mathbf{G} = \mathbf{R}\mathbf{H}$, the received power is given by $E[\text{trace}\left(\mathbf{R} \mathbf{H} \mathbf{H}^H \mathbf{R}^H\right)] = E[\text{trace}\left(\mathbf{R}\mathbf{R}^H\right)] $. This constraint is equivalent to requiring that the columns of $\mathbf{R}$, $\mathbf{r}_i$ are normalized such that $\|\mathbf{r}_i\|^2_2 =1$. 
 For the case in which the correlation is a pre-multiplication, $\mathbf{G} = \mathbf{H}\mathbf{R}$, the rotation property of the trace gives the same constraint on the columns of 
 $\mathbf{R}$.
 
*Need to discuss water filling at some point still.

\section{Correlation Normalization}
In order to use tools from free probability to analyze the AED of this channel, I first need to confirm that the components of the above channel are free. For the channel shown in \ref{section:channel} there are two components  to consider.
\begin{enumerate}
\item 
	First just consider the Concatenated matrices with correlation
	\begin{equation}
	\mathbf{H}_{Total} = \mathbf{R}_{R}\mathbf{H}_{2}\boldsymbol{\Phi}\mathbf{R}_{S}\mathbf{H}_{1}\mathbf{R}_{T}
	\end{equation}

\item 
	Adding matrices with the same type of correlation	
\end{enumerate}

\section{Evaluating Channel Components}

\subsection{Basic Correlated Channel}\label{ssection:basic_correlation}
	Before considering the general channel,  I will go through the calculation of the AED the basic MIMO 
	channel with correlation given by
	\begin{equation}
	\mathbf{H}_{Total} = \mathbf{R}_{R}\mathbf{G}
	\end{equation}
	in which $\mathbf{R}_{R}$ is a deterministic matrix with normalized columns describing the correlation and
	elements $\mathbf{g}_{i,j} \in \mathcal{NC}(0,\frac{1}{N})$ and are iid.
	
\subsubsection{Evaluating Freeness}
Rotating the covariance matrix $\mathbf{C} = \mathbf{H}_{Total} \mathbf{H}_{Total}^H$ to get $ \tilde{\mathbf{C}}=\mathbf{R}_{R}^H \mathbf{R}_{R} \mathbf{G} \mathbf{G}^H
$ gives a matrix which has the same non-zero eigenvalues. 
$ \mathbf{R}_{R}^H \mathbf{R}_{R}$ is free from any matrix with upper bounded norm and a limit distribution.
The correlation matrix $\mathbf{G} \mathbf{G}^H$ satisfies these conditions so these two matrices are free. 
(Pg. 77 of Random Matrix Theory Notes. TODO: find literature source)

\subsubsection{AED Expression}
	
	The following steps and equations are used to find the AED of $\mathbf{C} = \mathbf{H}_{Total} \mathbf{H}_{Total}^H$.
	\begin{enumerate}
	\item 
		Find AED from the stieltjes transform.
		\begin{equation}
		p(x) = \frac{1}{\pi} \underset{y \rightarrow \infty}{\text{lim}} \; \text{Img}(G(x+jy))
		\end{equation}
	\item 
		Find Stieltjes transform from $\gamma\left(\cdot \right)$
		\begin{equation}
		G(s)=  \frac{1}{s} (1+\gamma\left(\frac{1}{s}\right))
		\end{equation}
	\item 
		To find an expression for $\gamma(z)$, evaluate the equation
		\begin{equation}
		\gamma^{-1}(z) =  \frac{z}{1+z} S(z)
		\end{equation}
		at $z= \gamma(z)$
		to get the fixed point equation
		\begin{equation}
		\gamma(z) = \frac{z(1+ \gamma(z))}{S(\gamma(z))}
		\end{equation}
		I can find the AED using steps 1-3 as long as I can evaluate $S(z)$ is in closed form. 
		In general, however, this is not the case (e.g when $S(z)$ includes a deterministic correlation matrix). 
		In the following steps I show how I am trying to find $S(z)$ when I do not have a closed form expression. 
	\item 
		For the case with all square matrices, 
		\begin{equation}\label{rotation_property}
		S_{\mathbf{C}}(z) =  S_{\tilde{\mathbf{C}}}\left(z\right)
		\end{equation}
		with the rotated channel $\tilde{\mathbf{C}} = \mathbf{R}_{R}^H\mathbf{R}_{R}\mathbf{G}\mathbf{G}^H$.
		This gives
		\begin{equation}
		S_{\tilde{\mathbf{C}}}\left( z \right) = S_ {\mathbf{R}_{Rx}^H\mathbf{R}_{Rx}}(z)
		\underbrace{S_ {\mathbf{G}\mathbf{G}^H}(z)}_{\text{close form}}.
		\end{equation}
	\item 
		To find the S-Transform of the correlation matrix, $S_ {\mathbf{R}_{R}^H\mathbf{R}_{R}}(z)$,
		use the property 
		\begin{equation}
		 S(z) = \gamma^{-1}(z)\frac{1+z}{z}.
		\end{equation}
	\item
		Now find an expression for $\gamma^{-1}(z)$ using the known Steiltjes transform of the deterministic
		 correlation matrix. 
		 \par
		 If I use the procedure shown in the following, the value of $S(z)$ is 
		generally not correct when compared to a known S-Transforms (e.g. when I try this procedure for the 
		quarter circle and compare it to the known S-Transform, it is not correct).
	\item
		Evaluate the the equation 
		\begin{equation}
		G(s)=  -\frac{1}{s} (1+\gamma\left(\frac{1}{s}\right))
		\end{equation}
		at $s = \gamma^{-1}(z)$ 
		to find the fixed point equation
		 \begin{equation}
		\gamma^{-1}(z) = \frac{-G(\frac{1}{\gamma^{-1}(z)})}{z+1}
		\end{equation}
		* When I try this in MATLAB, I get a the error that fsolve does not converge.
	\item 
		For the case of deterministic matrices 
		\begin{equation}
		G(s) = \int \frac{1}{x-s} dPx
		\end{equation}
		becomes 
		\begin{equation}
		G(s) = \sum_{i=1}^{N} \frac{1}{N (\lambda_i - s)}.
		\end{equation}	
	\end{enumerate}


\subsection{LOS Component}\label{ssection:los}

The next step the correlated LOS component with non-zero mean is added. 
	\begin{equation}
	\mathbf{H}_{Total} = \mathbf{R}_{R}\mathbf{G}
	\end{equation}
	with 
	\begin{equation}
	\mathbf{G} = \underbrace{\mathbf{H}}_{\mathcal{NC}(0,\sigma)} + \underbrace{\alpha\mathbf{I}}_{\text{scaled identity}}
	\end{equation}
	\begin{enumerate}
	\item
	Using the property 
		\begin{equation}\label{svd_aed_property}
		G_{\lambda\lambda}(s) = \frac{1}{\sqrt{s}}\tilde{G}_{\lambda}(\sqrt{s})
		\end{equation}
	avoids having to work with non-hermetian and non-free terms in the covariance polynomial $\mathbf{H}_{Total}\mathbf{H}_{Total}^H$. 
	\item 
		Next obtain the symmetrized singular value from the corresponding R-Transform using
		\begin{equation}
		\tilde{G}_{\lambda}(\sqrt{s}) = \frac{1}{\tilde{R}(-\tilde{G}_{\lambda}(\sqrt{s})) - s}
		\end{equation}
		and 
			\begin{equation}
			\tilde{R}(w) = \tilde{R}_{\mathbf{R}\mathbf{H}}(w) + \tilde{R}_{\alpha\mathbf{R}\mathbf{I}}(w)
			\end{equation}
	\item 
		In order to find the 
	\end{enumerate}	


\subsection{Correlated LOS and IRS components}\label{ssection:los_correlated}
Now I consider the case in which the same correlation is applied to both the LOS and IRS components of the channel.	

\end{document}
