\documentclass[12pt,a4paper]{report}
\usepackage[utf8]{inputenc}
\usepackage{amsmath}
\usepackage{amsfonts}
\usepackage{amssymb}
\usepackage[colorlinks=true,linkcolor=blue]{hyperref}
\usepackage{graphicx,tikz}
\usepackage{float}
\usetikzlibrary{positioning}
\input defs.tex
\bibliographystyle{ieeetr}
\graphicspath{ {./figures/} }

\title{Progress Report}
\author{Peter Hartig}

\begin{document}
\maketitle
\tableofcontents

\section{Channel Model}\label{section:channel}
In this update I detail the channel model I am using as well as the current issue I am having in finding the resulting channel AED. 
\par
The channel considered includes correlation at the transmitter, receiver and IRS, as well as a LOS component with a non-zero mean given by 
	\begin{equation}
	\mathbf{H}_{Total} = \mathbf{R}_{R}(\underbrace{\mathbf{H}_{2}\boldsymbol{\Phi}\mathbf{R}_{S}\mathbf{H}_{1}}_{\text{IRS}} + \underbrace{\mathbf{G}}_{\text{LOS}})\mathbf{R}_{T}.
	\end{equation}
	 $\mathbf{R}_{R}$, $\mathbf{R}_{T}$ and $\mathbf{R}_{S}$ represent deterministic correlation matrices at the receiver, transmitter and IRS. $\mathbf{H}_{1}$ and $\mathbf{H}_{2}$ have i.i.d elements in $\mathcal{NC}(0,\frac{1}{N})$, $\mathbf{G}$ has i.i.d elements in $\mathcal{NC}(\mu,\frac{1}{N})$ and $\boldsymbol{\Phi}$ represent the phase shifts at the IRS elements.
	This simplify notation, I will consider all matrices to have dimension $N \times N$. 

\subsection{Correlation Normalization}

In order to evaluate the impact of correlation on the system capacity, the power of the correlation matrices, $\mathbf{R}_{R}$, $\mathbf{R}_{T}$ and $\mathbf{R}_{S},$ must be normalized to prevent adding power to the received signal. For $\mathbf{H}$ with $h_{i,j} \in \mathcal{NC}(0,\frac{1}{N})$ and received signal $\mathbf{y} = \mathbf{H} \mathbf{x}$ with $E[\mathbf{x}\mathbf{x}^H] = \mathbf{I}$, the expected received power is given by 
$E[\text{trace}\left( \mathbf{y} \mathbf{y}^H \right) ]= E[\text{trace}\left(\mathbf{H} \mathbf{H}^H \right)] = N$.
For the correlated channel given by $\mathbf{G} = \mathbf{R}\mathbf{H}$, the received power is given by $E[\text{trace}\left(\mathbf{R} \mathbf{H} \mathbf{H}^H \mathbf{R}^H\right)] = E[\text{trace}\left(\mathbf{R}\mathbf{R}^H\right)] $. This constraint is equivalent to requiring that the columns of $\mathbf{R}$, $\mathbf{r}_i$ are normalized such that $\|\mathbf{r}_i\|^2_2 =1$. 
 For the case in which the correlation is a pre-multiplication, $\mathbf{G} = \mathbf{H}\mathbf{R}$, the rotation property of the trace gives the same constraint on the columns of 
 $\mathbf{R}$.
 
%*Need to discuss water filling at some point still.

\section{Evaluating Channel Components}
\subsection{Note from Previous work}\label{ssection:basic_correlation}
In \cite[Section II Eqn. 4]{muller2012channel} it is mentioned that correlation can be added "easily" using free probability once the AED of the uncorrelated channel is known. Can we discuss these steps towards easily incorporating this correlation? They are not apparent to me. In the following I attempt to find the AED in a similar case but it seems quite tedious, I imagine there is an easier way. 

\subsection{Basic Correlated Channel}\label{ssection:basic_correlation}
	Before considering the general channel,  I will go through the calculation of the AED of a basic MIMO 
	channel with correlation given by
	\begin{equation}
	\mathbf{H}_{Total} = \mathbf{R}_{R}\mathbf{G}
	\end{equation}
	in which $\mathbf{R}_{R}$ is a deterministic matrix with normalized columns describing the correlation and
	elements $\mathbf{g}_{i,j} \in \mathcal{NC}(0,\frac{1}{N})$ and are iid.
	
%\subsubsection{Evaluating Freeness}
%First, I show that free probability can be used. 
%Rotating the covariance matrix $\mathbf{C} = \mathbf{H}_{Total} \mathbf{H}_{Total}^H$ to get $ \tilde{\mathbf{C}}=\mathbf{R}_{R}^H \mathbf{R}_{R} \mathbf{G} \mathbf{G}^H
%$ gives a matrix which has the same non-zero eigenvalues. The hermetian matrix
%$ \mathbf{R}_{R}^H \mathbf{R}_{R}$ is free from any matrix with upper bounded norm and a limit distribution.
%The correlation matrix $\mathbf{G} \mathbf{G}^H$ satisfies these conditions so these two matrices are free. 
%(Pg. 77 of Random Matrix Theory Notes. TODO: find literature source)

\subsubsection{AED Expression}
	
	The following steps and equations are used to find the AED of $\mathbf{C} = \mathbf{H}_{Total} \mathbf{H}_{Total}^H$.
	\begin{enumerate}
	\item 
		Expression for AED given the stieltjes transform.
		\begin{equation}
		p(x) = \frac{1}{\pi} \underset{y \rightarrow \infty}{\text{lim}} \; \text{Img}(G(x+jy))
		\end{equation}
	\item 
		Expression for  Stieltjes transform given $\gamma\left(\cdot \right)$
		\begin{equation}
		G(s)=  \frac{1}{s} (1+\gamma\left(\frac{1}{s}\right))
		\end{equation}
	\item 
		To find an expression for $\gamma(z)$, evaluate 
		\begin{equation}
		\gamma^{-1}(z) =  \frac{z}{1+z} S(z)
		\end{equation}
		at $z= \gamma(z)$
		to get the fixed point equation
		\begin{equation}
		\gamma(z) = \frac{z(1+ \gamma(z))}{S(\gamma(z))}
		\end{equation}
		I can find the AED using steps 1-3 as long as I can evaluate $S(z)$ in closed form. 
		In general, however, this is not the case (e.g when $S(z)$ includes a deterministic correlation matrix). 
		In the following steps I show how I am trying to find $S(z)$ when I do not have a closed form expression. 
	\item 
		For the case with all square matrices, 
		\begin{equation}\label{rotation_property}
		S_{\mathbf{C}}(z) =  S_{\tilde{\mathbf{C}}}\left(z\right)
		\end{equation}
		with the rotated channel $\tilde{\mathbf{C}} = \mathbf{R}_{R}^H\mathbf{R}_{R}\mathbf{G}\mathbf{G}^H$.
		This gives
		\begin{equation}
		S_{\tilde{\mathbf{C}}}\left( z \right) = S_ {\mathbf{R}_{Rx}^H\mathbf{R}_{Rx}}(z)
		\underbrace{S_ {\mathbf{G}\mathbf{G}^H}(z)}_{\text{closed form}}.
		\end{equation}
	\item 
		To find the S-Transform of the correlation matrix, $S_ {\mathbf{R}_{R}^H\mathbf{R}_{R}}(z)$,
		use the property 
		\begin{equation}
		 S(z) = \gamma^{-1}(z)\frac{1+z}{z}.
		\end{equation}
	\item
		Now I find an expression for $\gamma^{-1}(z)$ using the known Steiltjes transform of the deterministic
		 correlation matrix. 
		 \par
		 If I use the procedure shown in the following, the value of $S(z)$ is 
		generally not correct when compared to a known S-Transforms (e.g. when I try this procedure for the 
		quarter circle eigenvalues and compare it to the known S-Transform, it is not correct).
	\item
		Evaluate the the equation 
		\begin{equation}
		G(s)=  -\frac{1}{s} (1+\gamma\left(\frac{1}{s}\right))
		\end{equation}
		at $s = \gamma^{-1}(z)$ 
		to find the fixed point equation
		 \begin{equation}
		\gamma^{-1}(z) = \frac{-G(\frac{1}{\gamma^{-1}(z)})}{z+1}
		\end{equation}
		* When I try this in MATLAB, I get a the error that fsolve does not converge.
	\item 
		For the case of deterministic matrices 
		\begin{equation}
		G(s) = \int \frac{1}{x-s} dPx
		\end{equation}
		becomes 
		\begin{equation}
		G(s) = \sum_{i=1}^{N} \frac{1}{N (\lambda_i - s)}.
		\end{equation}	
		for known eigenvalues $\lambda_i$.
	\end{enumerate}


%\subsection{LOS Component}\label{ssection:los}
%
%The next component of channel to consider is the correlated LOS channel given by  
%	\begin{equation}
%	\mathbf{H}_{Total} = \mathbf{R}\mathbf{G}
%	\end{equation}
%	with $\mathbf{R}$ as a deterministic correlation matrix and 
%	\begin{equation}
%	\mathbf{G} = 
%	\underbrace{\mathbf{H}}_{\mathcal{NC}(0,\sigma)} + \underbrace{\mathbf{D}}_{\text{Deterministic}}.
%	\end{equation}
%	
%\subsubsection{Evaluating Freeness}
%
%	
%\subsubsection{AED Expression}
%
%	\begin{enumerate}
%	\item
%	Using the property 
%		\begin{equation}\label{svd_aed_property}
%		G_{\lambda\lambda}(s) = \frac{1}{\sqrt{s}}\tilde{G}_{\lambda}(\sqrt{s})
%		\end{equation}
%	avoids having to work with non-hermetian and non-free terms in the covariance polynomial $\mathbf{H}_{Total}\mathbf{H}_{Total}^H$. 
%	\item 
%		Next obtain the Stieltjes transform of the symmetric singular value from the corresponding R-Transform using
%		\begin{equation}
%		\tilde{G}_{\lambda}(\sqrt{s}) = \frac{1}{\tilde{R}(-\tilde{G}_{\lambda}(\sqrt{s})) - s}
%		\end{equation}
%		and 
%			\begin{equation}
%			\tilde{R}(w) = \tilde{R}_{\mathbf{R}\mathbf{H}}(w) + \tilde{R}_{\mathbf{R}\mathbf{D}}(w)
%			\end{equation}
%	\item 
%		I will omit the next steps until I have a solution to the basic correlated channel because the basic
%		correlated channel needs to be solved to handle this case. 
%	\end{enumerate}	

\section{A note on correlation}
The capacity of the seems to have a higher variance if there is correlation.
This might motivate considering the rate of convergence towards the AED. If correlated channels approach this slowly, it motivates the use of phase optimization. 
Without correlation (N=1000)
\begin{itemize}
\item
Capacity variance:
 0.0036801195330893642
 \item
Capacity average:
 6.798435661685981
 \item
Capacity min:
 6.601426890868061
\item
Capacity max:
 6.989762012296481
\end{itemize}

With correlation (Rank = 4) (N=1000)
\begin{itemize}
\item
Capacity variance:
 0.021175917522394564
 \item
Capacity average:
 5.401856644539409
 \item
Capacity min:
 4.976408104173949
 \item
Capacity max:
 5.882360991047958
\end{itemize}

\bibliography{bibliography}

\end{document}
