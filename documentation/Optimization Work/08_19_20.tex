\documentclass[12pt,a4paper]{report}
\usepackage[utf8]{inputenc}
\usepackage{amsmath}
\usepackage{amsfonts}
\usepackage{amssymb}
\usepackage[colorlinks=true,linkcolor=blue]{hyperref}
\usepackage{graphicx,tikz}
\usepackage{float}
\usetikzlibrary{positioning}
\input defs.tex
\bibliographystyle{ieeetr}
\graphicspath{ {./figures/} }

\title{Progress Report}
\author{Peter Hartig}

\begin{document}
\maketitle
\tableofcontents

\section{Numerical Results}
The system model is given by 
\begin{equation}
\vy = \mH_{Total} \mW \vs  + \vn
\end{equation}
with 
	\begin{equation*}
	\mH_{Total} = \underbrace{\mathbf{H}_{2}\boldsymbol{\Phi}\mathbf{H}_{1}}_{\text{IRS}} + \underbrace{\mathbf{G}}_{\text{LOS}},
	\end{equation*}
%		\mH_{Total} = \mathbf{R}_{R}^{\frac{1}{2}}(\underbrace{\mathbf{H}_{2}\boldsymbol{\Phi}\mathbf{R}_{S}^{\frac{1}{2}}\mathbf{H}_{1}}_{\text{IRS}} + \underbrace{\mathbf{G}}_{\text{LOS}})\mathbf{R}_{T}^{\frac{1}{2}}.
	$\mW$ as a linear precoder and $\vs$ as the vector of transmit symbols.
	\par
	Using the method for CSI estimation from \cite{nadeem2019intelligent} to estimate the CSI to be used in the choice of $\mW$, the following procedure is used. 
	\begin{enumerate}
	\item
		With all IRS surfaces turned off LOS channel is estimated as $\hat{\mG}$ by 
		transmitting some pilot vector $\mathbf{p}$. This is done using the MMSE estimate for the columns of $\hat{\mG}$ using $\vy$.
		In this case the received signal is 
			\begin{equation*}
			\vy = \mG\mathbf{p}  + \vn.
			\end{equation*}
			
	\item 	
		In the next time interval, we turn on a single IRS element and transmit the pilot sequence again such that the received signal is 
			\begin{equation*}
			\vy = (\mG + \mathbf{h}_{2,1}\mathbf{h}^H_{1,1}  )\mathbf{p}  + \vn
			\end{equation*}
			where $\mathbf{h}_{2,1}$ is the 1st column of $ \mathbf{H}_{2}$ and $\mathbf{h}_{1,1}^H$  is the 1st row of $\mathbf{H}_{1}$.
			By subtracting off $\hat{\mG}\mathbf{p}$ from the received signal we have 
						\begin{equation*}
			\vy = (\mathbf{h}_{2,1}\mathbf{h}^H_{1,1})\mathbf{p}  + \vn
			\end{equation*}
			from which we can estimate the rank 1 matrix $\mathbf{h}_{2,1}\mathbf{h}^H_{1,1}$.
	\item 
		Iterating this procedure allows us to estimate both paths of the channel. 
		** Question: This is different from what was discussed in our meeting because in the method Ali mentioned. In that case, the IRS surfaces were turned on during the
		first round of estimation. We want to use the matched filter so it seems like we should have all IRS turned on in the first round like Ali mentioned. 
	\end{enumerate}
The precoded transmission $\vx = \mathbf{F}\mathbf{s}$ is a function of this estimate if we assume a matched filter (or other types of precoding).

\bibliography{bibliography}

\end{document}
