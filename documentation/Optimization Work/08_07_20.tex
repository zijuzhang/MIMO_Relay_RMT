\documentclass[12pt,a4paper]{report}
\usepackage[utf8]{inputenc}
\usepackage{amsmath}
\usepackage{amsfonts}
\usepackage{amssymb}
\usepackage[colorlinks=true,linkcolor=blue]{hyperref}
\usepackage{graphicx,tikz}
\usepackage{float}
\usetikzlibrary{positioning}
\input defs.tex
\bibliographystyle{ieeetr}
\graphicspath{ {./figures/} }

\title{Progress Report}
\author{Peter Hartig}

\begin{document}
\maketitle
\tableofcontents

\section{Numerical Results}
I use the system model given by 
\begin{equation}
\vy = \mH_{Total} \mW \vs  + \vn
\end{equation}
with 
	\begin{equation}
	\mH_{Total} = \mathbf{R}_{R}^{\frac{1}{2}}(\underbrace{\mathbf{H}_{2}\boldsymbol{\Phi}\mathbf{R}_{S}^{\frac{1}{2}}\mathbf{H}_{1}}_{\text{IRS}} + \underbrace{\mathbf{G}}_{\text{LOS}})\mathbf{R}_{T}^{\frac{1}{2}}.
	\end{equation}
Below are the simulation results using the simulation system parameters as shown in the previous results but now plotted over $\theta \in [0, 2\pi]$ for $\phase = e^{j\theta}\mI$. This result suggests that the optimal $\phase$ with respect to the MSE depends  primarily on the fast-fading characteristics of the channel.
		\begin{figure}[H]
	\includegraphics[width= 16cm,height = 13cm]{figures/plotted_over_theta}
	  \caption{Comparison MSE over different amounts of correlation.}
	  	  \label{MSE_correlation}
	\end{figure} 

\end{document}
