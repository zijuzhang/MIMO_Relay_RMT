\documentclass[12pt,a4paper]{report}
\usepackage[utf8]{inputenc}
\usepackage{amsmath}
\usepackage{amsfonts}
\usepackage{amssymb}
\usepackage[colorlinks=true,linkcolor=blue]{hyperref}
\usepackage{graphicx,tikz}
\usepackage{float}
\usetikzlibrary{positioning}
\input defs.tex
\bibliographystyle{ieeetr}
\graphicspath{ {./figures/} }

\title{Progress Report}
\author{Peter Hartig}

\begin{document}
\maketitle
\tableofcontents
\section{System Overview}
The communication system model is given by 
\begin{equation}
\vy = \mH_{Total} \mW \vs  + \vn
\end{equation}
with 
	\begin{equation*}
	\mH_{Total} = \underbrace{\mathbf{H}_{2}\boldsymbol{\Phi}\mathbf{H}_{1}}_{\text{IRS}} + \underbrace{\mathbf{G}}_{\text{LOS}}.
	\end{equation*}
	The IRS component of the channel can be decomposed in the paths through the individual elements as
	\begin{equation}
	\mathbf{H}_{2}\boldsymbol{\Phi}\mathbf{H}_{1}= 
	\Sigma_{i=1}^{N_S} \phi_i \mathbf{h}_{2,i}\mathbf{h}^T_{1,i}
	\end{equation}
	for an IRS with $N_S$ reflective elements. 
In the following we will evaluate the performance of this system using the Mean Square Error given by 
\begin{equation}
E\left[\|  \vs - (\mG + \Sigma_{i=1}^{N_S} \phi_i \mathbf{h}_{2,i}\mathbf{h}^T_{1,i})\mathbf{W}\vs) \|^2 \right]
\end{equation}
in which $\mathbf{W}$ is the linear precoder at the transmitter. Note that receiver noise is not considered in this utility function. 
In the following ,we assume that the elements of the IRS can adjust the phases of the impinging waves and may also be "turned off" such that they reflect any wave.
CSI is estimated using the following procedure.
\begin{enumerate}
\item
We leave all IRS elements \emph{on} and set the phase to a known value (e.g. $\phi_i = 1 $) to first find the full channel   $\mH_{Total}$(and then transmit this to the Tx). At this point, we will not have sufficient information to estimate the individual components of the IRS component of the channel $\mathbf{h}_{2,i}\mathbf{h}^T_{1,i}$ and thus we cannot optimize with respect to $\phi_i $.
\item
	In the next iteration of pilots, we turn off all elements in order to estimate the LOS matrix $\mG $.
\item
	Next, by turning on a specific IRS element and sending additional pilots, we can estimate 
	the path through a specific IRS element given by $\mathbf{h}_{2,i}\mathbf{h}^T_{1,i}$.
	Using this information, we can now use the expression 
	\begin{equation}
E\left[\|  \vs - (\mG +  \mH_1 + \Sigma_{i=1}^{1} \phi_i \mathbf{h}_{2,i}\mathbf{h}^T_{1,i})\mathbf{W}\vs  \|^2 \right]
\end{equation}
in which $\mH_{Total} =  \mH_1 + \Sigma_{i=1}^{1} \phi_i \mathbf{h}_{2,i}\mathbf{h}^T_{1,i}$
	\item
		Repeating the procedure above, we estimate additional elements $\mathbf{h}_{2,i}\mathbf{h}^T_{1,i}$ and perform the optimization over increasing number of phases.
	\item 
		We are investigating how this optimization enhances performance w.r.t the MSE as well a how the correlation in the channel may change this. 
		Note that once correlation is considered, the order in which the channels through each element $\mathbf{h}_{2,i}\mathbf{h}^T_{1,i}$ may impact the optimization performance .
\end{enumerate}
\section{Optimization Problem}
First, we rewrite the above utility function as
	\begin{equation}
E\left[ \trace\left((\vs - \mathbf{H}\mathbf{W}\vs)(\vs - \mathbf{H}\mathbf{W}\vs)^H \right)\right]
\end{equation}
Assuming $\mathbf{H}$ and $\mathbf{W}$ to be constant over a coherence period, the expectation applies only to $\vs \vs^H$. If we consider equal power allocation at the transmitter, this becomes
	\begin{equation}
\trace\left((\mI - \mathbf{H}\mathbf{W})(\mI - \mathbf{H}\mathbf{W})^H \right).
\end{equation}
After finding $K \leq N_S$ IRS element channels $\mathbf{h}_{2,i}\mathbf{h}^T_{1,i}$, the optimization problem with respect to the IRS phases is given by
	\begin{align}
	    \underset{\phase}{\text{minimize }}
	    & \; \trace\left((\mI - \mathbf{H}\mathbf{W})
	   (\mI - \mathbf{H}\mathbf{W})^H \right)
	     \\
	    \text{subject to  } \; &
	    | \phi_i | = 1  \forall i \in {1 \cdots	 K}
	\end{align}\label{mse_problem}
	which is further simplified into
		\begin{align}
	    \underset{\phase}{\text{minimize }}
	    & \; - \trace \left( \mathbf{H}\mathbf{W}\right) - \trace \left( \mathbf{H}\mathbf{W}\right)^* +
	    \trace\left(\mathbf{H}\mathbf{W} \mathbf{W}^H \mathbf{H}^H\right)
	     \\
	    \text{subject to  } \; &
	    | \phi_i | = 1  \forall i \in {1 \cdots	 K}
	\end{align}\label{mse_problem_full}
	with $\mathbf{H} = (\mG +  \mH_1 + \Sigma_{i=1}^{K} \phi_i \mathbf{h}_{2,i}\mathbf{h}^T_{1,i})$ (to fit on page).
	Expanding the above utility function and neglecting the terms without any phase coefficients, the cost function becomes
	\begin{equation}
	\Sigma_{i=1}^{N} \left(\prod_{k=1}^{M_i}\phi_{k} c_i \right ) +  \Sigma_{i=1}^{N} \left(\prod_{k=1}^{M_i}\phi_{k} c_i \right)^H.
	\end{equation}
	From this expression it can be seen that the complex component of each term will be canceled by its complex conjugate.
	One method to perform this optimization is to iterate over each element of the IRS and choose $\phi_{i}$ such that 
	the corresponding component of the sum is real and negative. In this case, each step of the algorithm solves the problem
			\begin{align}
	    \underset{\phi_k}{\text{minimize }}
	    & \; \phi_k c + \phi_k^* c^* + a 
	     \\
	    \text{subject to  } \; &
	    | \phi_i | = 1   \; \forall i \in {1 \cdots	 K}
	\end{align}\label{mse_problem_s}
	in which $a \in \reals$ and $c \in \complex$ so the result is simply to choose $\phi_k$ such that $\phi_k c$ is negative and real.

\section{Optimization Implementation}
	First, I considered the case in which an arbitrary linear precoding filter is used at the transmitter (in this case the matched filter to the channel when all IRS elements have phase $\phi_i = 1 $) and the MSE is minimized.
	In this case, each step of the algorithm is the same as Problem \eqref{mse_problem_s}.
	The results for a system using a matched filter (for the channel before IRS updates) at the transmitter are shown in \ref{MSE_opt} and indicate that increasing the number of elements at the IRS 
	can allow for improved optimization gains. 
	\begin{figure}[H]
	\includegraphics[width= 16cm,height = 13cm]{figures/MSE_optimized}
	  \caption{MSE with/without optimization versus number of IRS elements without receiver noise.}
	  	  \label{MSE_opt}
	\end{figure}
		
	\begin{figure}[H]
	\includegraphics[width= 16cm,height = 13cm]{figures/MSE_optimized_noise}
	  \caption{MSE with/without optimization versus number of IRS elements with receiver noise.}
	  	  \label{MSE_opt_noise}
	\end{figure}
Seems strange because for infinite SNR case, Matched filter should have optimal MSE right?
Next the same results are generated for the case in which the zero-forcing filter to the channel with all phase elements set to 0 is used. 
\section{Next Steps}
The first next step will be to include the impact of the phases on the choice of the beamformer matrix. To simplify the optimization the beamformer was fixed in the above results. 
\bibliography{bibliography}
\end{document}
