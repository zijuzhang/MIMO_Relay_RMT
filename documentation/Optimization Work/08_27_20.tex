\documentclass[12pt,a4paper]{report}
\usepackage[utf8]{inputenc}
\usepackage{amsmath}
\usepackage{amsfonts}
\usepackage{amssymb}
\usepackage[colorlinks=true,linkcolor=blue]{hyperref}
\usepackage{graphicx,tikz}
\usepackage{float}
\usetikzlibrary{positioning}
\input defs.tex
\bibliographystyle{ieeetr}
\graphicspath{ {./figures/} }

\title{Progress Report}
\author{Peter Hartig}

\begin{document}
\maketitle
\tableofcontents
\section{System Overview}
The communication system model is given by 
\begin{equation}
\vy = \mH_{Total} \mW \vs  + \vn
\end{equation}
with 
	\begin{equation*}
	\mH_{Total} = \underbrace{\mathbf{H}_{2}\boldsymbol{\Phi}\mathbf{H}_{1}}_{\text{IRS}} + \underbrace{\mathbf{G}}_{\text{LOS}}.
	\end{equation*}
	The IRS component of the channel can be decomposed in the paths through the individual elements as
	\begin{equation}
	\mathbf{H}_{2}\boldsymbol{\Phi}\mathbf{H}_{1}= 
	\Sigma_{i=1}^{N_S} \phi_i \mathbf{h}_{2,i}\mathbf{h}^T_{1,i}
	\end{equation}
	for an IRS with $N_S$ reflective elements. 
In the following, we will assume that a matched filter is used at the transmitter. Further we will evaluate the performance of the system using the Mean Square Error given by 
\begin{equation}
E\left[\|  \vs - (\mG + \Sigma_{i=1}^{N_S} \phi_i \mathbf{h}_{2,i}\mathbf{h}^T_{1,i})(\mG + \Sigma_{i=1}^{N_S} \phi_i \mathbf{h}_{2,i}\mathbf{h}^T_{1,i})^H\vs) \|^2 \right]
\end{equation}
in which we do not consider the noise at the receiver. 
In the following we assume that the elements of the IRS can not only adjust the phases of the impinging waves but may also be "turned off" such that they are not included in the channel.
In order to minimize the overhead of pilot acquisition, we estimate the CSE in the following manner. 
\begin{enumerate}
\item
We leave all IRS elements \emph{on} to first find the full matched filter at the Tx. Assuming that we estimate the entire channel $\mH_{Total}$, we will not have sufficient information to estimate the individual components of the IRS component of the channel $\mathbf{h}_{2,i}\mathbf{h}^T_{1,i}$ and thus we cannot optimize with respect to $\phi_i $.
\item
	In the next iteration of pilots, we turn off all elements in order to estimate the LOS matrix $\mG $.
\item
	Next, by turning on a specific IRS element (can we use patterns using the correlation?) and sending additional pilots, we can estimate 
	the path through a specific IRS element given by $\mathbf{h}_{2,i}\mathbf{h}^T_{1,i}$.
	Using this information, we can now use the expression 
	\begin{equation}
E\left[\|  \vs - (\mG +  \mH_1 + \Sigma_{i=1}^{1} \phi_i \mathbf{h}_{2,i}\mathbf{h}^T_{1,i})(\mG +  \mH_1 + \Sigma_{i=1}^{1} \phi_i \mathbf{h}_{2,i}\mathbf{h}^T_{1,i})\vs  \|^2 \right]
\end{equation}
	\item
		Repeating the procedure above, we estimate additional elements $\mathbf{h}_{2,i}\mathbf{h}^T_{1,i}$ and perform the optimization over increasing number of phases.
	\item 
		We are investigating how this optimization enhances performance w.r.t the MSE as well a how the correlation in the channel may change this. 
		Note that once correlation is considered, we may want to decide the phase of elements in close proximity whose channels we have not yet estimated.
\end{enumerate}
\section{Optimization Problem}
First, we rewrite the above utility function as
	\begin{equation}
E\left[ trace\left((\vs - \mathbf{H}\mathbf{F}\vs)(\vs - \mathbf{H}\mathbf{F}\vs)^H \right)\right]
\end{equation}
Assuming $\mathbf{H}$ and $\mathbf{F}$ to be constant over a coherence period, the expectation applies only to $\vs \vs^H$ to give
	\begin{equation}
trace\left((\vs - \mathbf{H}\mathbf{F}\vs)(\vs - \mathbf{H}\mathbf{F}\vs)^H \right).
\end{equation}
First we consider equal power allocation at the transmitter such that after finding the first $K \leq N_S$ IRS channels $\mathbf{h}_{2,i}\mathbf{h}^T_{1,i}$, the optimization problem
is given by
	\begin{align}
	    \underset{\phase}{\text{minimize }}
	    & \; trace\left((\mI - \mathbf{A}\mathbf{A}^H)
	   (\mI - \mathbf{A}\mathbf{A}^H)^H \right)
	     \\
	    \text{subject to  } \; &
	    | \phi_i | = 1  \forall i \in {1 \cdots	 K}
	\end{align}\label{mse_problem}
	with $\mathbf{A} = (\mG +  \mH_1 + \Sigma_{i=1}^{K} \phi_i \mathbf{h}_{2,i}\mathbf{h}^T_{1,i})$ (to fit on page).
	We classify the terms of the resulting polynomial into two types:
	cross types (e.g $(\phi_i \mathbf{h}_{2,i}\mathbf{h}^T_{1,i})(\phi_i \mathbf{h}_{2,i}\mathbf{h}^T_{1,i})^H$) and non-cross terms (e.g $\mG (\phi_i \mathbf{h}_{2,i}\mathbf{h}^T_{1,i})^H$). For non-cross terms, the phases will always cancel out ($\phi_i \phi_i^H = 1$). For the cross-terms, note that
	for each term in the polynomial, the hermetian of the term is also in the polynomial sum (e.g both $\mG (\phi_i \mathbf{h}_{2,i}\mathbf{h}^T_{1,i})^H$ and $ (\phi_i \mathbf{h}_{2,i}\mathbf{h}^T_{1,i}\mG^H$). Because $trace(\mH^H) = trace(\mH)^*$, the complex component of these terms will always cancel. In the case of minimizing the MSE, this means that we want to minimize the resulting real component for all of these cross terms. In order to investigate further, we decompose the problem further. Neglecting the non-cross terms and using the linearity of the $trace$ operator with respect to scaling, we can equivalently consider the problem 
	\begin{align}
	    \underset{\phase}{\text{maximize }}
	    & \; Img \left( \Sigma_{i=1}^{K} \Sigma_{j=i}^{K} c_{i,j} \phi_{i} \phi_{j}  \right)
	     \\
	    \text{subject to  } \; &
	    | \phi_i | = 1  \forall i \in {1 \cdots	 K}.
	\end{align}\label{mse_problem_clear}
	Note that because the polynomial in Problem \eqref{mse_problem} is the trace of a Hermetian product, only the imaginary component along the diagonal of each cross term is canceled by its corresponding Hermetian component in the sum. Replacing $ Img$ with a trigonometric function it is clear that this utility function is non-convex over the domain.
\section{Phase Optimization}
One potential method of performing this optimization is based on the ideas of coordinate descent in which a single variable is updated while all others are held constant.
From problem \eqref{mse_problem_clear} it is clear that if all $\phi_{j} \; j \neq i $ are held constant,$\phi_{j} \; j \neq i $can be chosen as a phase which fulfills the objective function. 
For a given step of this algorithm in which $\phi_{i}$ the update is determined by a sum $\phi_{i}a + b$ and $\phi_{i}$ is chosen so at to maximize the 
complex component of the sum (which we will be canceled out by its complex conjugate.
The numerical results below indicate the impact of this optimization technique on the MSE of the IRS system described above. 
\bibliography{bibliography}
\end{document}
