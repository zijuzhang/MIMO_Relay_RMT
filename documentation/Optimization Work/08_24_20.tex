\documentclass[12pt,a4paper]{report}
\usepackage[utf8]{inputenc}
\usepackage{amsmath}
\usepackage{amsfonts}
\usepackage{amssymb}
\usepackage[colorlinks=true,linkcolor=blue]{hyperref}
\usepackage{graphicx,tikz}
\usepackage{float}
\usetikzlibrary{positioning}
\input defs.tex
\bibliographystyle{ieeetr}
\graphicspath{ {./figures/} }

\title{Progress Report}
\author{Peter Hartig}

\begin{document}
\maketitle
\tableofcontents

\section{CSI Estimation Setup}
The system model is given by 
\begin{equation}
\vy = \mH_{Total} \mW \vs  + \vn
\end{equation}
with 
	\begin{equation*}
	\mH_{Total} = \underbrace{\mathbf{H}_{2}\boldsymbol{\Phi}\mathbf{H}_{1}}_{\text{IRS}} + \underbrace{\mathbf{G}}_{\text{LOS}},
	\end{equation*}
%		\mH_{Total} = \mathbf{R}_{R}^{\frac{1}{2}}(\underbrace{\mathbf{H}_{2}\boldsymbol{\Phi}\mathbf{R}_{S}^{\frac{1}{2}}\mathbf{H}_{1}}_{\text{IRS}} + \underbrace{\mathbf{G}}_{\text{LOS}})\mathbf{R}_{T}^{\frac{1}{2}}.
	$\mW$ as a linear precoder and $\vs$ as the vector of transmit symbols.
	\par
	CSI is estimated for use in the choice of $\mW$ in following procedure. 
	In  , 
	\begin{enumerate}
	\item
		With all IRS surfaces turned off, the LOS channel is estimated as $\hat{\mG}$ by 
		transmitting a matrix of pilots $\mathbf{P}$ (with orthogonal rows)  and using the least squares solution with the matrix $\mathbf{Y}$ \cite{nadeem2019intelligent}.
		The resulting solution is given by
		\begin{equation}
			\hat{\mG} = [(\mathbf{P} \mathbf{P}^H)^{-1} \mathbf{P} \mathbf{Y}^H]^H
		\end{equation}
		With the received signal
		\begin{equation}
		 \mathbf{Y} = \mH_{Total} \mathbf{P} + \mathbf{N}
		\end{equation}
			the resulting CSI estimate is 
					\begin{equation}
		 \hat{\mH}_{Total} = \mH_{Total} + \mathbf{P}\mathbf{N}
		\end{equation}
		in which the elements of $\mathbf{P}\mathbf{N}$ have the same power and orthogonality as $\mathbf{N}$.
	\item 	
		In the next time interval, we turn on a single IRS element and transmit the pilot sequence again such that the received signal is 
			\begin{equation*}
			\mathbf{Y} = (\mG + \mathbf{h}_{2,1}\mathbf{h}^T_{1,1}  )\mathbf{P}  + \vn
			\end{equation*}
			where $\mathbf{h}_{2,1}$ is the 1st \emph{column} of $ \mathbf{H}_{2}$ and $\mathbf{h}_{1,1}^T$  is the 1st \emph{row} of $\mathbf{H}_{1}$.
			By subtracting off the known contribution from the LOS path, $\hat{\mG}\mathbf{P}$, from the received signal we have 
						\begin{equation*}
						\mathbf{Y} = (\mathbf{h}_{2,1}\mathbf{h}^T_{1,1})\mathbf{P}  + \vn
						\end{equation*}
			from which we can estimate the rank 1 matrix $\mathbf{h}_{2,1}\mathbf{h}^T_{1,1}$.
			Note that the matrix $\mathbf{P} $ need only have 2 columns in order to have sufficient data to estimate the rank 1 matrix.
			\par
			At this point, the estimate for the channel contains
				\begin{equation*}
				\hat{\mH_{Total}} = \hat{\mG} + \mathbf{h}_{2,1}\mathbf{h}^T_{1,1}.
				\end{equation*}
				Note that we can change the resulting sum of the matrices using the phase of components of the IRS matrix using  $\phi_i \hat{\mathbf{h}}_{2,1}\hat{\mathbf{h}^T_{1,1}} $.
	\item 
		Iterating this procedure allows us to estimate both paths of the channel. 
		such that for a system with $N_S$ IRS elements, the channel estimate after $N_S$ iterations of the above procedure yields the channel 
				\begin{equation*}
				\hat{\mH}_{Total} = \hat{\mG} + \Sigma_{i=1}^{N_S} \phi_i \hat{\mathbf{h}}_{2,i}\hat{\mathbf{h}^T_{1,i}} 
				\end{equation*}
				where $\phi_i$ is the phases selected by the IRS controller for element $i$. 
		From this representation, it is clear that if in the worst case, the matrices $\hat{\mathbf{h}}_{2,i}\hat{\mathbf{h}^T_{1,i}}$ were completely correlated
		then we would want these components to be added together coherently, implying that all phases $\phi_i$ should be the same.
	\end{enumerate}
	
\section{Optimization}
Using a matched filter at the transmitter with CSI found using the above procedure, the MSE problem for selecting the phases for each element is now given by
	\begin{subequations}
	\label{optim}
	\begin{align}
	    \underset{\phase}{\text{minimize }}
	    & \; E\left[\|  \vs - (\mG + \Sigma_{i=1}^{N_S} \phi_i \mathbf{h}_{2,i}\mathbf{h}^T_{1,i})(\hat{\mG} + \Sigma_{i=1}^{N_S} \phi_i \hat{\mathbf{h}}_{2,i}\hat{\mathbf{h}^T_{1,i}})^H\vs) \|^2 \right]\\
	    \text{subject to  } \; &
	    | \phi_i | = 1 & \forall i \in {1 \cdots	 N_S}
	\end{align}
	\end{subequations}	
	Assuming the CSI is known at the transmitter such that the expectation applies only to the transmit symbols, this problem becomes
	\begin{subequations}
	\label{optim}
	\begin{align}
	    \underset{\phase}{\text{minimize }}
	    & \; trace\left((\mI - \mH\hat{\mH}_{Total}^H) \mathbf{Q}_{\vs\vs} 
	    (\mI - \mH\hat{\mH}_{Total}) ^H\right)
	     \\
	    \text{subject to  } \; &
	    | \phi_i | = 1 & \forall i \in {1 \cdots	 N_S}
	\end{align}
	\end{subequations}		
	
	To Discuss
	\begin{enumerate}
	\item
		Not convex in this formulation. The above utility can be seen as a sum such as 
		\begin{equation}
		 x_1x_2 c_1 + x_2x_3c_2 + x_1x_3c_3 + \cdots + x_1x_2x_3c_i \cdots
		\end{equation}
		in which the coefficients $c_i \in \mathcal{C}$ so that the result is non-convex (Should be a sum of concave and convex terms though).
		Clearly the constraint $ | \phi_i | = 1$ is also non-convex
	\item
		Need to clarify the order of estimation because using the above method, we wouldn't be able to have a precoder based on the total channel until all $N_S$ steps were 					complete.	Should the order of estimation be reversed (i.e. start with all elements turned on and then turn them off 1 by 1?).
	\item
		How to connect intuition of the resulting metric from MSE with a matched with a performance metric like SER?
		Particularly when the channels are correlated, this interference will dominate performance.
%	\item
%		Can we factor out $\vs$ and just consider how close we can get the combined filters to be the identity?
	\end{enumerate}
\bibliography{bibliography}
\end{document}
