\documentclass[12pt,a4paper]{report}
\usepackage[utf8]{inputenc}
\usepackage{amsmath}
\usepackage{amsfonts}
\usepackage{amssymb}
\usepackage[colorlinks=true,linkcolor=blue]{hyperref}
\usepackage{graphicx,tikz}
\usepackage{float}
\usetikzlibrary{positioning}
\input defs.tex
\bibliographystyle{ieeetr}
\graphicspath{ {./figures/} }

\title{Progress Report}
\author{Peter Hartig}

\begin{document}
\maketitle
\tableofcontents
\section{Points from Last Meeting}
\begin{enumerate}
In the first iteration of CSI collection, 1 pilot is 
\item{Issue with Expectation Over Elements of Channel:} \par
	The MSE Objective function with deterministic $\mW $ is 
	\begin{equation}
	\Expect\left[\trace\left((\mathbf{s} - \mH \mW \mathbf{s} )  (\mathbf{s} - \mH \mW \mathbf{s} ) ^H  \right)\right]
	\end{equation}
		\begin{equation}
	\trace\left(\Expect\left[ (\mI - \mH \mW  )^H  (\mI - \mH \mW  ) \right] \Expect\left[ \mathbf{s}\mathbf{s}^H\right]\right)
	\end{equation}
	Assuming equal power allocation this becomes
	\begin{equation}
	\trace\left(\Expect\left[ (\mI - \mH \mW  )^H  (\mI - \mH \mW  )  \right]\right).
	\end{equation}
	Because the beamformer, $\mathbf{W}$, is deterministic, the optimization problem becomes
		\begin{subequations}
	\label{mse_problem_total}
	\begin{align}
	   	    \underset{\boldsymbol{\phi}}{\text{minimize }}
	    & \; - \trace \left( \Expect\left[\mathbf{H} \right] \mathbf{W} \right) - \trace \left( \Expect\left[\mathbf{H} \right]  \mathbf{W}\right)^* +
	    \trace\left(\Expect\left[\mathbf{H}^H \mathbf{H} \right] \mathbf{W} \mathbf{W}^H \right)
	     \\
	    \text{subject to  } \; &
	    | \phi_i | = 1   \; \forall i \in {1 \cdots	 K}
	\end{align}
	\end{subequations}	
	Assuming channel components corresponding to the IRS path to be zero-mean (reasonable), this becomes 
		\begin{subequations}
	\label{mse_problem_total}
	\begin{align}
	   	    \underset{\boldsymbol{\phi}}{\text{minimize }}
	    & \; \trace\left(\Expect\left[\mathbf{H}^H \mathbf{H} \right] \mathbf{W} \mathbf{W}^H \right)
	     \\
	    \text{subject to  } \; &
	    | \phi_i | = 1   \; \forall i \in {1 \cdots	 K}
	\end{align}
	\end{subequations}	
	In general we assume the channels through the elements of the IRS are correlated and in this case we use the exponential model.
	Now we decompose the channel $\mH$ into the individual paths through the IRS  $\Sigma_{i=1}^{N_S} \phi_i \mR_{R}^{\frac{1}{2}}\mathbf{h}_{2,i}\mathbf{h}^T_{1,i}
	\mR_{T}^{\frac{1}{2}}$. Expanding the utility function in \ref{mse_problem_total} shows that all $\phi_i$  from the channel $\mH$ cancel out from the non-cross terms. The resulting utility function only contains cross terms of the form 
	\begin{equation}
		\trace\left(\Expect\left[ \phi_i \mR_{R}^{\frac{1}{2}}\mathbf{h}_{2,i}\mathbf{h}^T_{1,i}
	\mR_{T}^{\frac{1}{2}} (\phi_j \mR_{R}^{\frac{1}{2}}\mathbf{h}_{2,j}\mathbf{h}^T_{1,j}
	\mR_{T}^{\frac{1}{2}})^H \right] \right)
	\end{equation}
	in which $i \neq j$, or equivalently
		\begin{equation}
		\phi_j ^* \phi_i  \trace\left(\mR_{R} \Expect\left[ \mathbf{h}_{2,i}\mathbf{h}^T_{1,i}
	\mR_{T} (\mathbf{h}_{2,j}\mathbf{h}^T_{1,j})^H\right] \right).
	\end{equation}
	Assuming all elements of the random vectors $\mathbf{h}_{2,i}$ and $\mathbf{h}^T_{1,i}$ to be independent (also for $i \neq j$) these cross terms will also have expected value of zero. As a result the final expression does not include any $\phi_i$ meaning that without having knowledge of the actual channels, the MSE cannot be optimized with respect to the phases.
\item{Deterministic Elements of Channel:} \par
	In the above problem, if the transmitter has access to the channel state information for the channels through the IRS, the optimization problem from the last report can be used with results seen in figure \ref{MSE_opt_noise}
\end{enumerate}
\end{document}
