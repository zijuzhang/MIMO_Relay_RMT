\documentclass[12pt,a4paper]{report}
\usepackage[utf8]{inputenc}
\usepackage{amsmath}
\usepackage{amsfonts}
\usepackage{amssymb}
\usepackage[colorlinks=true,linkcolor=blue]{hyperref}
\usepackage{graphicx,tikz}
\usepackage{float}
\usetikzlibrary{positioning}
\input defs.tex
\bibliographystyle{ieeetr}
\graphicspath{ {./figures/} }

\title{Progress Report}
\author{Peter Hartig}

\begin{document}
\maketitle
\tableofcontents

\section{Numerical Results}
I use the system model given by 
\begin{equation}
\vy = \mH_{Total} \mW \vs  + \vn
\end{equation}
with 
	\begin{equation}
	\mH_{Total} = \mathbf{R}_{R}^{\frac{1}{2}}(\underbrace{\mathbf{H}_{2}\boldsymbol{\Phi}\mathbf{R}_{S}^{\frac{1}{2}}\mathbf{H}_{1}}_{\text{IRS}} + \underbrace{\mathbf{G}}_{\text{LOS}})\mathbf{R}_{T}^{\frac{1}{2}}.
	\end{equation}
Elements of $\mH_{2},\mH_{1}$ and $\mG$ are normalized so as to have a normalized trace of 1 and $\mathbf{R}_{k}$ represents the exponential correlation model with elements $r_{i,j}=\rho^{i-j}$.
\par
The following numerical results evaluate the MSE for the channel above with $s_k \in \mathcal{NC}(0,\frac{1}{N_T})$, $n_k \in \mathcal{NC}(0,\frac{1}{N_R})$
$N_T = 10$ (transmitters), $N_R = 10$ (receivers) and $N_S = 50$ (IRS surfaces).
For each channel realization, I evaluate the MSE for the matched filter and zero-forcing filter using $\phase = e^{j\theta}\mI$ for 10 uniform spaced values of $\theta \in [0, 2\pi]$. I then find the mean and variance of the MSE over the 10 values of $\theta$. I repeat and average this across 1000 channel realizations. This numerical test is performed for different values of $\rho$ in the exponential correlation model. The results in Figure \ref{MSE_correlation} indicate that the variance of the MSE over different values of the phase matrix parameter, $\theta$,  is significant and increases as the correlation parameter, $\rho$ approaches 1.
		\begin{figure}[H]
	\includegraphics[width= 16cm,height = 13cm]{figures/over_10_thetas}
	  \caption{Comparison MSE over different amounts of correlation.}
	  	  \label{MSE_correlation}
	\end{figure}

Based on the measurement data collected in \cite{martin2000multiple}, the correlation between antennas can reach $.5$ even in a $4 \times 4 $ MIMO system without an IRS. 
The IRS should introduce the potential for additional correlation in the system so seems reasonable to allow for $\rho = .5$ (or maybe even higher). I have not been able to find any correlation measurement data from larger systems or with an IRS.

\section{Next Steps: Methods for Phase Optimization}
The above results motivate the investigation of phase optimization to reduce the MSE. A couple of previous works looking at such optimization are now reviewed. 
\begin{enumerate}
\item
	Method from \cite{huang2018achievable}.
\begin{itemize}
\item
	No LOS or correlation component in the channel model.
\item
	Algorithm requires full CSIT at IRS.
\item 
	Assumes a ZF precoder in order to simplify the sum-capacity objective function.
\item
	Uses an alternating algorithm approach with two main steps
	\begin{enumerate}
	\item 
		Minimize transmission power w.r.t $\theta$.
	\item
		Maximize rate with respect transmission power of each transmit antenna.
	\end{enumerate}
\end{itemize}

\item 
	Method from \cite{kammoun2020asymptotic}.
	\begin{itemize}
		\item 
			Algorithm only requires channel statistics for the IRS update.
		\item
			Unlike \cite{huang2018achievable}, optimization occurs over a beamforming matrix and then optimizes powers and phases. 
		\item
			Channel uses a correlation model sort of like the exponential model except for that I don't think the matrix is normalized for power.
		\item
			Uses max-min $\text{SINR}_K$ as objective.
		\item
			For case with single rank channel (fully correlated channel) they use constant phase at all elements. Different phases are used for full-rank case.
		\item 
			Figure 9. of this work is particular interesting as it provides evidence that their algorithm can significantly increase the minimum user rate when compared
			to another method of selecting the phases at the IRS. Note that here they allow the number of reflecting surfaces to be quite high but keep $N_R = N_T = 8$.
	\end{itemize}
\end{enumerate}
Assuming a matched filter at the transmitter the MSE problem for the choice of the IRS phases becomes
		\begin{subequations}
	\label{optim}
	\begin{align}
	    \underset{\phase}{\text{minimize }}
	    & \; \| \mH_{Total}(\mH_{Total}^H\vs + \vn) - \vs \| \label{potential_game} \\
	    \text{subject to  } \; &
	    | \theta_i | = 1 & \forall i \in {1 \cdots	 N_S}
		\label{pos_power_const}
	\end{align}
	\end{subequations}
	Is the utility function at least convex?
	Do I need to include the power constraint here? Probably
\bibliography{bibliography}
\end{document}
