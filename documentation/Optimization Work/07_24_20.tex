\documentclass[12pt,a4paper]{report}
\usepackage[utf8]{inputenc}
\usepackage{amsmath}
\usepackage{amsfonts}
\usepackage{amssymb}
\usepackage[colorlinks=true,linkcolor=blue]{hyperref}
\usepackage{graphicx,tikz}
\usepackage{float}
\usetikzlibrary{positioning}
\input defs.tex
\bibliographystyle{ieeetr}
\graphicspath{ {./figures/} }

\title{Progress Report}
\author{Peter Hartig}

\begin{document}
\maketitle
\tableofcontents

\section{Numerical Results}
I use the system model given by 
\begin{equation}
\vy = \mH_{Total} \mW \vs  + \vn
\end{equation}
with 

	\begin{equation}
	\mathbf{H}_{Total} = \mathbf{R}_{R}^{\frac{1}{2}}(\underbrace{\mathbf{H}_{2}\boldsymbol{\Phi}\mathbf{R}_{S}^{\frac{1}{2}}\mathbf{H}_{1}}_{\text{IRS}} + \underbrace{\mathbf{G}}_{\text{LOS}})\mathbf{R}_{T}^{\frac{1}{2}}.
	\end{equation}

The following numerical results evaluate the MSE the channel above with $s_k \in \mathcal{NC}(0,\frac{1}{N_R})$, $n_k \in \mathcal{NC}(0,\frac{1}{N_R})$.
For each channel realization, I evaluate the MSE for the matched filter and zero-forcing filter using $\phase = e^{j\theta}\mI$ for 10 uniform spaced values of $\theta \in [0, 2\pi]$. Then I collect the mean and variance of the 10 MSEs for the channel realization. I repeat and average across 1000 channel realizations. I perform this same analysis across different values of $\rho$ in the exponential correlation model (i.e.  $r_{i,j}=\rho^(i-j)$. The results in Figure \ref{MSE_correlation} indicate that the variance of the MSE over different values of the phase matrix parameter, $\theta$,  can be quite large. In particular when the correlation parameter, $\rho$ is close to 1.
		\begin{figure}[H]
	\includegraphics[width= 15cm,height = 10cm]{figures/over_10_thetas}
	  \caption{Comparison MSE over different amounts of correlation.}
	  	  \label{MSE_correlation}
	\end{figure}

Based on the measurement data collected in \cite{martin2000multiple}, the correlation between antennas can reach $.5$ even in a, $4 \times 4 $ MIMO systems. 
Since the IRS introduces the potential for additional correlation it seems reasonable to allow for $\rho = .5$.

\section{Next Steps: Methods for Phase Optimization}
\begin{enumerate}
\item
	Method from \cite{huang2018achievable}.
\begin{itemize}
\item
	No LOS or correlation considered in system model.
\item
	Algorithm requires full CSIT at IRS
\item 
	Assumes a ZF precoder in order to simplify sum-capacity objective function.
\item
	Alternating algorithm approach with two main steps
	\begin{enumerate}
	\item 
		Minimize transmission power w.r.t phases
	\item
		Maximize rate wrt power
	\end{enumerate}
\end{itemize}

\item 
	Method from \cite{kammoun2020asymptotic}.
	\begin{itemize}
		\item 
			Algorithm only requires channel statistics for the RIS update
		\item
			System model includes correlation.
		\item
			Does transmitter need perfect CSI or can they make do with just knowing the composite channel.
		\item
			Uses max-min SINR as objective
		\item
			Optimizes the beamformer, the powers and RIS (continuous phases allowed)
		\item
			Question: \cite[Section III ]{kammoun2020asymptotic} makes a claim that for a single rank channel the RIS only offers a "substantial" benefit if there
			is a single user.  This makes sense for a Decision-Feedback equalization sense but it is still not clear to me why the benefit
	\end{itemize}

\end{enumerate}
\bibliography{bibliography}
\end{document}
