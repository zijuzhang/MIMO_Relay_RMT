\documentclass[12pt,a4paper]{report}
\usepackage[utf8]{inputenc}
\usepackage{amsmath}
\usepackage{amsfonts}
\usepackage{amssymb}
\usepackage[colorlinks=true,linkcolor=blue]{hyperref}
\usepackage{graphicx,tikz}
\usepackage{float}
\usetikzlibrary{positioning}
\input defs.tex
\bibliographystyle{ieeetr}
\graphicspath{ {./figures/} }

\title{Progress Report}
\author{Peter Hartig}

\begin{document}
\maketitle
\tableofcontents
\section{Points from Last Meeting}
\begin{enumerate}
\item{High MSE Issue:} \par
	As discussed with was a normalization error. Now I normalize the transmit vector $\vs$ to have total power 1. I normalize the noise vector $\vn$ in the same way.
	Finally, for the linear beamformer $\mathbf{W}$, in the case of the matched filter, I normalize each column using $\frac{\mathbf{w}_i}{\|\mathbf{w}_i\|^2}$ so that
	$\Expect\left[\mH\mW\right] = \mI$.
	
	\begin{figure}[H]
	\includegraphics[width= 16cm,height = 13cm]{figures/Corrected_MSE}
	  \caption{MSE with/without optimization versus number of IRS elements with receiver noise.}
	  	  \label{MSE_opt_noise}
	\end{figure}	
	
\item{Expectation Over Elements of Channel:} \par
For the MSE Objective function with deterministic $\mW $ is 
	\begin{equation}
	\Expect\left[\trace\left((\mathbf{s} - \mH \mW \mathbf{s} )  (\mathbf{s} - \mH \mW \mathbf{s} ) ^H  \right)\right]
	\end{equation}
		\begin{equation}
	\trace\left(\Expect\left[ (\mI - \mH \mW  )  (\mI - \mH \mW  ) ^H \right] \Expect\left[ \mathbf{s}\mathbf{s}^H\right]\right)
	\end{equation}
	Assuming equal power allocation here since matrix size are relatively large this becomes
	\begin{equation}
	\trace\left(\Expect\left[ (\mI - \mH \mW  )  (\mI - \mH \mW  ) ^H \right]\right).
	\end{equation}
	Because the beamformer, $\mathbf{W}$, is deterministic at this point, the optimization problem becomes
		\begin{subequations}
	\label{mse_problem_total}
	\begin{align}
	   	    \underset{\boldsymbol{\phi}}{\text{minimize }}
	    & \; - \trace \left( \Expect\left[\mathbf{H} \right] \mathbf{W} \right) - \trace \left( \Expect\left[\mathbf{H} \right]  \mathbf{W}\right)^* +
	    \trace\left(\Expect\left[\mathbf{H}^H \mathbf{H} \right] \mathbf{W} \mathbf{W}^H \right)
	     \\
	    \text{subject to  } \; &
	    | \phi_i | = 1   \; \forall i \in {1 \cdots	 K}
	\end{align}
	\end{subequations}	
	Assuming channel components corresponding to the IRS path to be zero-mean, this becomes 
		\begin{subequations}
	\label{mse_problem_total}
	\begin{align}
	   	    \underset{\boldsymbol{\phi}}{\text{minimize }}
	    & \; \trace\left(\Expect\left[\mathbf{H}^H \mathbf{H} \right] \mathbf{W} \mathbf{W}^H \right)
	     \\
	    \text{subject to  } \; &
	    | \phi_i | = 1   \; \forall i \in {1 \cdots	 K}
	\end{align}
	\end{subequations}	
	In general we assume the channels through the elements of the IRS are correlated and in this case we use the exponential model.
	Now we decompose the channel $\mH$ into the individual paths through the IRS  $\Sigma_{i=1}^{N_S} \phi_i \mR_{R}^{\frac{1}{2}}\mathbf{h}_{2,i}\mathbf{h}^T_{1,i}
	\mR_{T}^{\frac{1}{2}}$. Expanding the utility function in \ref{mse_problem_total} shows that all $\phi_i$ cancel out from any non-cross term. The resulting utility function only contains cross terms of the form 
	\begin{equation}
		\trace\left(\Expect\left[ \phi_i \mR_{R}^{\frac{1}{2}}\mathbf{h}_{2,i}\mathbf{h}^T_{1,i}
	\mR_{T}^{\frac{1}{2}} (\phi_i \mR_{R}^{\frac{1}{2}}\mathbf{h}_{2,j}\mathbf{h}^T_{1,j}
	\mR_{T}^{\frac{1}{2}})\right] \right)
	\end{equation}
	or equivalently
		\begin{equation}
		\phi_j ^* \phi_i  \trace\left(\mR_{R} \Expect\left[ \mathbf{h}_{2,i}\mathbf{h}^T_{1,i}
	\mR_{T} (\mathbf{h}_{2,j}\mathbf{h}^T_{1,j})^H\right] \right).
	\end{equation}
	Assuming the all elements of $\mathbf{h}_{2,i}$ and $\mathbf{h}^T_{1,i}$ to be independent, these cross terms will also have expected value of zero and 
	thus the final expression does not include any $\phi_i$.
\item{Deterministic Elements of Channel:} \par
	In the above problem, if the transmitter has access to the channel state information for the channels through the IRS, the optimization problem from the last report can be used with results seen in figure \ref{MSE_opt_noise}
\end{enumerate}
\end{document}
