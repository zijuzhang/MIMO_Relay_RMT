\documentclass[10pt,tgadventor, onlymath]{beamer}

\usepackage{graphicx,amsmath,amssymb,tikz,psfrag,neuralnetwork, fontawesome, xcolor}

%\input defs.tex
\graphicspath{ {./figures/} }
\input defs.tex

%% formatting

\mode<presentation>
{
\usetheme{default}
\usetheme{Berlin}

\usecolortheme{seahorse}
}
\setbeamertemplate{navigation symbols}{}
\usecolortheme[rgb={0.03,0.28,0.59}]{structure}
\setbeamertemplate{itemize subitem}{--}
\setbeamertemplate{frametitle} {
	\begin{center}
	  {\large\bf \insertframetitle}
	\end{center}
}

\newcommand\footlineon{
  \setbeamertemplate{footline} {
    \begin{beamercolorbox}[ht=2.5ex,dp=1.125ex,leftskip=.8cm,rightskip=.6cm]{structure}
      \footnotesize \insertsection
      \hfill
      {\insertframenumber}
    \end{beamercolorbox}
    \vskip 0.45cm
  }
}
\footlineon

\AtBeginSection[] 
{ 
	\begin{frame}<beamer> 
		\frametitle{Outline} 
		\tableofcontents[currentsection,currentsubsection] 
	\end{frame} 
} 


\tikzstyle{state}=[shape=circle,draw=blue!30,fill=blue!10]
\tikzstyle{observation}=[shape=rectangle,draw=orange!30,fill=orange!10]
\tikzstyle{lightedge}=[<-, dashed]
\tikzstyle{mainstate}=[state, thick]
\tikzstyle{mainedge}=[<-, thick]
\tikzstyle{block} = [draw,rectangle,thick,minimum height=2em,minimum width=2em]
\tikzstyle{sum} = [draw,circle,inner sep=0mm,minimum size=2mm]
\tikzstyle{connector} = [->,thick]
\tikzstyle{line} = [thick]
\tikzstyle{branch} = [circle,inner sep=0pt,minimum size=1mm,fill=black,draw=black]
\tikzstyle{guide} = []
\tikzstyle{snakeline} = [connector, decorate, decoration={pre length=0.2cm,
                         post length=0.2cm, snake, amplitude=.4mm,
                         segment length=2mm},thick, magenta, ->]



%% begin presentation

\title{\large \bfseries Asymptotic Capacity of Systems with Intelligent Reflective Surfaces}
\author{Peter Hartig \\ \and Supervisor: Saba Asaad}

\date{\today}

\begin{document}

\frame{
\thispagestyle{empty}
\titlepage
}

\section{Background}
\subsection{MIMO Channel}

\begin{frame}
\frametitle{Massive MIMO}
				\centering
		\begin{figure}
		\includegraphics[scale=.18]{Pictures/mimo_antenna}
	\end{figure}
	\cite{argos2020}
\end{frame}

\begin{frame}
\frametitle{Simplifying the MIMO Channel: OFDM}
Assuming flat-fading channels between each transmit and receive antenna.
	\begin{itemize}
		\item 			
			Interpret each symbol to be the complex coefficient in frequency of a specific complex sinusoid (a sub-carrier). 
		\item
			These sinusoids pass through a channel whose frequency response is flat.
		\item
			By considering both the channel and transmit signals in the frequency domain for each "sub-carrier" in OFDM, we have a system given by
			\begin{equation}
				\vy = \mH \vx.
			\end{equation}
		\item 
			Forms of OFDM are defined for wireless protocols in 5G and 802.11
	\end{itemize}
\end{frame}

\begin{frame}
\frametitle{Channel Correlation}
	\begin{itemize}
		\item 
			From OFDM perspective this implies correlation between the same sub-carrier at different antennas.
		\item 
			FIR filters can represent some relevant types of correlation.
			Toeplitz can be modeled asymptotically as circulant matrices. AED of circulant matrices is DFT of 1st row
		\item
			The exponential correlation model is one such FIR filter with elements $c_{i,j} = \rho^{i-j} $ with $ 0 \leq \rho \leq 1$.
		\item 
			Measurement results for similar models show that $\rho$ can take on values up to $.6$ in $x$ GHz (a band relevant to 5G)
	\end{itemize}
\end{frame}

\begin{frame}
\frametitle{Channel Normalization}
\begin{itemize}
\item
	We want to analyze the channel behavior as $N \rightarrow \infty$ without sending capacity to infinity. 
\begin{itemize}
\item
	Normalize the power of the transmitted symbols and channel gains such that $\Expect[\vy^H \vy ] = 1$.
	Enforcing that $\Expect[\vx^H \vx ] = \frac{1}{N}\mathbf{I}$ and $\Expect[\|h_{i,j}\|^2_2 ]= \frac{1}{M}$ for $\mH \in \complex^{M \times N} $ satisfies this.
\end{itemize}
\item
	The popular result that MIMO capacity grows as $\min{N_T, N_R}$ does not include such a normalization. Which type of gain is now eliminated?
\item
	Allows for a matrix to have a converging eigenvalue distribution despite $N, M \rightarrow \infty$.
\end{itemize}
\end{frame}

\subsection{IRS Channel}
\begin{frame}
\frametitle{Relay vs. Intelligent Reflective Surface}
\begin{columns}
\begin{column}{0.5\linewidth}
\centering 
	\underline{Relay: Amplify and Forward}
	\\
	\begin{equation*}
	\mathbf{y} = (\mathbf{H}_2\underbrace{\mathbf{F}}_{\text{Relay}}\mathbf{H}_1 + \underbrace{\mathbf{G}}_{\text{LOS}})\mathbf{x}
	\end{equation*}
	
	\begin{itemize}
	\item 
		$\mathbf{F}$ is diagonal
	\item 
		Requires expensive RF hardware
	\item 
		Can often ignore line of sight
	\item 
		Typically half-duplex
	\end{itemize}
\end{column}
\begin{column}{0.5\linewidth}
\centering 

	\underline{IRS}
	\\
	\begin{equation*}
	\mathbf{y} = (\mathbf{H}_2\underbrace{\boldsymbol{\Phi}}_{\text{IRS}}\mathbf{H}_1 + \underbrace{\mathbf{G}}_{\text{LOS}})\mathbf{x}
	\end{equation*} 
	\begin{itemize}
	\item 
		$\boldsymbol{\Phi}$ is diagonal with $| \phi_i | =1$
	\item 
		No RF chain required (low cost)
	\item
		Shorter distances $\rightarrow$ keep LOS
	\item 
		Inherently full-duplex
	\end{itemize}
\end{column}
\end{columns}

\end{frame}




\begin{frame}
\frametitle{System Model}
\centering
\begin{equation*}
	\mathbf{y} = (\mathbf{H}_2\boldsymbol{\Phi}\mathbf{H}+ \mathbf{G})\mathbf{x}
\end{equation*}
with 
$\mathbf{H}_{1}\in \mathbb{C}_{S \times T},\mathbf{H}_{2} \in \mathbb{C}_{R \times S}, \mathbf{G} \in \mathbb{C}_{R \times T}$ and $\boldsymbol{\Phi}$ as $S \times S$ a diagonal matrix with $| \phi_i | =1$.

	\begin{figure}
		\centering
		\includegraphics[scale=.3]{irs_correct}
	\end{figure}
\end{frame}

\begin{frame}
\frametitle{Channel Normalization}
	\begin{itemize}
		\item 
			If transmit power is held constant as $N \rightarrow \infty$, the linear increase in capacity of the MIMO system depends on constant channel statistics.
		\item 
			In order to have a converging AED, we need to normalize the power of the channel.
		\item
			Show how can these results extend to the case without statistics changing? (is this even relevant?)
	\end{itemize}
\end{frame}

\section{Goals}
\begin{frame}
\frametitle{Project Goals}
\begin{enumerate}
\item
	Generalize capacity analysis for a channel model relevant to the IRS.
\item 
	Optimize phase matrix at the IRS (when relevant).
	\begin{itemize}
	\item
		Investigate the cases in which phase optimization is worthwhile. 
	\end{itemize}
\item
	* Time permitting - Consider secrecy capacity of IRS system (phase optimization may be relevant for some cases to increase secrecy capacity).
\item 
	Confirm findings with numerical results throughout.
\end{enumerate}
\end{frame}
\section{Current Status}


\begin{frame}
\frametitle{Channel Capacity}
\begin{equation*}
	\mathbf{y} = (\mathbf{H}_2\boldsymbol{\Phi}\mathbf{H}_1 + \mathbf{G})\mathbf{x}
\end{equation*}

With $\mathbf{x}\mathbf{x}^H = \mathbf{I}$ and fixed $\boldsymbol{\Phi}$, capacity is given by 
\begin{equation*}\label{capacity}
\Expect\left[\Log\left(|\mathbf{I}_{R}+\frac{P_{\text{Total}}}{T \sigma_n}[\mathbf{H}_{2}\boldsymbol{\Phi}\mathbf{H}_{1} + \mathbf{G}][\mathbf{H}_{2}\boldsymbol{\Phi}\mathbf{H}_{1} + \mathbf{G}]^H|\right)\right].
\end{equation*}
Using 
\begin{equation*}
\mathbf{H}_{\text{Total}} = \mathbf{H}_{2}\boldsymbol{\Phi}\mathbf{H}_{1} + \mathbf{G}
\end{equation*}
and simplifying, \eqref{capacity} becomes
\begin{equation*}
N \int_0^{\infty}\Log\left(1+\frac{P_{\text{Total}}}{T \sigma_n} \lambda_{\mathbf{H}_{\text{Total}}\mathbf{H}_{\text{Total}}^H}\right) \textcolor{red}{p_{\lambda_{\mathbf{H}_{\text{Total}}\mathbf{H}_{\text{Total}}^H}}(\lambda)} d\lambda
\end{equation*}
with $N$ = rank($\mathbf{H}_{\text{Total}}$).
Need Asymptotic Eigenvalue Distribution (AED).
\end{frame}


\subsubsection{Previous Work + Free Probability}


\begin{frame}
\frametitle{Using Free Probability to find $p_{\lambda_{\mathbf{H}_{\text{Total}}\mathbf{H}_{\text{Total}}^H}}(\lambda)$}
Free probability provides a set of tools to find the AED of polynomials in free, self-adjoint operators (matrices).

\begin{itemize}
\item 
	For $\mathbf{A}$ and $\textbf{B}$ consider $\mathbf{C} = \mathbf{A} + \textbf{B}$ and $\mathbf{D} = \mathbf{A}\textbf{B}$
\item 
	Additive free convolution: $R_{\mathbf{C}}(w) = R_{\mathbf{A}}(w) + R_{\mathbf{B}}(w)$
\item 
	Multiplicative free convolution: $S_{\mathbf{D}}(z) = S_{\mathbf{A}}(z)S_{\mathbf{B}}(z)$
\end{itemize}

\begin{equation*}
\mathbf{H}_{\text{Total}}\mathbf{H}_{\text{Total}}^H 
= 
\underbrace{
\mathbf{H}_{2}\boldsymbol{\Phi}\mathbf{H}_{1}\mathbf{H}_{1}^H\boldsymbol{\Phi}^H\mathbf{H}_{2}^H +
\underbrace{\mathbf{H}_{2}\boldsymbol{\Phi}\mathbf{H}_{1}\mathbf{G}^H}_{\color{red}{\text{non-hermetian}}}+
\underbrace{\mathbf{G}\mathbf{H}_{1}^H\boldsymbol{\Phi}^H\mathbf{H}_{2}^H}_{\color{red}{\text{non-hermetian}}}+
\mathbf{G}\mathbf{G}^H
}_
{\color{red}{\text{not free}}}
\end{equation*}
%Need to show that these are not free!
\end{frame}

\begin{frame}
\frametitle{Free Probability}
The symmetrized density function.
\begin{equation*}
\tilde{p}(x) = \frac{p(x) + p(-x)}{2}
\end{equation*}
The symmetrized singular value, eigenvalue relationship.
\begin{equation*}\label{svd_aed_property}
\tilde{G}_{H}(s) = sG_{HH^{\dagger}}(s^2).
\end{equation*}
Additive free convolution still applies for singular values.
We only need the self-adjoint terms of the polynomial!
\end{frame}

\begin{frame}
\frametitle{Free Probability: Canceling Phase Terms}
Say 
\begin{equation*}
C_2 = \mathbf{H}_{2}\boldsymbol{\Phi}\mathbf{H}_{1}[\mathbf{H}_{2}\boldsymbol{\Phi}\mathbf{H}_{1}]^H
\end{equation*}
and a rotated version 
\begin{equation*}
\tilde{C}_2 = \underbrace{\boldsymbol{\Phi}^H\mathbf{H}_{2}^H\mathbf{H}_{2}\boldsymbol{\Phi}}_{C_{1}}\mathbf{H}_{1}\mathbf{H}_{1}^H.
\end{equation*}
Use the property 
\begin{equation*}\label{rotation_property}
S_{C_2}(z) = \frac{z+1}{z+\chi_2} S_{\tilde{C}_2}(\frac{z}{\chi_2}).
\end{equation*}
\pause
Repeat for
\begin{equation*}
C_{1} = \boldsymbol{\Phi}^H\mathbf{H}_{2}^H\mathbf{H}_{2}\boldsymbol{\Phi}
\end{equation*}
\begin{equation*}
\tilde{C}_{1} = \textcolor{red}{\boldsymbol{\Phi}\boldsymbol{\Phi}^H}\mathbf{H}_{2}^H\mathbf{H}_{2}
\end{equation*}
with $\boldsymbol{\Phi}\boldsymbol{\Phi}^H = \mathbf{I}$.
\end{frame}



\begin{frame}
	\frametitle{Capacity of constant channels with random phase
	}
	\centering 
	$\mathbf{H}_{2} \in \mathbb{C}_{R \times T}$, 
	$\mathbf{H}_{1} \in \mathbb{C}_{S \times T}$, with $\mathcal{CN}(0,\frac{1}{\sqrt{RS}})$
	and $ \mathbf{G} \in \mathbb{C}_{R \times T}$ with $\mathcal{CN}(0,\frac{1}{R})$. 
	Statistics from $1000$ random realizations of $\boldsymbol{\Phi}$.
	\bigskip
	
\begin{columns}
\begin{column}{0.5\linewidth}
\centering
Asymptotic case \\ $T = 100$, $ R = 100$ and $S = 100$
\begin{itemize}
\item
Capacity variance:
 0.0038458296727919886
 \item
Capacity average:
 6.764651253343401
 \item
Capacity min:
 6.574540172102809
 \item
Capacity max:
 6.98960781887363
\end{itemize}
%\textcolor{red}{\chi_N}}
\end{column}
\begin{column}{0.5\linewidth}
\centering
Non-asymptotic case\\ $T = R = 4$ and $S = 100$
\begin{itemize}
\item
Capacity variance:
 0.4157628312024758
 \item
Capacity average:
 4.014673474337778
 \item
Capacity min:
 2.146914646653352
 \item
Capacity max:
 6.130275960052021
\end{itemize}
\end{column}
\end{columns}
\end{frame}

\begin{frame}
	\frametitle{Numerical Results: Asymptotic Capacity}
	
	\begin{equation*}
	\mathbf{y} = (\mathbf{H}_2\underbrace{\boldsymbol{\Phi}}_{\text{IRS}}\mathbf{H}_1 + \underbrace{\mathbf{G}}_{\text{LOS: Rank $\leq 1$}})\mathbf{x}
	\end{equation*} 
\begin{figure}
\includegraphics[width= .75\linewidth, height = 5.5cm]{los}
\end{figure}
\end{frame}

\begin{frame}
	\frametitle{Numerical Results: Asymptotic Capacity}
		\begin{equation*}
	\mathbf{y} = (\underbrace{\sigma}_{\text{attenuation}}\mathbf{H}_2\underbrace{\boldsymbol{\Phi}}_{\text{IRS}}\mathbf{H}_1 + \underbrace{\mathbf{G}}_{\text{LOS}})\mathbf{x}
	\end{equation*} 
\begin{figure}
\includegraphics[width= .75\linewidth, height = 5.5cm]{attenuation}
\end{figure}
\end{frame}



\begin{frame}
\frametitle{Current Status}
\begin{enumerate}
\item
	Find capacity of asymptotic IRS systems  \faCheck
	\begin{itemize}
	\item 
		Optimize phase matrix	\faQuestion
	\item
		Generalize results
	\end{itemize}

\item 
	Confirm findings with numerical results throughout
	
\item
	* Time permitting - Consider secrecy capacity of IRS system
\end{enumerate}
\end{frame}


\section{Conclusion}

\begin{frame}
  \centering \Large
  \emph{Thank You.}
  \\
	\bigskip
    \centering \Large
  \emph{Questions or Comments?}

\end{frame}

\begin{frame}[allowframebreaks]
        \frametitle{References}
        \bibliography{bibliography.bib}
\end{frame}



\end{document}
