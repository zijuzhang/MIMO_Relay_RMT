The spectrum of electromagnetic frequencies used by wireless communication has been expanded to progressively higher frequencies with new standards. Industry research for the 6G standard has already investigated systems using frequency bands in the 120, 200 and 340 GHz range \cite{Koziol}. This shift is motivated by large bands of currently unoccupied frequency and the possibility for decreased antenna size (allowing for increased antenna density) \cite{akyildiz2018combating}. One primary hurdle towards efficient communication at these frequencies is the high attenuation of radio energy due both to scattering effects and molecular absorption. \cite{TODO} show through extensive data collection that relevant attenuation will be (x). 
As a relevant example, radar systems have long used this band of frequencies due to the resulting high resolution. For example, one such system operating in the 2 - 4 GHz band requires 25 kW of power for a detection range of 400 km \cite{TODO}. 
Countering this type of loss is a key step towards meeting the increased data rates and reliability required for future networks. 
\par
In previous generations of wireless communications, attenuation due to path loss has been alleviated by reducing the effective transmission distance 
and supplementing the transmission power using relays \cite{dahlman20134g}. Two, key costs to deploying relays are the radio frequency chains used for receiving and transmitting, and the power used for transmission (CITE?). Deploying enough relays to counter
the attenuation levels relevant to new standards would pose impractical costs. One new solution that has attracted significant interest is the use of surfaces made of many, sub-wavelength sized elements with favorable electromagnetic reflection and whose reflective coefficient can be controlled to select the phase of the reflected wave. In this work, such a surface will generally be referred to as a reconfigurable intelligent surface (RIS).
While this work focuses on the theory of communications channels with such surfaces, it is worth noting that these surfaces have been successfully implemented in works such as \cite{tan2018enabling} in which an RIS with 224 elements was built and tested. 
\par
The motivating use for the RIS is to alleviate the power attenuation problem without incurring the costs of an rf-chain and transmission power 
associated with a relay. In addition, the absence of an rf-chain on the RIS offers some benefits over relays. First, reflection from an RIS does not introduce the noise associated with an rf-chain. Second, without the band-pass filters of an rf-chain, RISs have a frequency response over the entire spectrum (explain). Third, the RIS is inherently full-duplex, a feature that is difficult to implement for co-located rf-chains on an active relay. 
The thorough review in \cite{basar2019wireless} provides rigorous motivation of the use of RISs and a few notable points are included here.
Consider a single antenna, single receiver channel using an RIS with $N$ reflective elements. In this case, the phase control at the RIS can be used to add the multiple paths coherently at the receiver and intuitively increase the received signal power. While such results are motivating, there are two points that should be noted. First, in this example from \cite{basar2019wireless}, the received power would still scale with $N^2$ even if the reflective surfaces had no control over the reflection phase. Second, it is important to note that scaling of received SNR proportional to $N^2$ will only hold to the point limited by energy conservation \cite{bjornson2019demystifying}.
This leaves us with a question. What is the true potential benefits for the RIS in future systems?
Before investigating this question we review key aspects of MIMO systems need to fully develop a system model with which to explore the RIS. 