Proofs
\begin{enumerate}
\item 
	Eigen/Singular Relationship
\item
	Prove freeness of the relevant points. 
\item 
	rotations have same non-zero eigenvalues
\item 
Proof of split using singular values.\\
Consider the channel given by 
\begin{equation}
\mH_{Total} = \mH_{2} + \mH_{1}.
\end{equation}
with covariance matrix 
\begin{equation}
\mC =
\mH_{2}\mathbf{H}_{2}^H + 
\mH_{1}\mathbf{H}_{1}^H + 
\mH_{2}\mathbf{H}_{1}^H +
\mH_{1}\mathbf{H}_{2}^H .
\end{equation}
Because the components of this sum are not free (CITE) (and some are non-hermetian), additive free convolution cannot be used.
We note that using the singular value decomposition $\mH_{Total} = \mU \boldsymbol{\Sigma} \mV^H$ we can
see that the covariance matrix can be written 
\begin{equation}
\mC = \mU \boldsymbol{\Sigma}\boldsymbol{\Sigma}^H \mU^H
= \mU \boldsymbol{\Lambda} \mU^H.
\end{equation}
With $\mC \succeq 0 \implies \lambda(\mC\mC)_i \geq 0$ we see that the definition of the singular values 
$\sigma(\mC)_i = \sqrt{\lambda(\mC\mC)_i}$ leads to an expression relating the density function of the singular values of a channel matrix
, $p_{\sigma}(x)$, and the eigenvalues of the channel covariance matrix, $ p_{\lambda}(x)$,  given by $p_{\sigma}(x^2) = p_{\lambda}(x)$.
Defining a new, symmetric density function of the singular value distribution
 	 \begin{equation}\label{symmetric}
 	 \tilde{p}_{\sigma}(x) = \frac{p_{\sigma}(x) + p_{\sigma}(-x)}{2},
 	 \end{equation}
 	 we now substitute $\tilde{p}_{\sigma}(x)$ into the definition of the Stieljes transform \eqref{stieltjes}
 	 to give the relationship
 	 	 	 \begin{align*}
	 	 \tilde{G}_\sigma (s) & =  \frac{1}{2} \int_{-\infty}^{\infty} \frac{p_{\sigma}(x) + p_{\sigma}(-x)}{x-s}dx
	 	 \\&  =  
	 	 \frac{1}{2} \int_{-\infty}^{\infty} \frac{dp_{\sigma}(x)}{x - s} + 
	 	 \frac{1}{2} \int_{-\infty}^{\infty} \frac{dp_{\sigma}(x)}{-x - s}
	 	 \\&  =  
	 	 	 	 \frac{1}{2} \int_{-\infty}^{\infty} \frac{-2s dp_{\sigma}(x)}{(-x - s)(x - s)}
	  	 \\&  =  
	 	 	 	\int_{-\infty}^{\infty} \frac{-s dp_{\sigma}(x)}{-x^2 + s^2} 	
 	 	  	 \\&  =  
 	 	 	-s  \int_{-\infty}^{\infty}  \frac{dp_{\sigma}(x)}{ x^2 - s^2} 	 
 	 	 \\&  =  
			-s  \int_{-\infty}^{\infty}  \frac{dp_{\lambda}(x)}{ x - s^2} 
	 	 \\&  = 
	 	 	-s G_{\lambda}(s^2)
	 	 \end{align*}
With the R-transform definition unchanged for the symmetric distribution, and the additional useful property (CITE)
\begin{equation}
\tilde{R}_{\mH_{Total}}(w) = \tilde{R}_{\mH_{2}}(w) + \tilde{R}_{\mH_{1}}
\end{equation}
We find that we only need the Stieltjes transform for the covariance matrix of each component of the sum which can in general be found using the tools
of multiplicative free convolution.

\item 
Proof of canceling phases.\\
Consider the channel given by 
\begin{equation}
\mathbf{H}_{Total} = \mathbf{H}_{2}\boldsymbol{\Phi}\mathbf{H}_{1}.
\end{equation}
This product of matrices suggests the use of multiplicative free convolution to solve for the AED using the S-Transform. 
Following the same procedure show in \cite{muller2002asymptotic}, we begin with the covariance matrix
\begin{equation}
\mC_2 = \mathbf{H}_{2}\boldsymbol{\Phi}\mathbf{H}_{1}\mathbf{H}_{1}^H\boldsymbol{\Phi}^H\mathbf{H}_{2}^H
\end{equation}
and note that this has the same non-zero eigenvalues as the matrix (Proof)
\begin{equation}
\tilde{\mC}_2 = \boldsymbol{\Phi}^H\mathbf{H}_{2}^H\mathbf{H}_{2}\boldsymbol{\Phi}\mathbf{H}_{1}\mathbf{H}_{1}^H.
\end{equation}
The relationship between the S-Transform of $\mC_2$ and $\tilde{\mC}_2$ is given by 
\begin{equation}\label{rotation_property}
S_{C_N}(z) = \frac{z+1}{z+\chi_2} S_{\tilde{C}_N}(\frac{z}{\chi_2}).
\end{equation}
With \begin{equation}
\mC_1 = \boldsymbol{\Phi}^H\mathbf{H}_{2}^H\mathbf{H}_{2}\boldsymbol{\Phi}
\end{equation}
we see that $S_{\tilde{\mC}_2}(z) = S_{\mC_1}(z) S_{\mathbf{H}_{1}\mathbf{H}_{1}^H}(z)$.  Repeating the above procedure on $\mC_1$ we find
\begin{equation}
\tilde{\mC}_{1} = \boldsymbol{\Phi}\boldsymbol{\Phi}^H\mathbf{H}_{2}^H\mathbf{H}_{2}
\end{equation}
with $\boldsymbol{\Phi}\boldsymbol{\Phi}^H = \mathbf{I}$ we are left with 
$S_{\mC_1}(z) =  S_{\mathbf{H}_{2}^H\mathbf{H}_{2}}(z)$. As a result we see that 
the phase matrix cancels. The same canceling applies to any unitary matrix $\mathbf{A}$ such that   $\mathbf{A}\mathbf{A}^\dagger =\mathbf{A}^\dagger \mathbf{A} = \mI$.
\end{enumerate}