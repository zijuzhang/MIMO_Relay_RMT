\documentclass[12pt,a4paper]{report}
\usepackage[utf8]{inputenc}
\usepackage{amsmath}
\usepackage{amsfonts} 
\usepackage{amssymb}
\usepackage{graphicx}
\usepackage[colorlinks=true,linkcolor=blue]{hyperref}
\usepackage{float}
\input defs.tex
\bibliographystyle{alpha}
\graphicspath{ {./figures/} }
%\hyphenpenalty=10000

\title{Free Probability Analysis for Reflective Surface aided Massive MIMO systems}
\author{Peter Hartig}

\begin{document}
\maketitle
\begin{abstract}
Surfaces which can control the phase of reflected radio waves, known as intelligent reflective surfaces (IRS), have been the subject of interest in recent years as a means to better control the wireless propagation environment. This work investigates the impact of an IRS on the MIMO communication systems. Using the tools of free probability, a system is analyzed in the asymptotic domain to derive an expression for the system capacity. A key finding is that the phase control offered by the IRS has no impact in the asymptotic domain. Motivated by the contradiction between this finding and the substantial previous literature investigating the optimization of the IRS, we consider the convergence of a realistic system towards asymptotic behavior. We then propose an optimization method for the non-convex problem of minimizing the Mean Square Error with respect to the IRS phases.
\end{abstract}
%
%\newpage
\tableofcontents
\chapter{Introduction}
\section{Background}\label{Background}
In new standards for wireless communication, the spectrum of electromagnetic frequencies used has been expanded to progressively higher frequencies. Industry research for the 6G standard has already investigated systems using frequency bands in the 120, 200 and 340 GHz range \cite{Koziol}. This shift is motivated by large bands of currently unoccupied frequency and the possibility for decreased antenna size (to enable increased antenna density) \cite{akyildiz2018combating}. One primary hurdle towards efficient communication at these frequencies is the high attenuation of radio energy due both to scattering effects and molecular absorption. \cite{TODO} show through extensive empirical results that relevant attenuation can reach (x). 
As a relevant example, radar systems have long used this band of frequencies due to the resulting high resolution. For example, one such system operating in the 2 - 4 GHz band requires 25 kW of power for a detection range of 400 km \cite{TODO} (NEED NEW EXAMPLE). 
Countering this power loss is a key step towards meeting the increased data rates and reliability required for future wireless communication networks. 
\par
In previous generations of wireless communications, attenuation due to path loss has been alleviated by incorporating relays into the system to reduce the point-to-point transmission distance and supplement the transmission power \cite{dahlman20134g}. Two key costs to deploying relays are the radio frequency chains used for receiving and transmitting, and the power used for transmission (CITE?). As of today, most relays are only capable of operating in a half-duplex mode and thus a third cost to the system (by reducing capacity) is often incurred by using relays. Deploying enough relays to counter
the attenuation levels relevant to new standards would pose impractical costs. One new solution that has attracted significant interest is the use of surfaces made of many, sub-wavelength sized elements with favorable reflection for the relevant electromagnetic spectrum and whose reflective coefficient can be controlled to select the phase of the reflected wave. In this work, such a surface will generally be referred to as an intelligent reflective surface (IRS).
While this work focuses on the theory of IRS communication channels, it is worth noting that these surfaces have been successfully implemented in works such as \cite{tan2018enabling} in which an IRS with 224 elements was built and tested. 
\par
The motivating use for the IRS is to alleviate the power attenuation problem without incurring the costs of an rf-chain and transmission power 
associated with an active relay. In addition, the absence of an rf-chain on the IRS offers some benefits over relays. First, reflection from an IRS does not introduce the noise associated with an rf-chain. Second, without the band-pass filters of an rf-chain, IRSs have a frequency response over the entire spectrum (explain). Third, the IRS is inherently full-duplex, a feature that is difficult to implement for co-located rf-chains on an active relay. 
The thorough review in \cite{basar2019wireless} provides rigorous motivation of the use of IRSs and a few notable points are included here.
Consider a single antenna, single receiver channel using an IRS with $N$ reflective elements. In this case, the phase control at the IRS can be used to add the multiple paths coherently at the receiver and intuitively increase the received signal power. While such results are motivating, there are two points that should be noted. First, in this example from \cite{basar2019wireless}, the received power would still scale with $N^2$ even if the reflective surfaces had no control over the reflection phase. Second, it is important to note that scaling of received SNR proportional to $N^2$ will only hold to the point limited by energy conservation \cite{bjornson2019demystifying}.
This leaves us with a question. What is the true potential benefits for the IRS in future systems?
Before investigating this question we review key aspects of MIMO systems need to fully develop a system model with which to explore the IRS. 
\par
\subsection{MIMO with correlation}\label{mimo_corr}
\par
Early investigation which lead to the current interest in MIMO systems showed the key property that with constant transmit power, the capacity (and mutual information) in an $N \times N$ MIMO system with independent, i.i.d Gaussian channels scales linearly in $N$ \cite{foschini1998limits}. While the assumption of \emph{identically distributed} channels is reasonable when adding more antennas, the assumption of \emph{independent} channels between each pair of transmit and receive antennas is not. To understand the potential for leveraging this capacity scaling in deployed massive MIMO systems, the investigation of correlated MIMO channels has gained substantial interest. In the following, a number of these studies are reviewed. Some of these chosen works use channel correlation models similar to the one we will develop for the RIS system model. In particular, this method models the correlation between antennas as decreasing  exponentially with the distance between antennas. Definitely need to show this is a valid model (Cite).
\par
Might need to give more motivation for the use of asymptotic analysis.
\par
One method that has been shown to reduce the complexity of this calculation is to allow the system size $N \rightarrow \infty$ while normalizing the received power. 
This is motivated by the observation that when the random channel is properly normalized, the eigenvalues of the channel covariance matrix often converge to a deterministic distribution which will be referred to as the Asymptotic Eigenvalue Distribution (AED). 
Because the received power would generally not be normalized as the system size in increased, results found while using this normalization be extended by simply scaling the asymptotic result with $N$.
(Still need to clarify the types of MIMO gains that are observed under this normalization and how to deal with non-zero mean components).
\par
Evaluating the capacity for the case in which the transmitter has channel state information (CSIT ) can be difficult as the solution to this optimization is not generally in a closed-form. In \cite{loyka2001channel}, this problem is avoided using an upper bound on the capacity of a general MIMO channel derived from Jensen's inequality. Using this upper bound and the exponential correlation model, an interpretation is given in which increasing correlation between antennas corresponds to a decrease in the average SNR at the receiver. 
Similarly, \cite{chuah2002capacity} investigates the exponentially correlated MIMO system using close approximations to compares optimal versus equal power distribution at the transmitter for cases with high and low SNR. One interesting finding from this work is that for the case with low
SNR (what does this mean?) the water-filling capacity is actually increased with correlation while in all other cases, correlation has a negative impact on the capacity. The intuition because this finding is that the equal power case is only impacted by the SNR and not the distribution of the channel eigenvalues (unlike the water-filling case).
Finally, \cite{skupch2005free} used multiplicative free convolution from free probability theory to analyze the exponential correlation model at either the transmitting or receiving antenna. Similarly, they also observe a decrease in the capacity as correlation increase but only at high values of correlation in the system. Part of this result will be used in our analysis. 
Finally,
\cite{taricco2008asymptotic} analyzed the correlated Ricean fading channel using the Replica Method. 
In general it is observed that the slope of the linear scaling of (mutual information) is reduced by the addition of correlation. 
\par
Before moving on to apply any of these asymptomatic results with our own system model, it is fitting to consider two questions. 
First, whether or not the assumption of  $N \rightarrow \infty$ is reasonable? Intuitively, we would expect that under correlation this convergence may occur only at very high $N$. This result was confirmed in  \cite{martin2004asymptotic} using the exponential correlation model at both the transmitter and receiver. 
Second, what values for the parameters of the correlation model are reasonable for a real-world system?
In test data collected with channel parameters (carrier frequency, bandwidth, power...) relevant to 3G and 4G, wireless correlation between channels was shown to be as high as $0.5$ for a point-to-point, $4 \times 4$ (linear array?) system \cite{martin2000multiple} (TODO justify if this is a comparison for current systems).
\par
\subsection{RIS Optimization}\label{irs_opt}
The previous section looked at a select few entries from the substantial literature investigating the capacity of MIMO systems.
This analysis can often be difficult because the definition of capacity implies an optimization over any controlled variables. 
While in select cases this optimization may yield a closed form solution that may be used in the analysis, this is not generally the case. 
In fact, the current literature on RIS system has skipped this analysis entirely because the RIS optimization is a non-convex problem. Rather 
numerous algorithms have been proposed and analyzed numerically. We now briefly review a few selected results from the current RIS literature. 
\par
Go through and review
\par
One key point to note is that in none of the above works and indeed so far as the author's knowledge have these works considered relevant case of a MIMO channel with channel correlation. 
\par
In summary, this section has highlighted two primary points. First, that channel correlation can degrade the capacity scaling of massive MIMO systems. Second, that an RIS may provide additional degrees of freedom with which to increase capacity. We now investigate these questions by formulating and analyzing an RIS-aided, MIMO system with channel correlation. 
\section{Notation}
The following notation is used throughout the remainder. 
$\Expect[x]$ is the expected value of a random variable $x$.
Bold font, lower-case letters ($\vx$) and upper-case letters ($\mX$) denote vector and matrices respectively and all vectors are assumed to be column vectors unless noted otherwise. 
The transpose and hermetian of $\mX$ are denoted by $\mX^T$ and $\mX^H$ respectively.
$\mI_N$ denotes the identity matrix of dimension $N$.
For square $\mX$ with dimension $N\times N$ the normalized trace is denoted by $\Tr(\mX)  = \underset{N \rightarrow \infty}{\text{lim}}
\Expect[\frac{1}{N}\sum_{i=1}^N x_{i,i}]$
\section{Outline}
The following sections are now previewed. Section \ref{system_model} covers the system model used in the remainder as well as a review of the necessary theory to analyze the resulting channel. In section \ref{Results}, the theoretical results and corresponding numerical results are discussed. 

\chapter{Analyzing the IRS System}\label{system_model}
\section{Analysis Preliminaries}
\subsection{Power Normalization}
Consider the SISO channel with an RIS given by $y = \vh_{rs}^H \phase \vh_{st}$. In this case, $ \phase$ should be chosen optimally such that 
the phases all add constructively at the receiver (\ie $h_{rs,i} \phi_k h_{st,j}$ add constructively at the receiver). 
In the case of an i.i.d Rayleigh channel with elements $\CNO$ and $\sigma^2_{\text{noise}} =1$, the SNR at the receiver is given by $\gamma = N$.
Intuitively if the channel gains are normalized such that the received power is normalized for any system size $N$ the IRS offers no gain. 
In this case we can achieve this normalization by enforcing that elements of the a channel matrix $\mH \in \CN^{N \times K}$ have variance $\frac{1}{N}$.
This observation highlights an point that must be considered when performing asymptotic analysis; the received power must be normalized.

\subsection{Mutual Information}\label{sectiond:mut_info}
Consider the general MIMO communication system given by 
	\begin{equation}\label{com_system}
	\vy = \mH_{Total}\vx + \vn
	\end{equation}
	in which $\mH_{Total}$ is a random channel (we can always represent linearly?).
A key metric of interest for this system is the mutual information between the transmitted and received signal given by 
\begin{equation}\label{mut_ent}
I(\vy;\vx) = h(\vy) - h(\vy|\vx).
\end{equation}
If both $\vx$ and $\vn$ are circularly symmetric Gaussian random vectors, $\vy$ is also circularly symmetric Gaussian with entropy given by \cite{telatar1999capacity} 
\begin{equation}\label{entropy}
h(\vy) = \Expect[\Log(|\pi e \mathbf{Q}_{y}|)]
\end{equation}
 for $\mathbf{Q}_{y} = \Expect{[\vy \vy^H]}$. Note also that for a given $\mathbf{Q}_{y}$ the entropy is maximized if and only if $\vy$ is circularly symmetric Gaussian. (explain what implication this has)
For a given transmit covariance matrix, $\mathbf{Q_x}$ and $\Expect\left[\vn\vn^H \right] = \mI$, substituting equation \eqref{entropy} into equation
\eqref{mut_ent} and simplifying gives
\begin{equation}\label{mut_inf}
\Expect \left[\Log(|\mathbf{I} + \mHt \mQ_x \mHt^H|)\right].
\end{equation}
Maximizing the mutual informationwith respect to both the
transmit covariance matrix and phases at the RIS gives an expression for the system capacity

\begin{equation}\label{capacity}
\mathbb{C} = \underset{\boldsymbol{\Phi},\mathbf{Q_x}}{\mathop{max}} \Expect \left[\Log(|\mathbf{I} + \mHt \mQ_x \mHt^H|)\right]
\end{equation}
or equivalently,
\begin{equation}\label{capacity_tricky}
\mathbb{C} = \underset{\boldsymbol{\Phi},\mathbf{Q_x}}{\mathop{max}} \Expect \left[\sum_{i=1}^{N}\Log(1 + \lambda_i)\right].
\end{equation}
where $\lambda_i$ are the eigenvalues of $\mHt \mQ_x \mHt^H$.

Recalling that in general we are interested in evaluating the capacity in order to determine a feasible transmission rate for a given communication channel, equation \eqref{capacity_tricky} poses a couple of hurdles to evaluation. First, we need to be able to evaluate an expectation over the joint pdf $p(\lambda_1 \cdots	 \lambda_N)$. Second, there is an optimization problem whose solution may in general depend on the distribution of the channel. Including this solution in the capacity expression without a closed form may be difficult.
To resolve these hurdles, two simplifications will be used.
First, we will enforce that the transmitter chooses a deterministic $\mQ_x$. 
By enforcing a deterministic choice of $\mQ_x$, we no longer have to consider this component of the expectation.
Second we will normalize the channel, and therefore the received power, to allow the received signal covariance matrix eigenvalue distribution to converge. 

If $\mHt$ is not known at the transmitter, an intuitive choice for the transmitter signal covariance matrix is  $\mQ_x = \frac{P_{\text{total}}}{N_t}\mathbf{I}$ which leads to the form of equation \eqref{capacity}
\begin{equation}\label{no_csi_capacity}
\mathbb{C} = \underset{\phase}{\mathop{max}} \; N_R \Expect \left[\Log(1 + \frac{P_{\text{total}}}{\sigma_nN_t}\lambda_i)\right].
\end{equation}
where $\lambda_i$ are the eigenvalues of $\mHt \mHt^H$.
In this form, we can clearly see the linear scaling of capacity with $N_R$ that has motivated MIMO research.
If we assume that $\Expect\left[ \vy \vy^H \right]$ has a converging eigenvalue distribution as the system dimensions increase to infinity, we can expand the expectation over the AED to get the expression
\begin{equation}\label{no_csi_capacity_aed}
\mathbb{C} = \underset{\phase}{\mathop{max}} \; N_R  \int_{0}^{\infty}\Log(1 + \frac{1}{\sigma_n}x)p_{\lambda\lambda^H}(x) dx
\end{equation}
in which $p_{\lambda\lambda^H}(x)$ is the channel AED and the bounds of the integral reflect the range for the eigenvalues of a positive semidefinite matrix.
Note that because the choice for the distributions of $\vx$ and $\vn$ maximize $h(\vy)$, equation \eqref{capacity} provides an upper bound on the capacity of the channel for any choices of the distribution of $\vx $ with Gaussian noise $\vn$. (Perhaps mention that normally we cannot generate continuous Gaussian x so we use channel coding).


Note that because $\Log$ is a concave function, we can upper bound the mutual information with
\begin{equation}\label{mut_inf}
\Expect[\Log(1 + \frac{1}{\sigma_n}\lambda)] \leq \Log(1 + \frac{1}{\sigma_n}\Expect[\lambda])
\end{equation}
\subsection{Free Probability}\label{Free_Prob_Intro}
\begin{itemize}

\item 
	A brief introduction to free probability.
	\par 
	Free probability is the counterpart to classical probability used to analyze cases in which random variables (in this case random matrices) do 
	not commute. In the same way that classical probability provides methods to find the pdf for the sums and products of commuting random
	 variables, free probability provides tools to find the pdfs for products and sums of non-commuting random variables.
	In the following work, a number of tools from free probability will be used. These tools are now introduced with appropriate context and intuition 
	for this investigation but for a thorough coverage of these topics, see \cite{tulino2004random} and \cite{mingo2017free}.
	 \begin{itemize}
	 \item 
	 	The cornerstone of free probability is the concept freeness of two random variables in the context of a projection. In our case, freeness between two 
	 	matrices is given by 

	 	\item 
	 	Stieltjes Transform: 
	 	Generally, the projection of matrix is not given in closed form. The projection can be uniquely represented using
		\begin{equation}\label{stieltjes}
		G(s) = \int_{-\infty}^{\infty} \frac{dp(x)}{(x - s)} \; \text{for} \; \img(s) > 0
		\end{equation}	
		This representation will allow us to find the projection of the sum and project of matrices using addition and multiplication.
		\item 
		R-Transform:
		Mention self-adjoint
		Given the Stieltjes transform of a RV, the corresponding R-Transform given by 
		X. 
		The R-Transform of the sum of two, free RVs can be found using
		Y
		This process use known as free additive convolution.
		\item
		S-Transform:
		Mention self-adjoint
		Similarly, given the Stieltjes transform of a RV, the S-Transform given by 
		X, X;
		The S-Transform of the product of two, free RVs can be found using
		Y
		This process is known as free multiplicative convolution.
	 	\item 
	 	If there is a case in which finding the singular value distribution is easier than the eigenvalue distribution we can use equation  
	 	 \eqref{sigular_eigen} to move between the two. Considering a symmetric function of the singular value distribution
	 	 \begin{equation}\label{symmetric}
	 	 \tilde{p}_{\sigma}(x) = \frac{p_{\sigma}(x) + p_{\sigma}(-x)}{2}
	 	 \end{equation}
	 	 the definition of the Steiltjes transform gives the following steps, leading to a relationship between the Stieltjes transforms. 
	 	 \begin{align*}
	 	 \tilde{G}_\sigma (s) & =  \frac{1}{2} \int_{-\infty}^{\infty} \frac{p_{\sigma}(x) + p_{\sigma}(-x)}{x-s}
	 	 \\&  =  
	 	 \frac{1}{2} \int_{-\infty}^{\infty} \frac{dp_{\sigma}(x)}{x - s} + 
	 	 \frac{1}{2} \int_{-\infty}^{\infty} \frac{dp_{\sigma}(x)}{-x - s}
	 	 \\&  =  
	 	 	 	 \frac{1}{2} \int_{-\infty}^{\infty} \frac{-2s dp_{\sigma}(x)}{(-x - s)(x - s)}
	  	 \\&  =  
	 	 	 	\int_{-\infty}^{\infty} \frac{-s dp_{\sigma}(x)}{-x^2 + s^2} 	
 	 	  	 \\&  =  
 	 	 	-s  \int_{-\infty}^{\infty}  \frac{dp_{\sigma}(x)}{ x^2 - s^2} 	 
 	 	 \\&  =  
			-s  \int_{-\infty}^{\infty}  \frac{dp_{\lambda}(x)}{ x - s^2} 
	 	 \\&  = 
	 	 	-s G_{\lambda}(s^2)
	 	 \end{align*}
	 	\item 
	 		Another useful property that will be used is the relationship between the eigenvalues of a matrix and its rotation.
	 		Specifically, consider the covariance matrix given by $\mH_2\mH_1 \mH1^H \mH2^H$.
	 		By the rotational property of the trace, 
	 		\begin{align*}
	 		trace({\mH_2\mH_1 \mH1^H \mH2^H }) = \trace({\mH_1^H \mH_2^H \mH_2 \mH_1 }).
	 		\end{align*}
	 		... Not sure if I need to prove this here.  
	 	\item 
	  	build up to show that phases cancel via s-transform
	 	Stieltjes of AED (IE for hermetian matrices)
	 	Stieltjes of SVD (when they are components covariance matrix)
	 	Getting Steiltes of SVD using R-Transform
	 	Finding R-Transform of Individual components
	 	Using Stieltjes of individuals to get R-transforms
	 	Using aed stieltjes to get svd individual
	 	Using S-transform to get stieltjes of individual aed.
	\end{itemize}	 	
	
	Now describe with S-Transform and show that it can have a singular distribution in the limit at we go to rank 1. 
		
\item 
	Show that at least for uncorrelated case, the matrices in the sum are free. 

\item 
	Give proofs in an appendix?
	
\end{itemize}
\section{Applying Free Probability to the IRS Chanel}
\subsection{The Asyptotic IRS Channel}
Using the approach from Section \ref{Free_Prob_Intro} to find the asymptotic eigenvalue distribution, we now analyze a channel model that is representative of a MIMO communication system with an RIS. 
\par
In general, the received signal from any line of sight path between a transmitter and receiver would contain  sufficient energy to disregard the contribution of any reflected signal (although this component may be very low rank).  Therefore, we focus on a channel model in which all signal power to the receiver is through the path of the RIS  given by

%	\begin{equation}
%	\mathbf{H}_{Total} = \mathbf{R}_{R}^{\frac{1}{2}}(\underbrace{\mathbf{H}_{2}\boldsymbol{\Phi}\mathbf{R}_{S}^{\frac{1}{2}}\mathbf{H}_{1}}_{\text{IRS}} + \underbrace{\mathbf{G}}_{\text{LOS}})\mathbf{R}_{T}^{\frac{1}{2}}.
%	\end{equation}
	\begin{equation}\label{correlated_channel}
	\mathbf{H}_{Total} = \mathbf{R}_{R}^{\frac{1}{2}}\mathbf{H}_{2}\underbrace{\phase}_{\text{IRS}}\mathbf{R}_{S}^{\frac{1}{2}}\mathbf{H}_{1}\mathbf{R}_{T}^{\frac{1}{2}}.
	\end{equation}
The random channels $\mathbf{H}_{1}$ and $\mathbf{H}_{2}$ represent the random channel gains from transmitter to RIS and from RIS to receiver, respectively. Similarly, $ \mathbf{R}_{T},  \mathbf{R}_{S} $ and  $\mathbf{R}_{S}$ represent the correlation introduced at the transmitter, RIS and receiver, respectively. 
When compared with the example seen in Section \ref{Free_Prob_Intro}, the structure of $\mathbf{H}_{Total}$ suggests that we can to use multiplicative free convolution in order to evalute the AED. With this in mind, we would like to choose a correlation model  that not only represents the system but one with a known S-Transform expression. 
One such model is exponential correlation. Defined by a single parameter $\rho$, the elements of the corresponding correlation matrix are given by $\rho^{i-j}$.
\par
To begin using the tools of free probability, we must first separate out the free sub-components of the covariance matrix given by 
	\begin{equation}
	\mC = \mR_{R}^{\frac{1}{2}}\mH_{2}\phase \mR_{S}^{\frac{1}{2}}\mH_{1}\mR_{T}^{\frac{1}{2}}
	\left(\mR_{R}^{\frac{1}{2}}\mH_{2}\phase \mR_{S}^{\frac{1}{2}}\mH_{1}\mR_{T}^{\frac{1}{2}}\right)^H.
	\end{equation}
Given the S-transform of these sub-components, the total S-transform is simply the resulting product.
Noting from Equation \eqref{no_csi_capacity_aed} that the capacity is only affected by the non-zero eigenvalues of the covariance matrix, we equivalently
consider the AED of the covariance matrix given by 
	\begin{align}
	\mC &  =  \mH_{1}^H \mR_{S}^{\frac{H}{2}}\phase^H \mH_{2}^H \mR_{R}^{\frac{H}{2}}
	 \mR_{R}^{\frac{1}{2}}\mH_{2}\phase \mR_{S}^{\frac{1}{2}}\mH_{1}\mR_{T}^{\frac{1}{2}}\mR_{T}^{\frac{H}{2}}
	 \\
	 &  =  
	 \mH_{1}^H \mR_{S}^{\frac{H}{2}}\phase^H \mH_{2}^H \mR_{R}\mH_{2}\phase \mR_{S}^{\frac{1}{2}}\mH_{1}\mR_{T}.
	\end{align}
In order to continue decomposing this covariance matrix into sub-components, we exploit a property of the S-Transform given by (proven in Appendix \ref{rotation_property})
\begin{equation}
S_{C_N}(z) = \frac{z+1}{z+\chi_2} S_{\tilde{C}_N}(\frac{z}{\chi_2}).
\end{equation}
If all system dimensions are $N$, these S-Trasforms (and AEDs) identical so for this section we will assume that all system dimensions are $N$.
Repeated iteration of the rotational property above gives the equivalent matrix
	\begin{equation}
	\mC =
\underbrace{\phase^H \phase}_{\mI}
	\mR_{S}
	 \mH_{2}^H \mH_{2}
	 \mH_{1}\mH_{1}^H 
	  \mR_{R}
	 \mR_{T}.
	\end{equation}
Note that using the Eigen Value Decomposition of $\mR_{R}$ and $\mR_{T}$, we find the product $\mR_{R} \mR_{T}$ is also a Toeplitz matix (which can be represented by a circulant matrix in the asymptotic system size).
In \cite[Section 4.3.2]{muller2013applications} it is shown that the hermetian matrix $\mH\mH_^H$ with $h_{i,j} \in \NC\{0, \frac{1}{N}\}$
is free 
Having shown the relative freeness of the matrix sub-components, we must now find the S-transform for each in order to find the product.
\begin{enumerate}
\item 
	For both $\mH_{2}^H \mH_{2}$ and $\mH_{1}\mH_{1}^H$ with all system dimensions equal, the S-Transform as shown in \cite{muller2002random} is given by 
	\begin{equation}
	S(z) = \frac{1}{1+z}
	\end{equation}
\item 
	As shown in \cite{gray2006toeplitz}, with $N \rightarrow \infty$ Toeplitz matrices have the spectrum (and thus AED) of an equivalent circulant matrix (whose spectrum and thus AED is given by the discrete Fourier transform of the first row). Using the definition of the Stieltjes transform and the definition of the S-Transform in terms of the Stieljes transform, the S-Transform of the exponential correlation model is shown to be \cite{skupch2005free}
	\begin{equation}
	S(z) = \frac{\alpha z^2 + z \sqrt{alpha^2 z^2 -(z^2-1)}}{z-1}
	\end{equation}
\end{enumerate}
\par
Without further analysis, this expression shows the interesting result that under the assumptions of Free Probability and equal power transmission, the phases at the RIS (or indeed any unitary matrix in the channel) cancel out of the expression for the AED. For the remaining terms, we simply need an expression for the S-Transform in order to evaluate the asymptotic system capacity.s
\par 
Using the previous analysis we can observe an extension of this result that may be relevant to variations of this system model. 
While we do not consider a line of sight component to this channel, using the tools from a similar analysis in \cite{muller2012channel}, it can be shown that 
a similar cancellation of phases will occur for the RIS channel with an LOS (or non-zero mean) component of the signal.
Note that if we are simply adding a non-zero mean component. We can find the corresponding AED by looking considering the channel without the non-zero mean and then adding the single eigenvalue from the non-zero mean component; in which case the phases will again cancel out. 




\subsection{Non-Asymptotic Results}
Up to this point we have shown an analysis which seems to contradict the observations from previous investigations review in Section \ref{Background} showing the benefits of exploiting RIS phases. The key to the coexistence of these results is determining when the asymptotic assumptions required by Free Probability hold. 
As mentioned in in Section \ref{Background}, \cite {} have shown that for matrices of random elements, the  convergence of the matrix behavior towards that predicted by Free Probability is slowed by adding correlations between elements. We would now like to at least numerically investigate this question to investigate the performance of RIS MIMO systems with parameters reflecting state of the art.


 As seen in \ref{Background} current large systems have been shown for 
sizes S= 224. We now consider whether this size is reasonable under correlations realized in systems (NEED TO FIND and justify). 

\chapter{Results}\label{Results}
\section{Asymptotic Results}
\subsection{Asymptotic Results Discussion}

We first consider the case in which all matrices of the channel given by (ref) are free. As shown in (Appendix), in this case the 
phase shifts at the RIS, $\phase$, cancels out of the AED expression. 
By considering the case in which a single transmitter and receiver communicate over the channel $\mathbf{h}_r^T \phase \mathbf{h}_t$ it is clear that
the impact of the phase matrix must diminish under the assumptions of free probability, \ie  as the number of antennas at transmitter and receiver increase asymptotically and that the component matrices of the channel are free. 
\par
While the realized capacity of the asymptotic channel converge towards the asymptotic capacity as $nt$ and $nr$ increase, the rate of this convergence is particularly relevant in this application as this implies that the number of rf-chains at the transmitter and receiver approach infinity. One notable observation is that this rate
decreases as the correlation in the channel increase. Under these circumstances, it may still be beneficial to perform some optimization of the phases at the IRS. 
\begin{itemize}
\item 
	
\end{itemize}
\subsection{Asymptotic Assumptions Under Correlation}
First mention that for iid random matrices, convergence is very fast (numerical). Then mention show for correlation within matrix. 

\chapter{Conclusion}
\chapter{Appendix}
Proofs
\begin{enumerate}
\item 
	Eigen/Singular Relationship
\item
	Prove freeness of the relevant points. 
\item 
	rotations have same non-zero eigenvalues
\item 
Proof of split using singular values.\\
Consider the channel given by 
\begin{equation}
\mH_{Total} = \mH_{2} + \mH_{1}.
\end{equation}
with covariance matrix 
\begin{equation}
\mC =
\mH_{2}\mathbf{H}_{2}^H + 
\mH_{1}\mathbf{H}_{1}^H + 
\mH_{2}\mathbf{H}_{1}^H +
\mH_{1}\mathbf{H}_{2}^H .
\end{equation}
Because the components of this sum are not free (CITE) (and some are non-hermetian), additive free convolution cannot be used.
We note that using the singular value decomposition $\mH_{Total} = \mU \boldsymbol{\Sigma} \mV^H$ we can
see that the covariance matrix can be written 
\begin{equation}
\mC = \mU \boldsymbol{\Sigma}\boldsymbol{\Sigma}^H \mU^H
= \mU \boldsymbol{\Lambda} \mU^H.
\end{equation}
With $\mC \succeq 0 \implies \lambda(\mC\mC)_i \geq 0$ we see that the definition of the singular values 
$\sigma(\mC)_i = \sqrt{\lambda(\mC\mC)_i}$ leads to an expression relating the density function of the singular values of a channel matrix
, $p_{\sigma}(x)$, and the eigenvalues of the channel covariance matrix, $ p_{\lambda}(x)$,  given by $p_{\sigma}(x^2) = p_{\lambda}(x)$.
Defining a new, symmetric density function of the singular value distribution
 	 \begin{equation}\label{symmetric}
 	 \tilde{p}_{\sigma}(x) = \frac{p_{\sigma}(x) + p_{\sigma}(-x)}{2},
 	 \end{equation}
 	 we now substitute $\tilde{p}_{\sigma}(x)$ into the definition of the Stieljes transform \eqref{stieltjes}
 	 to give the relationship
 	 	 	 \begin{align*}
	 	 \tilde{G}_\sigma (s) & =  \frac{1}{2} \int_{-\infty}^{\infty} \frac{p_{\sigma}(x) + p_{\sigma}(-x)}{x-s}dx
	 	 \\&  =  
	 	 \frac{1}{2} \int_{-\infty}^{\infty} \frac{dp_{\sigma}(x)}{x - s} + 
	 	 \frac{1}{2} \int_{-\infty}^{\infty} \frac{dp_{\sigma}(x)}{-x - s}
	 	 \\&  =  
	 	 	 	 \frac{1}{2} \int_{-\infty}^{\infty} \frac{-2s dp_{\sigma}(x)}{(-x - s)(x - s)}
	  	 \\&  =  
	 	 	 	\int_{-\infty}^{\infty} \frac{-s dp_{\sigma}(x)}{-x^2 + s^2} 	
 	 	  	 \\&  =  
 	 	 	-s  \int_{-\infty}^{\infty}  \frac{dp_{\sigma}(x)}{ x^2 - s^2} 	 
 	 	 \\&  =  
			-s  \int_{-\infty}^{\infty}  \frac{dp_{\lambda}(x)}{ x - s^2} 
	 	 \\&  = 
	 	 	-s G_{\lambda}(s^2)
	 	 \end{align*}
With the R-transform definition unchanged for the symmetric distribution, and the additional useful property (CITE)
\begin{equation}
\tilde{R}_{\mH_{Total}}(w) = \tilde{R}_{\mH_{2}}(w) + \tilde{R}_{\mH_{1}}
\end{equation}
We find that we only need the Stieltjes transform for the covariance matrix of each component of the sum which can in general be found using the tools
of multiplicative free convolution.

\item 
Proof of canceling phases.\\
Consider the channel given by 
\begin{equation}
\mathbf{H}_{Total} = \mathbf{H}_{2}\boldsymbol{\Phi}\mathbf{H}_{1}.
\end{equation}
This product of matrices suggests the use of multiplicative free convolution to solve for the AED using the S-Transform. 
Following the same procedure show in \cite{muller2002asymptotic}, we begin with the covariance matrix
\begin{equation}
\mC_2 = \mathbf{H}_{2}\boldsymbol{\Phi}\mathbf{H}_{1}\mathbf{H}_{1}^H\boldsymbol{\Phi}^H\mathbf{H}_{2}^H
\end{equation}
and note that this has the same non-zero eigenvalues as the matrix (Proof)
\begin{equation}
\tilde{\mC}_2 = \boldsymbol{\Phi}^H\mathbf{H}_{2}^H\mathbf{H}_{2}\boldsymbol{\Phi}\mathbf{H}_{1}\mathbf{H}_{1}^H.
\end{equation}
The relationship between the S-Transform of $\mC_2$ and $\tilde{\mC}_2$ is given by 
\begin{equation}\label{rotation_property}
S_{C_N}(z) = \frac{z+1}{z+\chi_2} S_{\tilde{C}_N}(\frac{z}{\chi_2}).
\end{equation}
With \begin{equation}
\mC_1 = \boldsymbol{\Phi}^H\mathbf{H}_{2}^H\mathbf{H}_{2}\boldsymbol{\Phi}
\end{equation}
we see that $S_{\tilde{\mC}_2}(z) = S_{\mC_1}(z) S_{\mathbf{H}_{1}\mathbf{H}_{1}^H}(z)$.  Repeating the above procedure on $\mC_1$ we find
\begin{equation}
\tilde{\mC}_{1} = \boldsymbol{\Phi}\boldsymbol{\Phi}^H\mathbf{H}_{2}^H\mathbf{H}_{2}
\end{equation}
with $\boldsymbol{\Phi}\boldsymbol{\Phi}^H = \mathbf{I}$ we are left with 
$S_{\mC_1}(z) =  S_{\mathbf{H}_{2}^H\mathbf{H}_{2}}(z)$. As a result we see that 
the phase matrix cancels. The same canceling applies to any unitary matrix $\mathbf{A}$ such that   $\mathbf{A}\mathbf{A}^\dagger =\mathbf{A}^\dagger \mathbf{A} = \mI$.
\end{enumerate}
\bibliography{bibliography}
\end{document}
