\documentclass[12pt,a4paper]{report}
\usepackage[utf8]{inputenc}
\usepackage{amsmath}
\usepackage{amsfonts}
\usepackage{amssymb}
\usepackage{graphicx}
\usepackage[colorlinks=true,linkcolor=blue]{hyperref}
\usepackage{float}
\input defs.tex
\bibliographystyle{alpha}
\graphicspath{ {./figures/} }
%\hyphenpenalty=10000

\title{Free Probability Analysis for Reflect Surface aided Massive MIMO systems}
\author{Peter Hartig}

\begin{document}
\maketitle
\begin{abstract}
Surfaces which can control the phase of reflected radio waves, known as intelligent reflective surfaces (IRS), have been the subject of interest in recent years as a means to better control the wireless propagation environment. This work investigates the impact of an IRS on the MIMO communication systems. Using the tools of free probability, a system is analyzed in the asymptotic domain to derive an expression for the system capacity. A key finding is that the phase control offered by the IRS has no impact in the asymptotic domain. Motivated by the contradiction between this finding and the substantial previous literature investigating the optimization of the IRS, we consider the convergence of a realistic system towards asymptotic behavior. We then propose an optimization method for the non-convex problem of minimizing the Mean Square Error with respect to the IRS phases.
\end{abstract}
%
%\newpage
\tableofcontents
\chapter{Introduction}
\section{Background}\label{Background}
The spectrum of electromagnetic frequencies used by wireless communication has been expanded to progressively higher frequencies with new standards. Industry research for the 6G standard has already investigated systems using frequency bands in the 120, 200 and 340 GHz range \cite{Koziol}. This shift is motivated by large bands of currently unoccupied frequency and the possibility for decreased antenna size (allowing for increased antenna density) \cite{akyildiz2018combating}. One primary hurdle towards efficient communication at these frequencies is the higher net attenuation of radio energy due both to scattering effects and molecular absorption. \cite{TODO} show through extensive data collection that relevant attenuation will be (x). 
As a relevant example, radar systems have long used this band of frequencies due to the resulting high resolution. For example, one such system operating in the 2 - 4 GHz band requires 25 kW of power for a detection range of 400 km \cite{TODO}. 
Countering this type of loss is a key step towards meeting the increased data rates and reliability required for future networks. 
\par
In previous generations of wireless communications, attenuation due to path loss has been alleviated by reducing the effective transmission distance 
and supplementing the transmission power using relays \cite{dahlman20134g}. Two key costs to deploying relays are the radio frequency chains used for receiving and transmitting, and the power used for transmission. Deploying enough relays to counter
the attenuation levels relevant to new standards would pose impractical costs. One new solution that has attracted significant interest is the use of surfaces made of many, sub-wavelength sized elements with favorable electromagnetic reflection and whose reflective coefficient can be controlled to select the phase of the reflected wave. In this work, such a surface will generally be referred to as a reconfigurable intelligent surface (RIS).
While this work focuses on the theory surrounding such surfaces, it is worth noting that these surfaces have been successfully implemented in works such as \cite{tan2018enabling} in which an RIS with 224 elements was built and tested. 
\par
The motivating use for the RIS is to alleviate the power attenuation problem without incurring the costs of an rf-chain and transmission power 
associated with a relay. The absence of an rf-chain on the RIS offers some benefits over relays. First, reflection does not introduce the noise associated with an rf-chain. Second, without the band-pass filters of an rf-chain, they have a frequency response over the entire spectrum (explain). Third, they are inherently full-duplex, a feature that is difficult to implement for co-located rf-chains on an active relay. 
The thorough review in \cite{basar2019wireless} provides rigorous motivation of the use of RISs and a few notable points are included here.
For a single antenna, single receiver channel using an RIS with $N$ reflective elements, the power at the receiver is can be proportional to $N^2$ (? Does this hold if we remove the ability to choose the angle?)(Also does this normalize the path power at all?). While such results are indeed motivating, it is important to note that scaling of received SNR proportional to $N^2$ will not hold asymptotically due to the conservation of transmitted energy \cite{bjornson2019demystifying}.
\par
Need to show how many surfaces (are the corresponding size) are needed in order to be on the same power gain order that is lost by the frequency increase (100dB).
\par
The previous section has shown that countering the increased attenuation of high frequencies using RIS and MIMO gains will require systems 
sizes on the order of x. Now motivate using asymptotic analysis.
\par
\par
Investigation leading to the current interest in MIMO systems showed the key property that (with constant transmit power), capacity (and mutual information) in an $N \times N$ MIMO system with independent channels scales linearly in $N$ \cite{foschini1998limits}. However the assumption of independent channels between each pair of transmit and receive antennas is not generally reasonable. This prompted much investigation into the impact of correlation to this scaling. In the following, a number of these studies are reviewed. These chosen works use channel correlation models similar to the one we will develop for our system model. In particular, this models the correlation between antennas as decreasing  exponentially with the distance between antennas. Definitely need to show this is a valid model (Cite).
\par
1. Looking at correlation for finite size systems.
In test data collected for (parameters) relevant to 3G and 4G wireless correlation between channels was shown to be as high as $0.5$ for a point-to-point, $4 \times 4$ system \cite{martin2000multiple} (corresponding capacity drop?). (Is this a bigger problem for higher frequencies?).
Explain why calculating capacity (even without correlation) for finite case is difficult/cumbersome. Describe early results for MIMO.
\par
One method that has been shown to reduce the complexity of this calculation is the assumption that $N \rightarrow \infty$. 
This is motivated by the observation that the eigenvalues of a channel covariance matrix converge to a deterministic distribution for many random 
matrices as  $N \rightarrow \infty$. In the remainder, this will be referred to as the Asymptotic Eigenvalue Distribution (AED). 
By making this assumption the impact of $N$ is only a scaling for some deterministic value. 
\par
In \cite{loyka2001channel}, an upper bound on the capacity is found using Jensen's inequality. For this upper bound, increasing correlation between antennas can be interpreted as a decrease in the average receiver SNR. 
Similarly, \cite{skupch2005free} used multiplicative free convolution from free probability theory to analyze the exponential correlation model at either the transmitting or receiving antenna. Part of this result will be used in our analysis. 

\cite{taricco2008asymptotic} analyzed the correlated ricean fading channel using the Replica Method. 
In general it is observed that the slope of the linear scaling of (mutual information) is reduced by the addition of correlation. 
One important point to note is that the above works do not consider the case of CSIT. When the water-filling capacity of the 
correlated MIMO channel is considered,  \cite{chuah2002capacity} notes an exception to this trend. For the case with low
SNR (what does this mean?) the water-filling capacity is actually increased with correlation.
\par
Before moving on to use any of these asymptomatic results, it is fitting to consider whether or not the assumption of  $N \rightarrow \infty$ is reasonable. Intuitively, we would expect that under correlation this convergence may occur only at very high $N$. This result was confirmed in  \cite{martin2004asymptotic} using the exponential correlation model at both the transmitter and receiver. 





\par
Motivated by the potential benefit of the RIS, we now investigate for the RIS system with correlation
\par
Now discuss all of the previous work looking at the optimization of the phases. 
General difficultly of the problem due to the non-convexity of the problem. Look at some different ways it's been applied.

\section{Notation}
The following notation is used throughout the remainder. 
$\Expect[x]$ is the expected value of a random variable $x$.
Bold font, lower-case letters ($\vx$) and upper-case letters ($\mX$) denote vector and matrices respectively and all vectors are assumed to be column vectors unless noted otherwise. 
The transpose and hermetian of $\mX$ are denoted by $\mX^T$ and $\mX^H$ respectively.
$\mI_N$ denotes the identity matrix of dimension $N$.
For square $\mX$ with dimension $N\times N$ the normalized trace is denoted by $\Tr(\mX)  = \underset{N \rightarrow \infty}{\text{lim}}
\Expect[\frac{1}{N}\sum_{i=1}^N x_{i,i}]$

\section{Outline}
The following sections are now previewed. Section \ref{system_model} covers the system model used throughout as well as a review of the necessary theory to characterize the resulting channel. In section \ref{Results}, the theoretical results and corresponding numerical results where appropriate are discussed. 

\chapter{System Model}\label{system_model}
\section{Power Normalization}
Consider the SISO channel with an RIS given by $y = \vh_{rs}^H \phase \vh_{st}$. In this case, $ \phase$ should be chosen optimally such that 
the phases all add constructively at the receiver (\ie $h_{rs,i} \phi_k h_{st,j}$ add constructively at the receiver). 
In the case of an i.i.d Rayleigh channel with elements $\CNO$ and $\sigma^2_{\text{noise}} =1$, the SNR at the receiver is given by $\gamma = N$.
Intuitively if the channel gains are normalized such that the received power is normalized for any system size $N$ the IRS offers no gain. 
In this case we can achieve this normalization by enforcing that elements of the a channel matrix $\mH \in \CN^{N \times K}$ have variance $\frac{1}{N}$.
This observation highlights an point that must be considered when performing asymptotic analysis; the received power must be normalized.


\section{The Correlated IRS Channel}
In the most general case we consider the channel given by 

	\begin{equation}
	\mathbf{H}_{Total} = \mathbf{R}_{R}^{\frac{1}{2}}(\underbrace{\mathbf{H}_{2}\boldsymbol{\Phi}\mathbf{R}_{S}^{\frac{1}{2}}\mathbf{H}_{1}}_{\text{IRS}} + \underbrace{\mathbf{G}}_{\text{LOS}})\mathbf{R}_{T}^{\frac{1}{2}}.
	\end{equation}

\section{Mutual Information}\label{sectiond:mut_info}
A key metric of interest for the channel described in the previous section is the mutual information between the transmitted and received signal. If the channel $\mHt$ is known at the receiver, the mutual information is given by 
\begin{equation}\label{mut_ent}
I(\vy;\vx) = h(\vy) - h(\vy|\vx).
\end{equation}
If both $\vx$ and $\vn$ are circularly symmetric Gaussian random vectors, $\vy$ is also circularly symmetric Gaussian with entropy given by \cite{telatar1999capacity} 
\begin{equation}\label{entropy}
h(\vy) = \Expect[\Log(|\pi e \mathbf{Q}_{y}|)]
\end{equation}
 for $\mathbf{Q}_{y} = \Expect{[\vy \vy^H]}$. Note also that for a given $\mathbf{Q}_{y}$ the entropy is maximized if and only if $\vy$ is circularly symmetric Gaussian.

For a given transmit covariance matrix, $\mathbf{Q_x}$ and $\Expect\left[\vn\vn^H \right] = \mI$, substituting equation \eqref{entropy} into equation
\eqref{mut_ent} and simplifying gives
\begin{equation}\label{mut_inf}
\Expect \left[\Log(|\mathbf{I} + \mHt \mQ_x \mHt^H|)\right].
\end{equation}
Using the model for $\mHt$ from the previous section, and maximizing with respect to both the
transmit covariance matrix and phases at the IRS gives an expression for the system capacity

\begin{equation}\label{capacity}
\mathbb{C} = \underset{\boldsymbol{\Phi},\mathbf{Q_x}}{\mathop{max}} \Expect \left[\Log(|\mathbf{I} + \mHt \mQ_x \mHt^H|)\right]
\end{equation}
or equivalently,
\begin{equation}\label{capacity_tricky}
\mathbb{C} = \underset{\boldsymbol{\Phi},\mathbf{Q_x}}{\mathop{max}} \Expect \left[\sum_{i=1}^{N}\Log(\mathbf{I} + \lambda_i)\right].
\end{equation}
where $\lambda_i$ are the eigenvalues of $\mHt \mQ_x \mHt^H$.

Recalling that in general we are interested in evaluating the capacity in order to determine a feasible transmission rate for a given communication channel, equation \eqref{capacity_tricky} poses a couple of hurdles to evaluation. First, we need to be able to evaluate an expectation over the joint pdf $p(\lambda_1 \cdots	 \lambda_N)$. Second, there is an optimization problem whose solution will generally depend on the distribution of the channel which would require a closed-form of the optimization problem solution. 
Two simplifications will be used in order to move forward. 
First, we will enforce that the transmitter chooses a deterministic $\mQ_x$. 
By enforcing a deterministic choice of $\mQ_x$, we no longer have to consider this component of the optimization.
Second we will normalize the channel, and therefore the received power, to allow the received signal covariance matrix eigenvalue distribution to converge. 



If $\mHt$ is not known at the transmitter, a common choice for the transmitter signal covariance matrix is  $\mQ_x = \frac{P_{\text{total}}}{N_t}\mathbf{I}$, giving the simplified form of equation \eqref{capacity}
\begin{equation}\label{no_csi_capacity}
\mathbb{C} = \underset{\phase}{\mathop{max}} \; N_R \Expect[\Log(1 + \frac{P_{\text{total}}}{\sigma_nN_t}\lambda_i)].
\end{equation}
where $\lambda_i$ are the eigenvalues of $\mHt \mHt^H$.
In this form, the linear scaling of capacity with N is clear.
If we assume that $\Expect\left[ \vy \vy^H \right]$ has a converging eigenvalue distribution as the system dimensions increase to infinity, we can expand the expectation to get the expression
\begin{equation}\label{no_csi_capacity_aed}
\mathbb{C} = \underset{\phase}{\mathop{max}} \; N_R  \int_{x=0}^{\infty}\Log(1 + \frac{1}{\sigma_n}\lambda)p_{\lambda\lambda^H}(x) dx
\end{equation}
in which the bounds of the integral reflect the range for the eigenvalues of a positive semidefinite matrix. In the following we will refer to this distribution as the Asymptotic Eigenvalue Distribution (AED).

From this, we see that if the normalized channel has a converging eigenvalue distribution we can simply scale the capacity for a single channel by the system size $N_R$. 

Note that because the choice of $h(\vx) $ and $h(\vn)$ maximize $h(\vy)$, equation \eqref{capacity} provides an upper bound on the capacity of the channel for any choices of the distribution of $\vx $ with Gaussian noise $\vn$. (Perhaps mention that normally we cannot generate continuous Gaussian x so we use channel coding).


Note that because $\Log$ is a concave function, we can upper bound the mutual information with
\begin{equation}\label{mut_inf}
\Expect[\Log(1 + \frac{1}{\sigma_n}\lambda)] \leq \Log(1 + \frac{1}{\sigma_n}\Expect[\lambda])
\end{equation}


\section{Explanation of Asymptotic Analysis}
Equation \eqref{no_csi_capacity_aed} highlights the important point that knowledge of the eigenvalue distribution of the received signal co-variance matrix is sufficient information to find the capacity of a MIMO channel. As we are generally interested in random channels, finding this eigenvalue distribution is not a straightforward task.


\section{Motivation for AED as a projection of Matrices}
Equation \ref{no_csi_capacity_aed} shows that the capacity of a communication channel can be found using the AED of the channel covariance matrix.
For matrices with converging eigenvalue distributions we can use $p_{\lambda\lambda^H} = \Expect[\Tr{(mHt \mHt^H})] = $.
One property that will be useful in the following is the relationship between the eigenvalues of a channel covariance matrix and the singular values of
the channel.
For a general matrix $\mH \in \complex^{n \times m}$ the singular values are given by 
\begin{equation}\label{sigular_eigen}
\sigma(\mH)_i = \sqrt{\lambda({\mH\mH^H}_i)}.
\end{equation}
This also gives the relationship between the pdf for the singular and and eigenvalues $p_{\sigma}(x^2) = p_{\lambda}(x)$
\section{Free Probability}\label{Free_Prob_Intro}
\begin{itemize}

\item 
	A brief introduction to free probability.
	\par 
	Free probability is the counterpart to classical probability for the case in which random variables (in this case random matrices) do 
	not commute. In the same way that classical free probability provides methods to find the pdf for the products and sums of commuting random
	 variables, free probability provides tools to find the pdfs for products and sums of non-commuting random variables. It is important to note
	 that non-commutativity of matrices affects not only the pdf of matrix multiplications but also matrix sums. 
	In the following work, a number of tools from free probability will be used. These tools are now introduced with appropriate context and intuition 
	for this investigation but for a thorough coverage of these topics, see \cite{tulino2004random} and \cite{mingo2017free}.
	 \begin{itemize}
	 \item 
	 	The cornerstone of free probability is the concept freeness of two random variables in the context of a projection. In our case, freeness between two 
	 	matrices is given by 
	 	We now introduce useful tools from free probability  that can be applied to free random variables.

	 	\item 
	 	Stieltjes Transform: 
	 	Generally, the projection of matrix is not given in closed form. The projection can be uniquely represented using
		\begin{equation}\label{stieltjes}
		G(s) = \int_{-\infty}^{\infty} \frac{dp(x)}{(x - s)} \; \text{for} \; \img(s) > 0
		\end{equation}	
		This representation will allow us to find the projection of the sum and project of matrices using addition and multiplication.
		\item 
		R-Transform:
		Mention self-adjoint
		Given the Stieltjes transform of a RV, the corresponding R-Transform given by 
		X. 
		The R-Transform of the sum of two, free RVs can be found using
		Y
		This process use known as free additive convolution.
		\item
		S-Transform:
		Mention self-adjoint
		Similarly, given the Stieltjes transform of a RV, the S-Transform given by 
		X, X;
		The S-Transform of the product of two, free RVs can be found using
		Y
		This process is known as free multiplicative convolution.
	 	\item 
	 	If there is a case in which finding the singular value distribution is easier than the eigenvalue distribution we can use equation  
	 	 \eqref{sigular_eigen} to move between the two. Considering a symmetric function of the singular value distribution
	 	 \begin{equation}\label{symmetric}
	 	 \tilde{p}_{\sigma}(x) = \frac{p_{\sigma}(x) + p_{\sigma}(-x)}{2}
	 	 \end{equation}
	 	 the definition of the Steiltjes transform gives the following steps, leading to a relationship between the Stieltjes transforms. 
	 	 \begin{align*}
	 	 \tilde{G}_\sigma (s) & =  \frac{1}{2} \int_{-\infty}^{\infty} \frac{p_{\sigma}(x) + p_{\sigma}(-x)}{x-s}
	 	 \\&  =  
	 	 \frac{1}{2} \int_{-\infty}^{\infty} \frac{dp_{\sigma}(x)}{x - s} + 
	 	 \frac{1}{2} \int_{-\infty}^{\infty} \frac{dp_{\sigma}(x)}{-x - s}
	 	 \\&  =  
	 	 	 	 \frac{1}{2} \int_{-\infty}^{\infty} \frac{-2s dp_{\sigma}(x)}{(-x - s)(x - s)}
	  	 \\&  =  
	 	 	 	\int_{-\infty}^{\infty} \frac{-s dp_{\sigma}(x)}{-x^2 + s^2} 	
 	 	  	 \\&  =  
 	 	 	-s  \int_{-\infty}^{\infty}  \frac{dp_{\sigma}(x)}{ x^2 - s^2} 	 
 	 	 \\&  =  
			-s  \int_{-\infty}^{\infty}  \frac{dp_{\lambda}(x)}{ x - s^2} 
	 	 \\&  = 
	 	 	-s G_{\lambda}(s^2)
	 	 \end{align*}
	 	\item 
	 		Another useful property that will be used is the relationship between the eigenvalues of a matrix and its rotation.
	 		Specifically, consider the covariance matrix given by $\mH_2\mH_1 \mH1^H \mH2^H$.
	 		By the rotational property of the trace, 
	 		\begin{align*}
	 		trace({\mH_2\mH_1 \mH1^H \mH2^H }) = \trace({\mH_1^H \mH_2^H \mH_2 \mH_1 }).
	 		\end{align*}
	 		... Not sure if I need to prove this here.  
	 	\item 
	  	build up to show that phases cancel via s-transform
	 	Stieltjes of AED (IE for hermetian matrices)
	 	Stieltjes of SVD (when they are components covariance matrix)
	 	Getting Steiltes of SVD using R-Transform
	 	Finding R-Transform of Individual components
	 	Using Stieltjes of individuals to get R-transforms
	 	Using aed stieltjes to get svd individual
	 	Using S-transform to get stieltjes of individual aed.
	\end{itemize}	 	
	
	Now describe with S-Transform and show that it can have a singular distribution in the limit at we go to rank 1. 
		
\item 
	Show that at least for uncorrelated case, the matrices in the sum are free. 

\item 
	Give proofs in an appendix?
	
\end{itemize}

\section{The IRS Channel}
Using the approach for finding asyptotic eigenvalue distributions detailed in \ref{Free_Prob_Intro} we will then use the capacity analysis detailed in \ref{sectiond:mut_info}. We will now develop a channel model on which to perform this analysis. Specifically we develop a channel model that is representative of a MIMO communication system with an RIS. 
\par
In Section \cite{Background} previous studies were reviewed which showed the impact and relevance of correlation to massive MIMO systems, therefore we begin with an RIS channel given by 
	\begin{equation}
	\mH_{Total} = \mR_{R}^{\frac{1}{2}}(\underbrace{\mH_{2}\boldsymbol{\Phi}\mR_{S}^{\frac{1}{2}}\mH_{1}}_{\text{IRS}}.
	\end{equation}

The random matrices $\mH_{2}$ and $ \mH_{1}$ represent the channels from the transmitter to the IRS and from the IRS to the receiver respectively. Both are assumed to have i.i.d. elements with variance $\frac{1}{N}$.  The matrices $\mR_{R}, \mR_{S}$ and $\mR_{T}$ represent a deterministic model for the correlation at the receiver, IRS and transmitter respectively.
The resulting covariance matrix looks quite similar to the one used for the example in Section \ref{Free_Prob_Intro} but first we want to verify that 
the resulting components will be free. 


(TODO provide references to prove that the components are free)



(Will need to detail any constraints embedded in this system model)

\chapter{Results}\label{Results}
\section{Asymptotic Results}
Begin this section using the model from the last
The final part of this section should show that phases will always cancel out for free probability analysis. 

\subsection{Asymptotic Results Discussion}

We first consider the case in which all matrices of the channel given by (ref) are free. As shown in (Appendix), in this case the 
phase shifts at the RIS, $\phase$, cancels out of the AED expression. 
By considering the case in which a single transmitter and receiver communicate over the channel $\mathbf{h}_r^T \phase \mathbf{h}_t$ it is clear that
the impact of the phase matrix must diminish under the assumptions of free probability, \ie  as the number of antennas at transmitter and receiver increase asymptotically and that the component matrices of the channel are free. 
\par
While the realized capacity of the asymptotic channel converge towards the asymptotic capacity as $nt$ and $nr$ increase, the rate of this convergence is particularly relevant in this application as this implies that the number of rf-chains at the transmitter and receiver approach infinity. One notable observation is that this rate
decreases as the correlation in the channel increase. Under these circumstances, it may still be beneficial to perform some optimization of the phases at the IRS. 
\begin{itemize}
\item 
	
	Show that in the asymptotic case where free probability holds, the phase matrix no longer matters
	We first show that under assumptions of the channel which allow for use of free probability, $\phase$
	cancels out of \eqref{capacity}.
	
\item
	Show that numerical agrees with theoretical for uncorrelated 
\item
	Now consider the case where the assumptions of free probability no longer apply 
	Show how numerical has larger deviation from theoretic for the case of the correlated channel and perhaps 
	look at methods of phase optimization that can help counter this loss. Also need to show proof that for the
	correlated case, free probability assumptions no longer hold. 
\end{itemize}
\subsection{Asymptotic Assumptions Under Correlation}
First mention that for iid random matrices, convergence is very fast (numerical). Then mention show for correlation within matrix. 

\section{Non-Asymptotic Results}
In the previous section, two notable observations were made. First, that under the assumptions of Free Probability (Ref), the choice of $\phase$ at the 
RIS has no impact on the capacity of the system. Second, that for matrices of random elements, the  convergence of matrix behavior towards that predicted by Free Probability is slowed by adding correlations between elements. As seen in \ref{Background} current large systems have been shown for 
sizes (X). We now consider whether this size is reasonable under correlations realized in systems (NEED TO FIND and justify). 

\chapter{Conclusion}
\chapter{Appendix}
Proofs
\begin{enumerate}
\item
	Prove freeness of the relevant points. 
\item 
	rotations have same non-zero eigenvalues
\item 
Proof of split using singular values.\\
Consider the channel given by 
\begin{equation}
\mH_{Total} = \mH_{2} + \mH_{1}.
\end{equation}
with covariance matrix 
\begin{equation}
\mC =
\mH_{2}\mathbf{H}_{2}^H + 
\mH_{1}\mathbf{H}_{1}^H + 
\mH_{2}\mathbf{H}_{1}^H +
\mH_{1}\mathbf{H}_{2}^H .
\end{equation}
Because the components of this sum are not free (CITE) (and some are non-hermetian), additive free convolution cannot be used.
We note that using the singular value decomposition $\mH_{Total} = \mU \boldsymbol{\Sigma} \mV^H$ we can
see that the covariance matrix can be written 
\begin{equation}
\mC = \mU \boldsymbol{\Sigma}\boldsymbol{\Sigma}^H \mU^H
= \mU \boldsymbol{\Lambda} \mU^H.
\end{equation}
With $\mC \succeq 0 \implies \lambda(\mC\mC)_i \geq 0$ we see that the definition of the singular values 
$\sigma(\mC)_i = \sqrt{\lambda(\mC\mC)_i}$ leads to an expression relating the density function of the singular values of a channel matrix
, $p_{\sigma}(x)$, and the eigenvalues of the channel covariance matrix, $ p_{\lambda}(x)$,  given by $p_{\sigma}(x^2) = p_{\lambda}(x)$.
Defining a new, symmetric density function of the singular value distribution
 	 \begin{equation}\label{symmetric}
 	 \tilde{p}_{\sigma}(x) = \frac{p_{\sigma}(x) + p_{\sigma}(-x)}{2},
 	 \end{equation}
 	 we now substitute $\tilde{p}_{\sigma}(x)$ into the definition of the Stieljes transform \eqref{stieltjes}
 	 to give the relationship
 	 	 	 \begin{align*}
	 	 \tilde{G}_\sigma (s) & =  \frac{1}{2} \int_{-\infty}^{\infty} \frac{p_{\sigma}(x) + p_{\sigma}(-x)}{x-s}dx
	 	 \\&  =  
	 	 \frac{1}{2} \int_{-\infty}^{\infty} \frac{dp_{\sigma}(x)}{x - s} + 
	 	 \frac{1}{2} \int_{-\infty}^{\infty} \frac{dp_{\sigma}(x)}{-x - s}
	 	 \\&  =  
	 	 	 	 \frac{1}{2} \int_{-\infty}^{\infty} \frac{-2s dp_{\sigma}(x)}{(-x - s)(x - s)}
	  	 \\&  =  
	 	 	 	\int_{-\infty}^{\infty} \frac{-s dp_{\sigma}(x)}{-x^2 + s^2} 	
 	 	  	 \\&  =  
 	 	 	-s  \int_{-\infty}^{\infty}  \frac{dp_{\sigma}(x)}{ x^2 - s^2} 	 
 	 	 \\&  =  
			-s  \int_{-\infty}^{\infty}  \frac{dp_{\lambda}(x)}{ x - s^2} 
	 	 \\&  = 
	 	 	-s G_{\lambda}(s^2)
	 	 \end{align*}
With the R-transform definition unchanged for the symmetric distribution, and the additional useful property (CITE)
\begin{equation}
\tilde{R}_{\mH_{Total}}(w) = \tilde{R}_{\mH_{2}}(w) + \tilde{R}_{\mH_{1}}
\end{equation}
We find that we only need the Stieltjes transform for the covariance matrix of each component of the sum which can in general be found using the tools
of multiplicative free convolution.

\item 
Proof of canceling phases.\\
Consider the channel given by 
\begin{equation}
\mathbf{H}_{Total} = \mathbf{H}_{2}\boldsymbol{\Phi}\mathbf{H}_{1}.
\end{equation}
This product of matrices suggests the use of multiplicative free convolution to solve for the AED using the S-Transform. 
Following the same procedure show in \cite{muller2002asymptotic}, we begin with the covariance matrix
\begin{equation}
\mC_2 = \mathbf{H}_{2}\boldsymbol{\Phi}\mathbf{H}_{1}\mathbf{H}_{1}^H\boldsymbol{\Phi}^H\mathbf{H}_{2}^H
\end{equation}
and note that this has the same non-zero eigenvalues as the matrix (Proof)
\begin{equation}
\tilde{\mC}_2 = \boldsymbol{\Phi}^H\mathbf{H}_{2}^H\mathbf{H}_{2}\boldsymbol{\Phi}\mathbf{H}_{1}\mathbf{H}_{1}^H.
\end{equation}
The relationship between the S-Transform of $\mC_2$ and $\tilde{\mC}_2$ is given by 
\begin{equation}\label{rotation_property}
S_{C_N}(z) = \frac{z+1}{z+\chi_2} S_{\tilde{C}_N}(\frac{z}{\chi_2}).
\end{equation}
With \begin{equation}
\mC_1 = \boldsymbol{\Phi}^H\mathbf{H}_{2}^H\mathbf{H}_{2}\boldsymbol{\Phi}
\end{equation}
we see that $S_{\tilde{\mC}_2}(z) = S_{\mC_1}(z) S_{\mathbf{H}_{1}\mathbf{H}_{1}^H}(z)$.  Repeating the above procedure on $\mC_1$ we find
\begin{equation}
\tilde{\mC}_{1} = \boldsymbol{\Phi}\boldsymbol{\Phi}^H\mathbf{H}_{2}^H\mathbf{H}_{2}
\end{equation}
with $\boldsymbol{\Phi}\boldsymbol{\Phi}^H = \mathbf{I}$ we are left with 
$S_{\mC_1}(z) =  S_{\mathbf{H}_{2}^H\mathbf{H}_{2}}(z)$. As a result we see that 
the phase matrix cancels. The same canceling applies to any unitary matrix $\mathbf{A}$ such that   $\mathbf{A}\mathbf{A}^\dagger =\mathbf{A}^\dagger \mathbf{A} = \mI$.
\end{enumerate}
\bibliography{bibliography}
\end{document}
