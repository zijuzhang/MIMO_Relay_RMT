\documentclass[12pt,a4paper]{report}
\usepackage[utf8]{inputenc}
\usepackage{amsmath}
\usepackage{amsfonts}
\usepackage{amssymb}
\usepackage{graphicx}
\usepackage[colorlinks=true,linkcolor=blue]{hyperref}
\usepackage{float}
\input defs.tex
\bibliographystyle{alpha}
\graphicspath{ {./figures/} }
%\hyphenpenalty=10000

\title{Free Probability Analysis for Reflect Surface aided Massive MIMO systems}
\author{Peter Hartig}

\begin{document}
\maketitle
\begin{abstract}
Surfaces which can control the phase of reflected radio waves, known as intelligent reflective surfaces (IRS), have been the subject of interest in recent years as a means to better control the wireless propagation environment. This work investigates the impact of an IRS on the MIMO communication systems. Using the tools of free probability, a system is analyzed in the asymptotic domain to derive an expression for the system capacity. A key finding is that the phase control offered by the IRS has no impact in the asymptotic domain. Motivated by the contradiction between this finding and the substantial previous literature investigating the optimization of the IRS, we consider the convergence of a realistic system towards asymptotic behavior. We then propose an optimization method for the non-convex problem of minimizing the Mean Square Error with respect to the IRS phases.
\end{abstract}
%
%\newpage
\tableofcontents
\chapter{Introduction}
\section{Background}\label{Background}
The spectrum of electromagnetic frequencies used by wireless communication has expanded to progressively higher frequencies with new standards. Industry research for 6G standard has already investigated 120, 200 and 340 GHz \cite{Koziol}. This expansion is motivated by large bands of currently unused frequency
and the possibility of decreased antenna size (allowing for increased antenna density) \cite{akyildiz2018combating}. A primary hurdle towards efficient communication at these frequencies is an increased attenuation of radio energy. For frequency bands relevant for upcoming wireless standards, energy attenuation due to free space path loss  molecular absorption loss
has been shown to exceed 100 dB. Countering this loss is a key steps towards meeting the increased data rates and reliability required for future networks.
\par
In previous generations of wireless communications, the problem of (power) attenuation has been alleviated by reducing the effective transmission distance 
and supplementing the transmission power using relays \cite{dahlman20134g}. Two key costs to deploying relays are the radio frequency chains used for receiving and transmitting, and the power used for transmission. (Would be nice to cite some actual costs here). Deploying this solution on a scale to counter
the attenuation levels relevant to new standards would pose impractical costs (why). One new solution that has attracted significant interest (check if deployed yet) are surfaces made of many elements with favorable reflective properties for high frequency electromagnetic waves (detail) and whose reflective coefficients are controlled to enable active selection of the phase of reflected waves. In this work, such a surface will generally be referred to as a reconfigurable intelligent surface (RIS).
\par
The motivating use for the RIS is to alleviate the power attenuation problem without incurring the costs of an rf-chain and transmission power 
associated with a relay. The absence of an rf-chain in the RIS offers additional benefits over relays. First, reflection does not introduce the noise associated with the rf-chain. Second, without the band-pass filters of an rf-chain, they have a frequency response over the entire spectrum (explain). Third, they are inherently full-duplex, a feature that is difficult to implement for co-located rf-chains on a relay. 
The thorough review in \cite{basar2019wireless} provides rigorous motivation of the use of RISs and a few notable points are included here.
For a single antenna, single receiver channel using an RIS with $N$ reflective elements, the power at the receiver is can be proportional to $N^2$ (? Does this hold if we remove the ability to choose the angle?).
For frequencies used in the current 5G standard, 30 GHz, an RIS with $100$ elements would have dimensions $1$ $\text{m}^2$.
(Why can we just ignore the fact that we might be able to change the reflection angle? In \cite{basar2019wireless} they introduce the idea with the ability to optimize the angle by then ignore it?).
Example of how these surfaces are implemented (Cite).
Discuss how it has be looked at from MISO and MIMO perspective.
\par
Need to show how many surfaces are needed in order to be on the same power gain order that is lost by the frequency increase (100dB).
\par
The previous section has shown that countering the increased attenuation of high frequencies using RIS and MIMO gains will requires systems 
sizes on the order of x. Now motivate using asymptotic analysis.
\par
\par
Early investigation which lead to the current interest in MIMO systems showed the key property that with constant transmit power, the capacity (and mutual information) in an $N \times N$ MIMO system with independent, i.i.d Gaussian channels scales linearly in $N$ \cite{foschini1998limits}. While the assumption of \emph{identically distributed} channels is reasonable when adding more antennas, the assumption of \emph{independent} channels between each pair of transmit and receive antennas is not. To understand the potential for leveraging this capacity scaling in deployed massive MIMO systems, the investigation of correlated MIMO channels has gained substantial interest. In the following, a number of these studies are reviewed. Some of these chosen works use channel correlation models similar to the one we will develop for the RIS system model. In particular, this method models the correlation between antennas as decreasing  exponentially with the distance between antennas. Definitely need to show this is a valid model (Cite).
\par
Might need to give more motivation for the use of asymptotic analysis.
\par
One method that has been shown to reduce the complexity of this calculation is to allow the system size $N \rightarrow \infty$ while normalizing the received power. 
This is motivated by the observation that when the random channel is properly normalized, the eigenvalues of the channel covariance matrix often converge to a deterministic distribution which will be referred to as the Asymptotic Eigenvalue Distribution (AED). 
Because the received power would generally not be normalized as the system size in increased, results found while using this normalization be extended by simply scaling the asymptotic result with $N$.
(Still need to clarify the types of MIMO gains that are observed under this normalization and how to deal with non-zero mean components).
\par
Evaluating the capacity for the case in which the transmitter has channel state information (CSIT ) can be difficult as the solution to this optimization is not generally in a closed-form. In \cite{loyka2001channel}, this problem is avoided using an upper bound on the capacity of a general MIMO channel derived from Jensen's inequality. Using this upper bound and the exponential correlation model, an interpretation is given in which increasing correlation between antennas corresponds to a decrease in the average SNR at the receiver. 
Similarly, \cite{chuah2002capacity} investigates the exponentially correlated MIMO system using close approximations to compares optimal versus equal power distribution at the transmitter for cases with high and low SNR. One interesting finding from this work is that for the case with low
SNR (what does this mean?) the water-filling capacity is actually increased with correlation while in all other cases, correlation has a negative impact on the capacity. The intuition because this finding is that the equal power case is only impacted by the SNR and not the distribution of the channel eigenvalues (unlike the water-filling case).
Finally, \cite{skupch2005free} used multiplicative free convolution from free probability theory to analyze the exponential correlation model at either the transmitting or receiving antenna. Similarly, they also observe a decrease in the capacity as correlation increase but only at high values of correlation in the system. Part of this result will be used in our analysis. 
Finally,
\cite{taricco2008asymptotic} analyzed the correlated Ricean fading channel using the Replica Method. 
In general it is observed that the slope of the linear scaling of (mutual information) is reduced by the addition of correlation. 
\par
Before moving on to apply any of these asymptomatic results with our own system model, it is fitting to consider two questions. 
First, whether or not the assumption of  $N \rightarrow \infty$ is reasonable? Intuitively, we would expect that under correlation this convergence may occur only at very high $N$. This result was confirmed in  \cite{martin2004asymptotic} using the exponential correlation model at both the transmitter and receiver. 
Second, what values for the parameters of the correlation model are reasonable for a real-world system?
In test data collected with channel parameters (carrier frequency, bandwidth, power...) relevant to 3G and 4G, wireless correlation between channels was shown to be as high as $0.5$ for a point-to-point, $4 \times 4$ (linear array?) system \cite{martin2000multiple} (TODO justify if this is a comparison for current systems).
\par
Motivated by the potential benefit of the RIS, we now investigate for the RIS system with correlation

\par
Now discuss all of the previous work looking at the optimization of the phases. 
General difficultly of the problem due to the non-convexity of the problem. Look at some different ways it's been applied.

\section{Notation}
The following notation is used throughout the remainder. 
$\Expect[x]$ is the expected value of a random variable $x$.
Bold font, lower-case letters ($\vx$) and upper-case letters ($\mX$) denote vector and matrices respectively and all vectors are assumed to be column vectors unless noted otherwise. 
The transpose and hermetian of $\mX$ are denoted by $\mX^T$ and $\mX^H$ respectively.
$\mI_N$ denotes the identity matrix of dimension $N$.
For square $\mX$ with dimension $N\times N$ the normalized trace is denoted by $\Tr(\mX)  = \underset{N \rightarrow \infty}{\text{lim}}
\Expect[\frac{1}{N}\sum_{i=1}^N x_{i,i}]$

\section{Outline}
The following sections are now previewed. In section \ref{system_model} covers the system model used throughout as well as a review of the necessary theory to characterize the resulting channel. In section \ref{Results}, the theoretical results and corresponding numerical results where appropriate are discussed. 

\chapter{System Model}\label{system_model}
\section{The Correlated IRS Channel}
In the most general case we consider the channel given by 

	\begin{equation}
	\mathbf{H}_{Total} = \mathbf{R}_{R}^{\frac{1}{2}}(\underbrace{\mathbf{H}_{2}\boldsymbol{\Phi}\mathbf{R}_{S}^{\frac{1}{2}}\mathbf{H}_{1}}_{\text{IRS}} + \underbrace{\mathbf{G}}_{\text{LOS}})\mathbf{R}_{T}^{\frac{1}{2}}.
	\end{equation}

\section{Mutual Information}\label{sectiond:mut_info}
A key metric of interest for the channel described in the previous section is the mutual information between the transmitted and received signal. If the channel $\mHt$ is known at the receiver, the mutual information is given by 
\begin{equation}\label{mut_ent}
I(\vy;\vx) = h(\vy) - h(\vy|\vx).
\end{equation}
If both $\vx$ and $\vn$ are circularly symmetric Gaussian random vectors, $\vy$ is circularly symmetric Gaussian with entropy given by \cite{telatar1999capacity} 
\begin{equation}\label{entropy}
h(\vy) = \Expect[\Log(|\pi e \mathbf{Q}_{y}|)]
\end{equation}
 for $\mathbf{Q}_{y} = \Expect{[\vy \vy^H]}$. Note also that for a given $\mathbf{Q}_{y}$ the entropy is maximized if and only if $\vy$ is circularly symmetric Gaussian.

For a given transmit covariance matrix, $\mathbf{Q_x}$ and $\Expect[\vn\vn^H] = \mI$, substituting equation \eqref{entropy} into equation
\eqref{mut_ent} gives
\begin{equation}\label{mut_inf}
\Expect [\Log(|\mathbf{I} + \mHt \mQ_x \mHt^H|)].
\end{equation}
Using the model for $\mHt$ from the previous section, and maximizing with respect to the
transmit covariance matrix and the phases at the IRS gives and expression for the capacity of the system

\begin{equation}\label{capacity}
\mathbb{C} = \underset{\boldsymbol{\Phi},\mathbf{Q_x}}{\mathop{max}} \Expect [\Log(|\mathbf{I} + \mHt \mQ_x \mHt^H|)].
\end{equation}
Note that because the choice of $h(\vx) $ and $h(\vn)$ maximize $h(\vy)$, equation \eqref{capacity} provides an upper bound on the capacity of the channel for any choices of $h(\vx) $ and $h(\vn)$.

If $\mHt$ is not known at the transmitter, the transmit power can be allocated
equally such that $\mQ_x = \mathbf{I}$. This allows for simplifying equation \eqref{capacity} to 
\begin{equation}\label{no_csi_capacity}
\mathbb{C} = \underset{\phase}{\mathop{max}} = N_R \Expect[\Log(1 + \frac{1}{\sigma_n}\lambda_i)]
\end{equation}
where $\lambda_i$ are the eigenvalues of $\mHt \mHt^H$.
Notice that in this form, the linear scaling of capacity with N is clear.
Expanding the expectation gives the expression
\begin{equation}\label{no_csi_capacity_aed}
\mathbb{C} = \underset{\phase}{\mathop{max}} = N_R  \int_{x=0}^{\infty}\Log(1 + \frac{1}{\sigma_n}\lambda_i)p_{\lambda\lambda^H}(x) dx
\end{equation}
in which the bounds of the integral represent the range for the eigenvalues of a positive semidefinite matrix. 

Note that because $\Log$ is a concave function, we can upper bound the mutual information with
\begin{equation}\label{mut_inf}
\Expect[\Log(1 + \frac{1}{\sigma_n}\lambda_i)] \leq \Log(1 + \frac{1}{\sigma_n}\Expect[\lambda_i])
\end{equation}
Now discuss capacity and consider equal power but optimize over phase
We now want to consider the case in 

\section{Motivation for AED as a projection of Matrices}
Equation \ref{no_csi_capacity_aed} shows that the capacity of a communication channel can be found using the AED of the channel covariance matrix.
For matrices with converging eigenvalue distributions we can use $p_{\lambda\lambda^H} = Expect[\Tr{(mHt \mHt^H})] = $.
One property that will be useful in the following is the relationship between the eigenvalues of a channel covariance matrix and the singular values of
the channel.
For a general matrix $\mH \in \complex^{n \times m}$ the singular values are given by 
\begin{equation}\label{sigular_eigen}
\sigma(\mH)_i = \sqrt{\lambda({\mH\mH^H}_i)}.
\end{equation}
This also gives the relationship between the pdf for the singular and and eigenvalues $p_{\sigma}(x^2) = p_{\lambda}(x)$
\section{Proof of Freeness}
\begin{itemize}

\item 
	A brief introduction to free probability.
	\par 
	Free probability is the counterpart to classical probability for the case in which random variables (in this case random matrices) do 
	not commute. In the same way that classical free probability provides methods to find the pdf for the products and sums of commuting random
	 variables, free probability provides tools to find the pdfs for products and sums of non-commuting random variables. It is important to note
	 that non-commutativity of matrices affects not only the pdf of matrix multiplications but also matrix sums. 
	In the following work, a number of tools from free probability will be used. These tools are now introduced with appropriate context and intuition 
	for this investigation but for a thorough coverage of these topics, see \cite{tulino2004random} and \cite{mingo2017free}.
	 \begin{itemize}
	 \item 
	 	The cornerstone of free probability is the concept freeness of two random variables in the context of a projection. In our case, freeness between two 
	 	matrices is given by 
	 	We now introduce useful tools from free probability  that can be applied to free random variables.

	 	\item 
	 	Stieltjes Transform: 
	 	Generally, the projection of matrix is not given in closed form. The projection can be uniquely represented using
		\begin{equation}\label{stieltjes}
		G(s) = \int_{-\infty}^{\infty} \frac{dp(x)}{(x - s)} \; \text{for} \; \img(s) > 0
		\end{equation}	
		This representation will allow us to find the projection of the sum and project of matrices using addition and multiplication.
		\item 
		R-Transform:
		Mention self-adjoint
		Given the Stieltjes transform of a RV, the corresponding R-Transform given by 
		X. 
		The R-Transform of the sum of two, free RVs can be found using
		Y
		This process use known as free additive convolution.
		\item
		S-Transform:
		Mention self-adjoint
		Similarly, given the Stieltjes transform of a RV, the S-Transform given by 
		X, X;
		The S-Transform of the product of two, free RVs can be found using
		Y
		This process is known as free multiplicative convolution.
	 	\item 
	 	If there is a case in which finding the singular value distribution is easier than the eigenvalue distribution we can use equation  
	 	 \eqref{sigular_eigen} to move between the two. Considering a symmetric function of the singular value distribution
	 	 \begin{equation}\label{symmetric}
	 	 \tilde{p}_{\sigma}(x) = \frac{p_{\sigma}(x) + p_{\sigma}(-x)}{2}
	 	 \end{equation}
	 	 the definition of the Steiltjes transform gives the following steps, leading to a relationship between the Stieltjes transforms. 
	 	 \begin{align*}
	 	 \tilde{G}_\sigma (s) & =  \frac{1}{2} \int_{-\infty}^{\infty} \frac{p_{\sigma}(x) + p_{\sigma}(-x)}{x-s}
	 	 \\&  =  
	 	 \frac{1}{2} \int_{-\infty}^{\infty} \frac{dp_{\sigma}(x)}{x - s} + 
	 	 \frac{1}{2} \int_{-\infty}^{\infty} \frac{dp_{\sigma}(x)}{-x - s}
	 	 \\&  =  
	 	 	 	 \frac{1}{2} \int_{-\infty}^{\infty} \frac{-2s dp_{\sigma}(x)}{(-x - s)(x - s)}
	  	 \\&  =  
	 	 	 	\int_{-\infty}^{\infty} \frac{-s dp_{\sigma}(x)}{-x^2 + s^2} 	
 	 	  	 \\&  =  
 	 	 	-s  \int_{-\infty}^{\infty}  \frac{dp_{\sigma}(x)}{ x^2 - s^2} 	 
 	 	 \\&  =  
			-s  \int_{-\infty}^{\infty}  \frac{dp_{\lambda}(x)}{ x - s^2} 
	 	 \\&  = 
	 	 	-s G_{\lambda}(s^2)
	 	 \end{align*}
	 	\item 
	 		Another useful property that will be used is the relationship between the eigenvalues of a matrix and its rotation.
	 		Specifically, consider the covariance matrix given by $\mH_2\mH_1 \mH1^H \mH2^H$.
	 		By the rotational property of the trace, 
	 		\begin{align*}
	 		trace({\mH_2\mH_1 \mH1^H \mH2^H }) = \trace({\mH_1^H \mH_2^H \mH_2 \mH_1 }).
	 		\end{align*}
	 		... Not sure if I need to prove this here.  
	 	\item 
	  	build up to show that phases cancel via s-transform
	 	Stieltjes of AED (IE for hermetian matrices)
	 	Stieltjes of SVD (when they are components covariance matrix)
	 	Getting Steiltes of SVD using R-Transform
	 	Finding R-Transform of Individual components
	 	Using Stieltjes of individuals to get R-transforms
	 	Using aed stieltjes to get svd individual
	 	Using S-transform to get stieltjes of individual aed.
	\end{itemize}	 	
	
	Now describe with S-Transform and show that it can have a singular distribution in the limit at we go to rank 1. 
		
\item 
	Show that at least for uncorrelated case, the matrices in the sum are free. 

\item 
	Give proofs in an appendix?
	
\end{itemize}

\chapter{Results}\label{Results}
\section{Asymptotic Results}
We first consider the case in which all components of the channel given by (ref) are free. As shown in (Appendix), in this case the 
phase shifts at the RIS, $\phase$, cancels out of the AED expression. 
By considering the case in which a single transmitter and receiver communicate over the channel $\mathbf{h}_r^T \phase \mathbf{h}_t$ it is clear that
the impact of the phase matrix must diminish under the assumptions of free probability, \ie  as the number of antennas at transmitter and receiver increase asymptotically and that the component matrices of the channel are free. 
\par
While the realized capacity of the asymptotic channel converge towards the asymptotic capacity as $nt$ and $nr$ increase, the rate of this convergence is particularly relevant in this application as this implies that the number of rf-chains at the transmitter and receiver approach infinity. One notable observation is that this rate
decreases as the correlation in the channel increase. Under these circumstances, it may still be beneficial to perform some optimization of the phases at the IRS. 
\begin{itemize}
\item 
	
	Show that in the asymptotic case where free probability holds, the phase matrix no longer matters
	We first show that under assumptions of the channel which allow for use of free probability, $\phase$
	cancels out of \eqref{capacity}.
	
\item
	Show that numerical agrees with theoretical for uncorrelated 
\item
	Now consider the case where the assumptions of free probability no longer apply 
	Show how numerical has larger deviation from theoretic for the case of the correlated channel and perhaps 
	look at methods of phase optimization that can help counter this loss. Also need to show proof that for the
	correlated case, free probability assumptions no longer hold. 
\end{itemize}
\subsection{Asymptotic Assumptions Under Correlation}
First mention that for iid random matrices, convergence is very fast (numerical). Then mention show for correlation within matrix. 

\section{Non-Asymptotic Results}
In the previous section, two notable observations were made. First, that under the assumptions of Free Probability (Ref), the choice of $\phase$ at the 
RIS has no impact on the capacity of the system. Second, that for matrices of random elements, the  convergence of matrix behavior towards that predicted by Free Probability is slowed by adding correlations between elements. As seen in \ref{Background} current large systems have been shown for 
sizes (X). We now consider whether this size is reasonable under correlations realized in systems (NEED TO FIND and justify). 

\chapter{Conclusion}
\chapter{Appendix}
Proofs
\begin{enumerate}
\item
	Prove freeness of the relevant points. 
\item 
	rotations have same non-zero eigenvalues
\item 
Proof of split using singular values.\\
Consider the channel given by 
\begin{equation}
\mH_{Total} = \mH_{2} + \mH_{1}.
\end{equation}
with covariance matrix 
\begin{equation}
\mC =
\mH_{2}\mathbf{H}_{2}^H + 
\mH_{1}\mathbf{H}_{1}^H + 
\mH_{2}\mathbf{H}_{1}^H +
\mH_{1}\mathbf{H}_{2}^H .
\end{equation}
Because the components of this sum are not free (CITE) (and some are non-hermetian), additive free convolution cannot be used.
We note that using the singular value decomposition $\mH_{Total} = \mU \boldsymbol{\Sigma} \mV^H$ we can
see that the covariance matrix can be written 
\begin{equation}
\mC = \mU \boldsymbol{\Sigma}\boldsymbol{\Sigma}^H \mU^H
= \mU \boldsymbol{\Lambda} \mU^H.
\end{equation}
With $\mC \succeq 0 \implies \lambda(\mC\mC)_i \geq 0$ we see that the definition of the singular values 
$\sigma(\mC)_i = \sqrt{\lambda(\mC\mC)_i}$ leads to an expression relating the density function of the singular values of a channel matrix
, $p_{\sigma}(x)$, and the eigenvalues of the channel covariance matrix, $ p_{\lambda}(x)$,  given by $p_{\sigma}(x^2) = p_{\lambda}(x)$.
Defining a new, symmetric density function of the singular value distribution
 	 \begin{equation}\label{symmetric}
 	 \tilde{p}_{\sigma}(x) = \frac{p_{\sigma}(x) + p_{\sigma}(-x)}{2},
 	 \end{equation}
 	 we now substitute $\tilde{p}_{\sigma}(x)$ into the definition of the Stieljes transform \eqref{stieltjes}
 	 to give the relationship
 	 	 	 \begin{align*}
	 	 \tilde{G}_\sigma (s) & =  \frac{1}{2} \int_{-\infty}^{\infty} \frac{p_{\sigma}(x) + p_{\sigma}(-x)}{x-s}dx
	 	 \\&  =  
	 	 \frac{1}{2} \int_{-\infty}^{\infty} \frac{dp_{\sigma}(x)}{x - s} + 
	 	 \frac{1}{2} \int_{-\infty}^{\infty} \frac{dp_{\sigma}(x)}{-x - s}
	 	 \\&  =  
	 	 	 	 \frac{1}{2} \int_{-\infty}^{\infty} \frac{-2s dp_{\sigma}(x)}{(-x - s)(x - s)}
	  	 \\&  =  
	 	 	 	\int_{-\infty}^{\infty} \frac{-s dp_{\sigma}(x)}{-x^2 + s^2} 	
 	 	  	 \\&  =  
 	 	 	-s  \int_{-\infty}^{\infty}  \frac{dp_{\sigma}(x)}{ x^2 - s^2} 	 
 	 	 \\&  =  
			-s  \int_{-\infty}^{\infty}  \frac{dp_{\lambda}(x)}{ x - s^2} 
	 	 \\&  = 
	 	 	-s G_{\lambda}(s^2)
	 	 \end{align*}
With the R-transform definition unchanged for the symmetric distribution, and the additional useful property (CITE)
\begin{equation}
\tilde{R}_{\mH_{Total}}(w) = \tilde{R}_{\mH_{2}}(w) + \tilde{R}_{\mH_{1}}
\end{equation}
We find that we only need the Stieltjes transform for the covariance matrix of each component of the sum which can in general be found using the tools
of multiplicative free convolution.

\item 
Proof of canceling phases.\\
Consider the channel given by 
\begin{equation}
\mathbf{H}_{Total} = \mathbf{H}_{2}\boldsymbol{\Phi}\mathbf{H}_{1}.
\end{equation}
This product of matrices suggests the use of multiplicative free convolution to solve for the AED using the S-Transform. 
Following the same procedure show in \cite{muller2002asymptotic}, we begin with the covariance matrix
\begin{equation}
\mC_2 = \mathbf{H}_{2}\boldsymbol{\Phi}\mathbf{H}_{1}\mathbf{H}_{1}^H\boldsymbol{\Phi}^H\mathbf{H}_{2}^H
\end{equation}
and note that this has the same non-zero eigenvalues as the matrix (Proof)
\begin{equation}
\tilde{\mC}_2 = \boldsymbol{\Phi}^H\mathbf{H}_{2}^H\mathbf{H}_{2}\boldsymbol{\Phi}\mathbf{H}_{1}\mathbf{H}_{1}^H.
\end{equation}
The relationship between the S-Transform of $\mC_2$ and $\tilde{\mC}_2$ is given by 
\begin{equation}\label{rotation_property}
S_{C_N}(z) = \frac{z+1}{z+\chi_2} S_{\tilde{C}_N}(\frac{z}{\chi_2}).
\end{equation}
With \begin{equation}
\mC_1 = \boldsymbol{\Phi}^H\mathbf{H}_{2}^H\mathbf{H}_{2}\boldsymbol{\Phi}
\end{equation}
we see that $S_{\tilde{\mC}_2}(z) = S_{\mC_1}(z) S_{\mathbf{H}_{1}\mathbf{H}_{1}^H}(z)$.  Repeating the above procedure on $\mC_1$ we find
\begin{equation}
\tilde{\mC}_{1} = \boldsymbol{\Phi}\boldsymbol{\Phi}^H\mathbf{H}_{2}^H\mathbf{H}_{2}
\end{equation}
with $\boldsymbol{\Phi}\boldsymbol{\Phi}^H = \mathbf{I}$ we are left with 
$S_{\mC_1}(z) =  S_{\mathbf{H}_{2}^H\mathbf{H}_{2}}(z)$. As a result we see that 
the phase matrix cancels. The same canceling applies to any unitary matrix $\mathbf{A}$ such that   $\mathbf{A}\mathbf{A}^\dagger =\mathbf{A}^\dagger \mathbf{A} = \mI$.
\end{enumerate}
\bibliography{bibliography}
\end{document}
