\documentclass[12pt,a4paper]{report}
\usepackage[utf8]{inputenc}
\usepackage{amsmath}
\usepackage{amsfonts}
\usepackage{amssymb}
\usepackage{graphicx}
\usepackage{float}
\input defs.tex
\bibliographystyle{alpha}
\graphicspath{ {./figures/} }
%\hyphenpenalty=10000

\title{Free Probability Analysis for Reflect Surface aided Massive MIMO systems}
\author{Peter Hartig}

\begin{document}
\maketitle
\begin{abstract}
This work investigates the impact of passive reflective surfaces on the capacity of MIMO systems. In particular, the system is analyzed in the asymptotic size domain in order to employ the tools of free probability. A system is considered first in the case of fully uncorrelated channels. In this case it is shown that the passive control of such a reflective surface has no impact on the capacity of the system. The system is then generalized to allow for correlation within the channel. In this case is it shown that the selection of the phases at the IRS still has an impact on the system. Finally we compare the capacity of the correlated channel with optimized phases at the IRS to the uncorrelated channel to show that....?
\end{abstract}
%
%\newpage
\tableofcontents


\chapter{Introduction}
Wireless communication has progressively moved to higher frequencies in recent standards in order to utilize large bands of unoccupied resource
and to enable increased antenna density (CITE). One hurdle to higher frequencies is increased radio frequency energy (power) attenuation (CITE). 
In past generations of wireless communication, this problem of (power) attenuation has been alleviated by reducing the effective transmission distance 
and supplementing the transmission power using relays (CITE). Two key costs to deploying relays are the radio frequency chains used for receiving and transmitter, and the power used for transmission. (Would be nice to cite some actual costs here). As the power attenuation problem has been increased by X-Fold moving from freq1 to freq2, new solutions have been considered. One solution that has attracted significant interest (check if deployed yet) are surfaces made of many elements with favorable reflective properties for high frequency electromagnetic waves (detail) and whose reflective coefficients are controlled to enable active selection of the phase of reflected waves. In this work, such a surface will generally be referred to as a reconfigurable intelligent surface (RIS).
\par
The motivating use for the RIS is to alleviate the power attenuation problem without incurring the costs of an rf-chain and transmission power 
associated with a relay. The absence of an rf-chain in the RIS offers additional benefits over relays. First, reflection does not introduce the noise associated with the rf-chain. Second, without the band-pass filters of an rf-chain, they have a frequency response over the entire spectrum. Third, they are inherently full-duplex, a feature that is difficult to implement for co-located rf-chains on a relay. 
The thorough review in \cite{basar2019wireless} provides rigorous motivation of the use of RISs and a few notable points are included here.
For a single antenna, single receiver channel using an RIS with $N$ reflective elements, the power at the receiver is proportional to $N^2$ (? Does this hold if we remove the ability to choose the angle?).
For frequencies used in the current 5G standard, 30 GHz, an RIS with $100$ elements would have dimensions $1$ $\text{m}^2$.
(Why can we just ignore the fact that we might be able to change the reflection angle? In \cite{basar2019wireless} they introduce the idea with the ability to optimize the angle by then ignore it?).
Example of how these surfaces are implemented (Cite).
Discuss how it has be looked at from MISO and MIMO perspective.
\par
One area that has not be investigated, (at least up to the author's knowledge) is the impact of the RIS for 
asymptotically large MIMO systems. In the following, the impact of the RIS on a general MIMO system is explored. 

\section{Notation}
The following notation is used throughout the remainder. 
$\Expect[x]$ is the expected value of a random variable $x$.
Bold font, lower-case letters ($\vx$) and upper-case letters ($\mX$) denote vector and matrices respectively and all vectors are assumed to be column vectors unless noted otherwise. 
The transpose and hermetian of $\mX$ are denoted by $\mX^T$ and $\mX^H$ respectively.
$\mI_N$ denotes the identity matrix of dimension $N$.

\section{Outline}
The following sections are now previewed. In section \ref{system_model} covers the system model used throughout as well as a review of the necessary theory to characterize the resulting channel. In section \ref{Results}, the theoretical results and corresponding numerical results where appropriate are discussed. 
\chapter{System Model}\label{system_model}
\section{The Correlated IRS Channel}


\section{Mutual Information}
A key metric of interest for the channel described in the previous section is the mutual information between the transmitted and received signal. If the channel information is known at the receiver, this is given by 
\begin{equation}
I(\vy;\vx) = h(\vy) - h(\vy|\vx).
\end{equation}
If both $\vx$ and $\vn$ are circularly symmetric Gaussian random vectors, $\vy$ is circularly symmetric Gaussian with entropy given by \cite{telatar1999capacity} $h(\vy) = \Log(|\pi e \mathbf{Q}_{y}|)$ with $\mathbf{Q}_{y} = \Expect{[\vy \vy^H]}$. Note also that for a given $\mathbf{Q}_{y}$ the entropy is maximized if and only if $\vy$ is circularly symmetric Gaussian.

For a given transmit covariance matrix, $\mathbf{Q_x}$, the mutual information is given by 
\begin{equation}\label{mut_inf}
\Expect [\Log(|\mathbf{I} + \mHt \mQ_x \mHt^H|)].
\end{equation}
Using the model for $\mHt$ from the previous section, and maximizing with respect to the
transmit covariance matrix and the phases at the IRS gives and expression for the capacity of the system

\begin{equation}\label{capacity}
\mathbb{C} = \underset{\boldsymbol{\Phi},\mathbf{Q_x}}{\mathop{max}} \Expect [\Log(|\mathbf{I} + \mHt \mQ_x \mHt^H|)].
\end{equation}
Note that because the choice of $h(\vx) $ and $h(\vn)$ maximize $h(\vy)$, \eqref{capacity} provides an upper bound on the capacity of the channel for any choices of $h(\vx) $ and $h(\vn)$.

If $\mHt$ is not known at the transmitter, the transmit power can be allocated
equally such that $\mQ_x = \mathbf{I}$. This allows for simplifying equation \eqref{capacity} to 
\begin{equation}\label{no_csi_capacity}
\mathbb{C} = \underset{\phase}{\mathop{max}} = N_R \Expect[\Log(1 + \frac{1}{\sigma_n}\lambda_i)]
\end{equation}
where $\lambda_i$ are the eigenvalues of $\mHt \mHt^H$.

Note that because $\Log$ is a concave function, we can upper bound the mutual information with
\begin{equation}\label{mut_inf}
\Expect[\Log(1 + \frac{1}{\sigma_n}\lambda_i)] \leq \Log(1 + \frac{1}{\sigma_n}\Expect[\lambda_i])
\end{equation}
Now discuss capacity and consider equal power but optimize over phase
We now want to consider the case in 

\section{Proof of Freeness}
\begin{itemize}

\item 
	A brief introduction to free probability.
	In order to introduce the tools that will be used in the following, an example of non-commutative probability now given to
	exhibit a case in which the tools of classic probability are not sufficient for finding a probability density function (pdf).
	\par 
	Free probability is the counterpart to classical probability for the case in which the random variables (in this case random matrices) do 
	not commute. In the same way that classical free probability provides methods to find the pdf for the products and sums of commuting random
	 variables, free probability provides tools to find the pdfs for products and sums of non-commuting random variables (It is important to note here
	 that by changing this nature of the random variables from commutative to non-commutive (is this embedded in kolmogorov's axioms?) 
	In the following work, a number of tools from free probability will be used. These tools are now introduced with appropriate context 
	but for a thorough coverage of these topics, see \cite{tulino2004random} and \cite{mingo2017free}.
	Now describe with S-Transform and show that it can have a singular distribution in the limit at we go to rank 1. 
		
\item 
	Show that at least for uncorrelated case, the matrices in the sum are free. 

\item 
	Give proofs in an appendix?
	
	
\end{itemize}


\chapter{Results}\label{Results}
\begin{itemize}
\item 
	Show that in the asymptotic case where free probability holds, the phase matrix no longer matters
	We first show that under assumptions of the channel which allow for use of free probability, $\phase$
	cancels out of \eqref{capacity}.
	
\item
	Show that numerical agrees with theoretical for uncorrelated 
\item
	Now consider the case where the assumptions of free probability no longer apply 
	Show how numerical has larger deviation from theoretic for the case of the correlated channel and perhaps 
	look at methods of phase optimization that can help counter this loss. Also need to show proof that for the
	correlated case, free probability assumptions no longer hold. 
\end{itemize}



\chapter{Conclusion}

\chapter{Appendix}
Proofs
\begin{enumerate}
\item
	Prove freeness of the relevant points. 
\item 
	rotations have same non-zero eigenvalues
\end{enumerate}


\bibliography{bibliography}
\end{document}
