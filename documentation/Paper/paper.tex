\documentclass[12pt,a4paper]{report}
\usepackage[utf8]{inputenc}
\usepackage{amsmath}
\usepackage{amsfonts}
\usepackage{amssymb}
\usepackage{graphicx}
\usepackage{float}
\input defs.tex
\bibliographystyle{alpha}
\graphicspath{ {./figures/} }
%\hyphenpenalty=10000

\title{Free Probability Analysis for Reflect Surface aided Massive MIMO systems}
\author{Peter Hartig}

\begin{document}
\maketitle
\begin{abstract}
This work investigates the impact of passive reflective surfaces on the capacity of MIMO systems. In particular, the system is analyzed in the asymptotic size domain in order to employ the tools of free probability. A system is considered first in the case of fully uncorrelated channels. In this case it is shown that the passive control of such a reflective surface has no impact on the capacity of the system. The system is then generalized to allow for correlation within the channel. In this case is it shown that the selection of the phases at the IRS still has an impact on the system. Finally we compare the capacity of the correlated channel with optimized phases at the IRS to the uncorrelated channel to show that....?
\end{abstract}
%
%\newpage
\tableofcontents


\chapter{Introduction}



\section{Notation}
The following notation is used throughout the remainder. 
$E\{x\}$ is the expected value of a random variable $x$.
Vectors are denoted by bold font, lower-case letters ($\mathbf{x}$) and are assumed to be column vectors.
Matrices are denoted by bold font, upper-case letters ($\mathbf{X}$). The transpose and hermetian of $\mathbf{X}$ are denoted by $\mathbf{X}^T$ and $\mathbf{X}^H$ respectively.
$\mathbf{I}$ denotes the identity matrix.

\section{Outline}
The following sections are now previewed. In section \ref{system_model} covers the system model used throughout as well as a review of the necessary theory to characterize the resulting channel. In section \ref{Results}, the theoretical results and corresponding numerical results where appropriate are discussed. 
\chapter{System Model}\label{system_model}
\section{Outline}



\section{Proof of Freeness}


\chapter{Proofs etc. }\label{Results}

\chapter{Numerical Results}\label{numerical}

\chapter{Conclusion}

\bibliography{bibliography}
\end{document}
