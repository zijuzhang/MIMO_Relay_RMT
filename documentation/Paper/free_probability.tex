\begin{itemize}

\item 
	A brief introduction to free probability.
	\par 
	Free probability is the counterpart to classical probability used to analyze cases in which random variables (in this case random matrices) do 
	not commute. In the same way that classical probability provides methods to find the pdf for the sums and products of commuting random
	 variables, free probability provides tools to find the pdfs for products and sums of non-commuting random variables.
	In the following work, a number of tools from free probability will be used. These tools are now introduced with appropriate context and intuition 
	for this investigation but for a thorough coverage of these topics, see \cite{tulino2004random} and \cite{mingo2017free}.
	 \begin{itemize}
	 \item 
	 	The cornerstone of free probability is the concept freeness of two random variables in the context of a projection. In our case, freeness between two 
	 	matrices is given by 

	 	\item 
	 	Stieltjes Transform: 
	 	Generally, the projection of matrix is not given in closed form. The projection can be uniquely represented using
		\begin{equation}\label{stieltjes}
		G(s) = \int_{-\infty}^{\infty} \frac{dp(x)}{(x - s)} \; \text{for} \; \img(s) > 0
		\end{equation}	
		This representation will allow us to find the projection of the sum and project of matrices using addition and multiplication.
		\item 
		R-Transform:
		Mention self-adjoint
		Given the Stieltjes transform of a RV, the corresponding R-Transform given by 
		X. 
		The R-Transform of the sum of two, free RVs can be found using
		Y
		This process use known as free additive convolution.
		\item
		S-Transform:
		Mention self-adjoint
		Similarly, given the Stieltjes transform of a RV, the S-Transform given by 
		X, X;
		The S-Transform of the product of two, free RVs can be found using
		Y
		This process is known as free multiplicative convolution.
	 	\item 
	 	If there is a case in which finding the singular value distribution is easier than the eigenvalue distribution we can use equation  
	 	 \eqref{sigular_eigen} to move between the two. Considering a symmetric function of the singular value distribution
	 	 \begin{equation}\label{symmetric}
	 	 \tilde{p}_{\sigma}(x) = \frac{p_{\sigma}(x) + p_{\sigma}(-x)}{2}
	 	 \end{equation}
	 	 the definition of the Steiltjes transform gives the following steps, leading to a relationship between the Stieltjes transforms. 
	 	 \begin{align*}
	 	 \tilde{G}_\sigma (s) & =  \frac{1}{2} \int_{-\infty}^{\infty} \frac{p_{\sigma}(x) + p_{\sigma}(-x)}{x-s}
	 	 \\&  =  
	 	 \frac{1}{2} \int_{-\infty}^{\infty} \frac{dp_{\sigma}(x)}{x - s} + 
	 	 \frac{1}{2} \int_{-\infty}^{\infty} \frac{dp_{\sigma}(x)}{-x - s}
	 	 \\&  =  
	 	 	 	 \frac{1}{2} \int_{-\infty}^{\infty} \frac{-2s dp_{\sigma}(x)}{(-x - s)(x - s)}
	  	 \\&  =  
	 	 	 	\int_{-\infty}^{\infty} \frac{-s dp_{\sigma}(x)}{-x^2 + s^2} 	
 	 	  	 \\&  =  
 	 	 	-s  \int_{-\infty}^{\infty}  \frac{dp_{\sigma}(x)}{ x^2 - s^2} 	 
 	 	 \\&  =  
			-s  \int_{-\infty}^{\infty}  \frac{dp_{\lambda}(x)}{ x - s^2} 
	 	 \\&  = 
	 	 	-s G_{\lambda}(s^2)
	 	 \end{align*}
	 	\item 
	 		Another useful property that will be used is the relationship between the eigenvalues of a matrix and its rotation.
	 		Specifically, consider the covariance matrix given by $\mH_2\mH_1 \mH1^H \mH2^H$.
	 		By the rotational property of the trace, 
	 		\begin{align*}
	 		trace({\mH_2\mH_1 \mH1^H \mH2^H }) = \trace({\mH_1^H \mH_2^H \mH_2 \mH_1 }).
	 		\end{align*}
	 		... Not sure if I need to prove this here.  
	 	\item 
	  	build up to show that phases cancel via s-transform
	 	Stieltjes of AED (IE for hermetian matrices)
	 	Stieltjes of SVD (when they are components covariance matrix)
	 	Getting Steiltes of SVD using R-Transform
	 	Finding R-Transform of Individual components
	 	Using Stieltjes of individuals to get R-transforms
	 	Using aed stieltjes to get svd individual
	 	Using S-transform to get stieltjes of individual aed.
	\end{itemize}	 	
	
	Now describe with S-Transform and show that it can have a singular distribution in the limit at we go to rank 1. 
		
\item 
	Show that at least for uncorrelated case, the matrices in the sum are free. 

\item 
	Give proofs in an appendix?
	
\end{itemize}