Using the approach from Section \ref{Free_Prob_Intro} to find the asymptotic eigenvalue distribution, we now analyze a channel model that is representative of a MIMO communication system with an IRS. 
\par
In general, the received signal from any line of sight path between a transmitter and receiver would contain  sufficient energy to disregard the contribution of any reflected signal (although this component may be very low rank).  Therefore, we focus on a channel model in which all signal power to the receiver is through the path of the IRS  given by

%	\begin{equation}
%	\mathbf{H}_{Total} = \mathbf{R}_{R}^{\frac{1}{2}}(\underbrace{\mathbf{H}_{2}\boldsymbol{\Phi}\mathbf{R}_{S}^{\frac{1}{2}}\mathbf{H}_{1}}_{\text{IRS}} + \underbrace{\mathbf{G}}_{\text{LOS}})\mathbf{R}_{T}^{\frac{1}{2}}.
%	\end{equation}
	\begin{equation}\label{correlated_channel}
	\mathbf{H}_{Total} = \mathbf{R}_{R}^{\frac{1}{2}}\mathbf{H}_{2}\underbrace{\phase}_{\text{IRS}}\mathbf{R}_{S}^{\frac{1}{2}}\mathbf{H}_{1}\mathbf{R}_{T}^{\frac{1}{2}}.
	\end{equation}
The random channels $\mathbf{H}_{1}$ and $\mathbf{H}_{2}$ represent the random channel gains from transmitter to IRS and from IRS to receiver, respectively. Similarly, $ \mathbf{R}_{T},  \mathbf{R}_{S} $ and  $\mathbf{R}_{S}$ represent the correlation introduced at the transmitter, IRS and receiver, respectively. 
When compared with the example seen in Section \ref{Free_Prob_Intro}, the structure of $\mathbf{H}_{Total}$ suggests that we can to use multiplicative free convolution in order to evalute the AED. With this in mind, we would like to choose a correlation model  that not only represents the system but one with a known S-Transform expression. 
One commonly used model exponential correlation. Defined by a single parameter $\rho$, the elements of the corresponding correlation matrix are given by $\rho^{i-j}$.
\par
To begin using the tools of free probability, we first separate out the free sub-components of the total covariance matrix given by 
	\begin{equation}
	\mC = \mR_{R}^{\frac{1}{2}}\mH_{2}\phase \mR_{S}^{\frac{1}{2}}\mH_{1}\mR_{T}^{\frac{1}{2}}
	\left(\mR_{R}^{\frac{1}{2}}\mH_{2}\phase \mR_{S}^{\frac{1}{2}}\mH_{1}\mR_{T}^{\frac{1}{2}}\right)^H.
	\end{equation}
Given the S-transform of these sub-components, the total S-transform is simply the resulting product.
Noting from Equation \eqref{no_csi_capacity_aed} that the capacity is only affected by the non-zero eigenvalues of the covariance matrix, we equivalently
consider the AED of the covariance matrix given by 
	\begin{align}
	\mC &  =  \mH_{1}^H \mR_{S}^{\frac{H}{2}}\phase^H \mH_{2}^H \mR_{R}^{\frac{H}{2}}
	 \mR_{R}^{\frac{1}{2}}\mH_{2}\phase \mR_{S}^{\frac{1}{2}}\mH_{1}\mR_{T}^{\frac{1}{2}}\mR_{T}^{\frac{H}{2}}
	 \\
	 &  =  
	 \mH_{1}^H \mR_{S}^{\frac{H}{2}}\phase^H \mH_{2}^H \mR_{R}\mH_{2}\phase \mR_{S}^{\frac{1}{2}}\mH_{1}\mR_{T}.
	\end{align}

This equivalence is shown by applying the Weinstein-Aronszajn identity to the characteristic polynomials of the to matrices.
In order to continue decomposing this covariance matrix into sub-components, we exploit a property of the S-Transform given by (proven in Appendix \ref{rotation_property})
\begin{equation}
S_{C_N}(z) = \frac{z+1}{z+\chi_2} S_{\tilde{C}_N}(\frac{z}{\chi_2}).
\end{equation}
If all system dimensions are $N$, these S-Trasforms are equal. For simplified analysis and notation, we will assume that all system dimensions are $N$.
Repeated iteration of the property above gives the equivalent matrix
	\begin{equation}
	\mC =
\underbrace{\phase^H \phase}_{\mI}
	\mR_{S}
	 \mH_{2}^H \mH_{2}
	 \mH_{1}\mH_{1}^H 
	  \mR_{R}
	 \mR_{T}.
	\end{equation}
Note that using the Eigen Value Decomposition of $\mR_{R}$ and $\mR_{T}$, we find the product $\mR_{R} \mR_{T}$ is also a Toeplitz matix (which can be represented by a circulant matrix in the asymptotic system size).
In \cite[Section 4.3.2]{muller2013applications} it is shown that the hermetian matrix $\mH\mH^H$ with $h_{i,j} \in \CNO$
is free (relative to what?)...
Having shown the relative freeness of the matrix sub-components, we must now find the S-transform for each in order to find the product.
\begin{enumerate}
\item 
	For both $\mH_{2}^H \mH_{2}$ and $\mH_{1}\mH_{1}^H$ with all system dimensions equal, the S-Transform as shown in \cite{muller2002random} is given by 
	\begin{equation}
	S(z) = \frac{1}{1+z}
	\end{equation}
\item 
	As shown in \cite{gray2006toeplitz}, with $N \rightarrow \infty$ Toeplitz matrices have the spectrum (and thus AED) of an equivalent circulant matrix (whose spectrum and thus AED is given by the discrete Fourier transform of the first row). Using the definition of the Stieltjes transform and the definition of the S-Transform in terms of the Stieljes transform, the S-Transform of the exponential correlation model is shown to be \cite{skupch2005free}
	\begin{equation}
	S(z) = \frac{\alpha z^2 + z \sqrt{alpha^2 z^2 -(z^2-1)}}{z-1}
	\end{equation}
\end{enumerate}
\par
Without further analysis, this expression shows the interesting result that under the assumptions of Free Probability and equal power transmission, the phases at the IRS (or any unitary matrix in the channel) cancel out of the expression for the AED. For the remaining terms, we simply need an expression for the S-Transform in order to evaluate the asymptotic system capacity.s
\par 
Using the previous analysis we can observe an extension of this result that may be relevant to variations of this system model. 
While we do not consider a line of sight component to this channel, using the tools from a similar analysis in \cite{muller2012channel}, it can be shown that 
a similar cancellation of phases will occur for the IRS channel with an LOS (or non-zero mean) component of the signal.
Note that if we are simply adding a non-zero mean component. We can find the corresponding AED by considering the channel without the non-zero mean and then adding the single eigenvalue from the non-zero mean component; in which case the phases will again cancel out. 



