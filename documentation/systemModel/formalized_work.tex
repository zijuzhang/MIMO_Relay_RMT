\documentclass[12pt,a4paper]{article}
\usepackage[utf8]{inputenc}
\usepackage{amsmath}
\usepackage{amsfonts}
\usepackage{amssymb}
\usepackage[colorlinks=true,linkcolor=blue]{hyperref}
\usepackage{graphicx,tikz}
\usepackage{float}
\usetikzlibrary{positioning}
\input defs.tex
\bibliographystyle{ieeetr}
\graphicspath{ {./figures/} }

\title{Solution Outline}
\author{Peter Hartig}

\begin{document}
\maketitle
%\tableofcontents
%\section{Solution Outline} 
\subsection{Model} 

Using the model from the pilot decontamination work we say the total channel is described by 
\begin{equation}
\mathbf{D} = \sum_{k=1}^{K}\mathbf{\mathbf{D}}_k 
\end{equation}
We now consider how to ultimately arrive at an expression for the AED of the covariance matrix $\mathbf{D}\mathbf{D}^H$.
\subsection{Approach 1} 
The expanded covariance matrix $\mathbf{D}\mathbf{D}^H$ contains both Hermetian (e.g. $\mathbf{D}_k \mathbf{D}_k ^H$) and non-hermetian terms ($\mathbf{D}_k \mathbf{D}_i^H, i \neq k$). Noting that the use of multiplicative free convolution and additive free convolution both require that the involved operators are \emph{hermetian} we could attempt to find the R-Transforms of the terms $\mathbf{D}_k \mathbf{D}_k ^H$ and $(\mathbf{D}_k \mathbf{D}_i^H+\mathbf{D}_i \mathbf{D}_k^H)$ which are both self-adjoint. 
For the terms $\mathbf{D}_k \mathbf{D}_k ^H$, we can find the S-Tranform using the methods shown in \ref{Polynomial_terms} and \cite{muller2002asymptotic}. From this S-Transform, we can move to the R-Transform in order to use additive free convolution. Finding the R-Transform of the terms $(\mathbf{D}_k \mathbf{D}_i^H+\mathbf{D}_i \mathbf{D}_k^H)$ is, however, not clear because the components of this sum are not are not self-adjoint.
Instead we consider a second approach. 


\subsection{Approach 2} 
This approach is largely based on the ideas presented in \cite{muller2012channel} and which are applied in \cite[Eqns. 54-56]{muller2014blind}. The exampled \cite{muller2012channel} uses a very similar model to the IRS work.  The idea behind this approach is based on the property of the Stieltjes transform 
	\begin{equation}\label{ST_property}
	G_{\lambda^2}(s) = \frac{G_{\lambda}(\sqrt{s})- G_{\lambda}(-\sqrt{s})}{2\sqrt{s}}.
	\end{equation} (*TO DISCUSS Can we find an equivalent to this property in the S-transform domain?)
	Note that this property also requies that the matrix $D$ is bi-unitarily invariant.
	Rather than looking for the Stieltjes transform of the eigenvalues of $\mathbf{D} \mathbf{D}^H$ we will
	first find the Stieltjes transform of the singular values of $\mathbf{D}$. 
	Given the definition of the singular values $\sigma(\mathbf{D}) = \sqrt{\lambda(\mathbf{D} \mathbf{D}^H)}$ and 
	noting that 
	for a covariance matrix (positive semi-definite) all eigenvalues are non-negative, the eigen
	values of $\mathbf{D} \mathbf{D}^H$ are the squared singular values of  $\mathbf{D}$. Therefore, if 
	we can find $G_{\mathbf{D}}(s)$, we can use Property \ref{ST_property} to find $G_{\mathbf{D} \mathbf{D}^H}(s)$.
	 This is performed using the following steps. 

\begin{enumerate}


\item 
	We first find the Stieltjes transform, $G_{\mathbf{D}_{k}\mathbf{D}_{k}^H}(s)$, for the AED of the 
	covariance matrix of individual terms, $\mathbf{D}_{k}$.
	This is discussed further in Section \ref{Polynomial_terms} and \cite[Section II]{muller2002asymptotic}.
\item
	Using Property \ref{ST_property} we find the Stieltjes transform of the symmetrized 
	singular value decomposition of $\mathbf{D}_{k}$, $\tilde{G}_{\mathbf{D}_{k}}(s)$. The symmetrized AED defined 
	in \cite[Eqn. 38]{muller2012channel} is given by 
	\begin{equation}
	\tilde{p}(x) = \frac{p(x)+P(-x)}{2}
	\end{equation}
\item
	Taking the R-transform, $\tilde{R}_{\mathbf{D}_{k}}(w)$ of $\tilde{G}_{\mathbf{D}_{k}}(s)$, we then use the free additive convolution property in order to 
	find the R-transform of the entire equivalent channel $\tilde{R}_{D}(w) = \sum_{k=1}^{K} \tilde{R}_{\mathbf{D}_{k}}(w) $ . 
	* Note that while additive free convolution holds for $\tilde{R}_{\mathbf{D}_{k}}(w)$, multiplicative free convolution does
	 not.
\item
	Using $\tilde{R}_{D}(w)$ we then find $\tilde{G}_{D}(s)$. We then use property \eqref{ST_property} in the reverse
	 order to get $\tilde{G}_{DD^H}(s)$  which allows us to evaluate the  symmetrized AED.

\item 
	Because all eigenvalues are are non-negative for the covariance matrix $\mathbf{D}\mathbf{D}^H$
	the symmetrized AED is equal to the true AED when evaluated at non-negative values and can be divided by 2
	when evaluated at 0. 

\end{enumerate}



\subsection{Solving for $G_{\mathbf{D}_{k}\mathbf{D}_{k}^H}(s)$}\label{Polynomial_terms}
Using the model from the IRS work, the individual terms of the channel are given by
\begin{equation}
\mathbf{D}_{k} = G_B \boldsymbol{\Theta}_N H_N \cdots \boldsymbol{\Theta}_1 H_1
\end{equation}
and 
\begin{equation}
C_N = \mathbf{D}_{k}\mathbf{D}_{k}^H = [G_B \boldsymbol{\Theta}_N H_N \cdots \boldsymbol{\Theta}_1 H_1]
[H_1 \boldsymbol{\Theta}_1 \cdots H_N \boldsymbol{\Theta}_N G_B]
\end{equation}
\begin{itemize}
\item 
	Because commuting matrix multiplication does not change the non-zero eigenvalue distribution, we consider 
	the matrix 
\begin{equation}
\tilde{C}_N = [\boldsymbol{\Theta}_1^H \cdots H_N^H \boldsymbol{\Theta}_N^H G_B^H][G_B \boldsymbol{\Theta}_N H_N \cdots \boldsymbol{\Theta}_1 H_1 H_1^H]
\textcolor{red}{test}
\end{equation}


\item 
	We use the rotational property of the Stieltjes transform and S-Transform. 
\begin{equation}
S_{C_N}(z) = \frac{z+1}{z+\chi_N} S_{\tilde{C}_N}(\frac{z}{\chi_N}).
\end{equation}
For the two terms of $\tilde{C}_N$, 
$ H_1 H_1^H$ and $[\boldsymbol{\Theta}_1^H \cdots H_N^H \boldsymbol{\Theta}_N^H G_B^H][G_B \boldsymbol{\Theta}_N H_N \cdots \boldsymbol{\Theta}_1]$
we can use multiplicative free convolution .

\item
	For the gaussian IID case, $ H_1 H_1^H$  has the marcenko-pastur distribution which has a well defined S-Transform. 

\item 
	Iterating the rotation property with the second term 
	\begin{equation}
	[\boldsymbol{\Theta}_1^H H_2 \cdots H_N^H \boldsymbol{\Theta}_N^H G_B^H]
	[G_B \boldsymbol{\Theta}_N H_N  \cdots H_2^H \boldsymbol{\Theta}_1]
	\end{equation}
	
	\begin{equation}
	[ \boldsymbol{\Theta}_N H_N  \cdots H_2^H \boldsymbol{\Theta}_1]
	[\boldsymbol{\Theta}_1^H H_2 \cdots H_N^H \boldsymbol{\Theta}_N^H G_B^H G_B]
	\end{equation}
	
	 it is seen that ultimately, the phase matrices will cancel out and the result is a product of well known S-Transforms.


\item 
*TO DISCUSS If there is no equivalent to the Property \ref{ST_property} in the S-Transform domain, how do we 
find a closed form of the Stieltjes transform given an arbitrary S-transform?
\end{itemize}

\item
*TO DISCUSS If the phases cancel out, then they cannot influence the final AED or the expected value. Does this mean that controlling these phases is not beneficial in the asymptotic limit?
Note these terms might affect the AED of the cross terms in the polynomial but we hypothesis that these cancel out anyways.


\bibliography{bibliography}
\end{document}
