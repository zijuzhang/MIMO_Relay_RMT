\documentclass[12pt,a4paper]{report}
\usepackage[utf8]{inputenc}
\usepackage{amsmath}
\usepackage{amsfonts}
\usepackage{amssymb}
\usepackage[colorlinks=true,linkcolor=blue]{hyperref}

\usepackage{graphicx, neuralnetwork,tikz}
\usepackage{float}
\usetikzlibrary{positioning}
\input defs.tex

\bibliographystyle{alpha}
\graphicspath{ {./figures/} }
\tikzstyle{process} = [rectangle, minimum width=3cm, minimum height=1cm, text centered, draw=black]
\tikzstyle{arrow} = [thick,->,>=stealth]

\tikzstyle{state}=[shape=circle,draw=blue!30,fill=blue!10]
\tikzstyle{observation}=[shape=rectangle,draw=orange!30,fill=orange!10]
\tikzstyle{lightedge}=[<-, dashed]
\tikzstyle{mainstate}=[state, thick]
\tikzstyle{mainedge}=[<-, thick]
\tikzstyle{block} = [draw,rectangle,thick,minimum height=2em,minimum width=2em]
\tikzstyle{sum} = [draw,circle,inner sep=0mm,minimum size=2mm]
\tikzstyle{connector} = [->,thick]
\tikzstyle{line} = [thick]
\tikzstyle{branch} = [circle,inner sep=0pt,minimum size=1mm,fill=black,draw=black]
\tikzstyle{guide} = []
\tikzstyle{snakeline} = [connector, decorate, decoration={pre length=0.2cm,
                         post length=0.2cm, snake, amplitude=.4mm,
                         segment length=2mm},thick, magenta, ->]




\title{Free Probability Analysis of Capacity in Relay Networks}
\author{Peter Hartig}

\begin{document}
\maketitle
\tableofcontents
\chapter{Introduction}
The goal of this work is to consider the IRS channel capacity and secrecy capacity using free probability theory. Specifically, this system is compared to two similar problems which have been shown to admit analysis using free probability. 
\par

\chapter{Problem Background}

\section{Information Theory Tools}

\begin{enumerate}
\item For a random MIMO communication channel and CSI known at the transmitter and Gaussian distributed transmit signal $\mathbf{x}$, the capacity ($\mathcal{C}$) of a point-to-point channel $\mathbf{H}$ is given by
\begin{equation}\label{capacity}
\Capacity{(H)} = \Expect\left[\underset{tr(\mathbf{Q}_{\mathbf{xx}}) < P_{Total}}{\mathrm{max}} \;\Log\left(|\mathbf{I}_{N_R}+\frac{1}{\sigma_n}\mathbf{H}\mathbf{Q}_{\mathbf{xx}}\mathbf{H}^H|\right)\right].
\end{equation}
*Question: Without an analytic solution to the waterfilling problem used to choose $\mathbf{Q}_{\mathbf{xx}}$, how can we take an expectation? -> need to do a Monte Carlo simulation (in MIMO book) so probably will not be able to include the beamformer in any free probability solution.

\item
In the case in which no CSI is known at the transmitter and the power at the transmitter, $P_{\text{Total}}$, is distributed equally across transmit antennas Equation \ref{capacity} becomes
\begin{equation}
\Capacity{(H)} = \Expect\left[\Log\left(|\mathbf{I}_{N_R}+\frac{P_{\text{Total}}}{N_T \sigma_n}\mathbf{H}\mathbf{H}^H|\right)\right].
\end{equation} 
Assuming asymptotically large antenna arrays, and given the distribution for the eigenvalues of $\mathbf{H}\mathbf{H}^H$ which we denote by $p_{\lambda}(\lambda)$, Equation \ref{capacity} can be rewritten as 
\begin{equation}
\Capacity{(H)} = N_r \Expect\left[\Log\left(1+\frac{P^{\text{Total}}}{N_T \sigma_n}\lambda\right)\right].
\end{equation}
 in which the expectation is taken over $p_{\lambda}(\lambda)$.

\item The secrecy capacity of the general wire-tapped communication channel with perfect knowledge of the channels to both the legitimate receiver, $\mathbf{H} \in \mathbb{C}_{N_r \times N_t}$, and eavesdropper, $\hat{\mathbf{H}}\in \mathbb{C}_{N_e \times N_t}$, is given by 

\begin{gather}\label{secrecy_capacity}
\Capacity{(H)} = \Expect\left[\underset{tr{\mathbf{Q}_{xx}} < P_{Total}}{\mathrm{max}} \;\left[\Log\left(|\mathbf{I}_{N_R}+\frac{1}{\sigma_n}\mathbf{H}\mathbf{Q}_{xx}\mathbf{H}^H|\right) -
\Log\left(|\mathbf{I}_{N_R}+\frac{1}{\sigma_n}\hat{\mathbf{H}}\mathbf{Q}_{xx}\hat{\mathbf{H}}^H|\right)
\right] \right].
\end{gather}
Again assuming $P_{\text{Total}}$ is distributed equally across transmit antennas such that $\mathbf{Q}_{xx} = \frac{P_{\text{Total}}}{N_t}\mathbf{I}_{N_t}$, and that both the legitimate user and eavesdropper have sufficient antennas such that $N_r > N_t$ and $N_e >N_t$,  this can be written as 

\begin{gather}\label{secrecy_capacity}
\Capacity{(H)} = \Expect\left[N_r \Log\left(1+\frac{P^{\text{Total}}}{N_T \sigma_n}\lambda \right) -
N_e \Log\left(1+\frac{P^{\text{Total}}}{N_T \sigma_n}\hat{\lambda} \right) \right]
\end{gather}
in which $\lambda$ and $\hat{\lambda}$ denote the eigenvalues of $\mathbf{H}\mathbf{H}^H$ and $\hat{\mathbf{H}}\hat{\mathbf{H}}^H$ respectively. Generally, finding capacity will require knowledge of the joint asymptotic eigenvalue distribution $p_{\lambda}(\lambda,\hat{\lambda})$.

\end{enumerate}
\section{Free Probability/Random Matrix Theory Tools}

Here the current tools available from free probability analysis are reviewed in order to clarify the types of polynomials (composite channels) which can be analyzed.
\begin{enumerate}
\item 
	Transformations of the AED
	Detail how to recover the true pdf given the Steiltjes, R/S transforms
	\begin{itemize}
	\item
		For general random matrices, we can still find the eigenvalue distributions using the problem of moments
	\end{itemize}

\item 
	Relevant matrices with known AEDs.
	\begin{itemize}
	\item General zero-mean iid $n\timesn$ with variance 1/$N$ -> Full Circle
	\item Sum of the matrix above with its adjoint. The result is self-adjoint so the eigenvalues are all real.
	\item 
	\end{itemize}
\item
	Free Families
	\begin{itemize}
	\item blank
	\end{itemize}


\item
	Matrix Composition: Detail requirements on the variables
	\begin{itemize}
	\item
	Additive free convolution using the R-Transform for two free variables with known R-Transforms.
	\item
	Multiplicitate free convolution using the S-Transform for two free variables with known S-Transforms.
	
	\end{itemize}

\end{enumerate}


\chapter{Comparison to Previous Work}
First, the two systems to which the IRS channel is compared are first given and relevant details are highlighted. 
\section{Notes from Basic Relay Work}
\subsection{System Model}

\begin{itemize}
\item Assuming decode and forwarding at relays and a matched filter precoding, the resulting received signal is given by 
\begin{equation}
\mathbf{y} = \mathbf{H}_{Total} \mathbf{s} +  \mathbf{n} = \sum_{r=1}^{R_L} \mathbf{H}_{r}\mathbf{H}_{r}^{\dagger}\mathbf{s} +  \mathbf{n}
\end{equation}
for which the composite channel is given by 
\begin{equation}
\mathbf{H}_{Total} = \mathbf{H}_{1}\mathbf{H}_{1}^{H} +\cdots +\mathbf{H}_{R_L}\mathbf{H}_{R_L}^{H}.
\end{equation}
Assuming the covariance matrix $\mathbf{Q}_s = \mathbf{I}_{N_R}$, the ergodic capacity is given by 
\begin{equation}\label{ergodic_capacity}
E\left[\text{log}_2(\mathbf{I}_{N_R} + \frac{P}{\sigma_n}\mathbf{H}_{Total}\mathbf{H}_{Total}^H)\right].
\end{equation}
We consider $\mathbf{H}_{Total}\mathbf{H}_{Total}^H$ as a polynomial in the self-adjoint variables $\mathbf{H}_{i}\mathbf{H}_{i}^{H}$.
\end{itemize}
\subsection{Free Probability Analysis}



\section{Notes Pilot Decontamination Work}
\subsection{System Model}
Here we consider the signal 
\begin{equation}
\mY = \mH\mX + \mH_I \mX_I + \mW
\end{equation}
with covariance matrix 
\begin{gather*}
\mQ_{YY} = (\mH\mX + \mH_I \mX_I + \mW)(\mH\mX + \mH_I \mX_I + \mW)^H = 
\\
\mH\mX\mX^H\mH^H + \mH\mX\mX_I^H\mH_I^H + \mH\mX\mW^H +\\ \mH_I \mX_I \mX^H\mH^H + \mH_I \mX_I\mX_I^H\mH_I^H + \mH_I \mX_I\mW^H + \mW\mX^H\mH^H + \mW\mX_I^H\mH_I^H + \mW\mH_I^H.
\end{gather*}
Clearly, the individual terms of this polynomial are not self-adjoint.
\subsection{Free Probability Analysis}
Need to figure out how the actual steiltjes transform of this polynomial is found.
- There is a fixed point equation for finding this transform.

- Ultimately we want to have a form of the Steiltjes transform for the entire channel AED using a fixed point equation.
\chapter{IRS Work}

\section{System Model}\label{system_model}

\begin{enumerate}
\item 
	We assume non-frequency selective channels using techniques like OFDM. (Is this always going to be reasonable?)
	
%\item 
%	We assume a transmitter without channel state information and thus power is distributed equally over all antennas such that
%	\begin{equation}
%	\mathbf{Q_{\mathbf{x}}} = \frac{P^{\text{Total}}}{N_T}
%	\end{equation}
%	in which $P^{\text{Total}}$ is the power available at the transmitter, and $N_T$ is the number of antennas at the transmitter. 
\item 
	We assume a transmitter with channel state information and beamforming capability such that the transmitted signal is given by 
	\begin{equation}
	\mathbf{x} = \mathbf{F}\mathbf{s}^0
	\end{equation}
	in  which $\mathbf{U}$ is a beamforming matrix, $\mathbf{s}^0$ is the normalized vector of transmitted symbols such that $\mathbf{Q_{\mathbf{s}^0,\mathbf{s}^0}} = \mathbf{I}_{N_R}$. 
	
\item Intelligent Reflective Surfaces (IRS) are placed between transmitter and receiver. The IRS is made of individual elements each of which can reflect the received signal with a unique, chosen phase shift.

%\item Individual users transmit simultaneously to the first set of IRS's. This transmit signal hops through an arbitrary number of IRS's. The final set of IRS's reflect the signal to a set of independent users. Possible? (For half duplex need to make sure number of relay sets makes sense)

\item 
The signal $\mathbf{x}$ is transmitted through $N$ sets of IRS's. The set $n, \forall n = [1 \cdots N]$, has $L^n$ reflecting surfaces. Each reflecting surface in set $n$ is indexed by $R^n_{i}, \forall i = [1\cdots L^n]$ and has $K_{i}^{n}$ elements.
\item 
 The  signal received at element $k$ of IRS $R_i^n$ is 
\begin{equation}
s^{n}_{i,k} = \sum_{j = 1}^{L^{n-1}} (\mathbf{h}^{n-1}_{j,i,k})^T \boldsymbol{\Theta}^{n-1}_{j}\mathbf{s}^{n-1}_{j}
\end{equation}
with $\boldsymbol{\Theta}^{n-1}_{j}$ as the phases shifts applied by $R_j^{n-1}$ and $\mathbf{h}^{n-1}_{j,i,k}$ as the vector of channel coefficients between $R_j^{n-1}$ and element $k$ of $R_i^n$.

\item
The \emph{vector} of received signals for all elements at $R^n_{i}$, is given by
\begin{equation}\label{received_vector}
\mathbf{s}^{n}_{i} = \sum_{j = 1}^{L^{n-1}} \mathbf{H}^{n-1}_{j,i}\boldsymbol{\Theta}^{n-1}_{j}\mathbf{s}^{n-1}_{j}.
\end{equation}
in which $\mathbf{H}^{n-1}_{j,i} = [\mathbf{h}^{n-1}_{j,i,1} \cdots \mathbf{h}^{n-1}_{j,i,K_{j}^{n-1}}]^T$.



\item For a specific receiving user with potentially multiple receiver antennas, the received signal is given in a recursive form by
\begin{equation}\label{general_received}
\mathbf{s}^{n}_{i} = \sum_{j = 1}^{L^{n-1}} \mathbf{H}^{n-1}_{j,i}\boldsymbol{\Theta}^{n-1}_{j} \left[
\sum_{k = 1}^{L^{n-2}} \mathbf{H}^{n-2}_{k,i}\boldsymbol{\Theta}^{n-2}_{k}\mathbf{s}^{n-2}_{k} \right].
\end{equation}

  
\item
	Note that this model assumes sufficient spacing between set $n$ and set $n+2$ such that we consider the signal reflected in set $n$ only at set $n+1$ and not for any $n<k$.


\item 
	Just as in the model from Relay by Luci, we write the composite channel as 

\end{enumerate}
To better analyze types of components within the polynomial characterizing the resulting, composite channel as in ..., we now consider two specific cases of this general system model.  

\subsection{Case in which $N=1$}
First we consider the case in which there is only one set (one hop) of IRSs between the transmitter and receiver. 
Equation \ref{general_received} now becomes 
\begin{equation}\label{N=1_received}
\mathbf{s}^{1} = \sum_{j = 1}^{L^{1}} \mathbf{H}^{1}_{j}\boldsymbol{\Theta}^{1}_{j}
\mathbf{H}^{0}_{j} \mathbf{U}\mathbf{s}
\end{equation}
in which $\mathbf{s}$ is the original vector of transmitted symbols. In this case, the composite channel is given by 
\begin{equation}
\mathbf{H}_{Total} = \sum_{j = 1}^{L^{1}} \mathbf{H}^{1}_{j}\boldsymbol{\Theta}^{1}_{j}
\mathbf{H}^{0}_{j}\mathbf{U}.
\end{equation}
The individual terms of the polynomial in \eqref{ergodic_capacity} are not self-adjoint unless $L^{1} = 1$ and the precoding matrix is chosen such that 
\begin{equation}
\mathbf{U} = [\mathbf{H}^{1}_{1}\boldsymbol{\Theta}^{1}_{1}\mathbf{H}^{0}_{1}]^H.
\end{equation}


\subsection{Case in which $L_n=1$}
We now consider the case with only a single IRS in each hop between the transmitter and receiver. 
Equation \ref{general_received} now becomes 
\begin{equation}\label{L_n=1_received}
\mathbf{s}^{n}_{i} = \sum_{j = 1}^{L^{n-1}} \mathbf{H}^{n-1}_{j,i}\boldsymbol{\Theta}^{n-1}_{j} \left[
\sum_{k = 1}^{L^{n-2}} \mathbf{H}^{n-2}_{k,i}\boldsymbol{\Theta}^{n-2}_{k}\mathbf{s}^{n-2}_{k} \right].
\end{equation}
In this case, the composite channel is given by 
\begin{equation}
\mathbf{H}_{Total} = \mathbf{H}^{N}_{1}\boldsymbol{\Theta}^{N}_{1}\mathbf{H}^{N-1}_{1}\boldsymbol{\Theta}^{N-1}_{1} \cdots  
\mathbf{H}^{0}_{j}\mathbf{U}.
\end{equation}
Similar to the case \ref{N=1_received}, the variable of the composite channel polynomial are only self-adjoint if 

** But can we actually do something with a term with so many different variables?

\section{System Model With Secrecy Capacity}
The model from Section \ref{system_model} can be easily extended to the case in which an eavesdropper may be located at any point along the transmission path. We denote the received signal at the eavesdropper by

\begin{equation}\label{general_received}
\hat{\mathbf{s}}^{n}_{i} = \sum_{j = 1}^{L^{n-1}} \hat{\mathbf{H}}^{n-1}_{j,i}\boldsymbol{\Theta}^{n-1}_{j} \mathbf{s}^{n-1}_{j}.
\end{equation}
Note that the term $\mathbf{s}^{n-1}_{j}$ is unchanged from \ref{received_vector} because up until the final hop, the path for the eavesdropper and legitimate receiver are the same. 

* We should only need to known the difference between the two composite channels to find the secrecy capacity. Even if we don't know the individual ones?
\section{Free Probability Analysis}

\chapter{My TODO}


\bibliography{bibliography}
\end{document}
