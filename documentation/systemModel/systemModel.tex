\documentclass[12pt,a4paper]{report}
\usepackage[utf8]{inputenc}
\usepackage{amsmath}
\usepackage{amsfonts}
\usepackage{amssymb}
\usepackage[colorlinks=true,linkcolor=blue]{hyperref}
\usepackage{graphicx,tikz}
\usepackage{float}
\usetikzlibrary{positioning}
\input defs.tex
\bibliographystyle{ieeetr}
\graphicspath{ {./figures/} }

\title{Free Probability Analysis of Capacity in Relay Networks}
\author{Peter Hartig}

\begin{document}
\maketitle
\tableofcontents
\chapter{Introduction}
The goal of this work is to consider the intelligent reflective surface (IRS) channel capacity and secrecy capacity using the tools of free probability. First, two similar problems which have been shown to admit analysis using free probability are considered. These examples are then compared to the IRS channel. 
\par

\chapter{Problem Background}

\section{Information Theory Tools}

\begin{enumerate}
\item For a random MIMO communication channel with Gaussian distributed transmit signal $\mathbf{x}$, and CSI known at the transmitter,  the capacity ($\mathcal{C}$) of a point-to-point channel characterized by the flat-fading channel matrix $\mathbf{H}$ is given by
\begin{equation}\label{capacity}
\Capacity{(\mathbf{H})} = \Expect\left[\underset{tr(\mathbf{Q}_{\mathbf{xx}}) < P_{Total}}{\mathrm{max}} \;\Log\left(|\mathbf{I}_{N_R}+\frac{1}{\sigma_n}\mathbf{H}\mathbf{Q}_{\mathbf{xx}}\mathbf{H}^H|\right)\right].
\end{equation}
Solving for the optimial $\mathbf{Q}_{\mathbf{xx}}$ generally lacks an analytic solution and requires solving the  "waterfilling" problem to choose $\mathbf{Q}_{\mathbf{xx}}$. As a result, taking an expectation of this general expressiong requires Monte Carlo simulation (as seen in MIMO course book).

\item
In the case in which CSI is \textbf{not} known at the transmitter, the power at the transmitter, $P_{\text{Total}}$, is distributed equally across transmit antennas and Equation \ref{capacity} becomes
\begin{equation}\label{capacity_no_csi}
\Capacity{(\mathbf{H})} = \Expect\left[\Log\left(|\mathbf{I}_{N_R}+\frac{P_{\text{Total}}}{N_T \sigma_n}\mathbf{H}\mathbf{H}^H|\right)\right].
\end{equation} 
Assuming asymptotically large antenna arrays, and given the distribution for the eigenvalues of $\mathbf{H}\mathbf{H}^H$ which we denote by $p_{\lambda}(\lambda)$, Equation \ref{capacity_no_csi} can be rewritten as 
\begin{equation}\label{capacity_AED}
\Capacity{(\mathbf{H})} = N_r \Expect\left[\Log\left(1+\frac{P^{\text{Total}}}{N_T \sigma_n}\lambda\right)\right].
\end{equation}
 in which the expectation is taken over $p_{\lambda}(\lambda)$ known as the asymptotic eigenvalue distribution (AED).

\item The secrecy capacity of the general wire-tapped communication channel with perfect knowledge of the channels to both the legitimate receiver, $\mathbf{H} \in \mathbb{C}_{N_r \times N_t}$, and an eavesdropper, $\hat{\mathbf{H}}\in \mathbb{C}_{N_e \times N_t}$, is given by 

\begin{gather}\label{secrecy_capacity}
\Capacity{(\mathbf{H})} = \Expect\left[\underset{tr({\mathbf{Q}_{xx}}) < P_{Total}}{\mathrm{max}} \;\left[\Log\left(|\mathbf{I}_{N_R}+\frac{1}{\sigma_n}\mathbf{H}\mathbf{Q}_{xx}\mathbf{H}^H|\right) -
\Log\left(|\mathbf{I}_{N_R}+\frac{1}{\sigma_n}\hat{\mathbf{H}}\mathbf{Q}_{xx}\hat{\mathbf{H}}^H|\right)
\right] \right].
\end{gather}
Again assuming $P_{\text{Total}}$ is distributed equally across transmit antennas such that $\mathbf{Q}_{xx} = \frac{P_{\text{Total}}}{N_t}\mathbf{I}_{N_t}$, and that both the legitimate user and eavesdropper have sufficient antennas such that $N_r > N_t$ and $N_e >N_t$,  this can be written as 

\begin{gather}\label{secrecy_capacity}
\Capacity{(\mathbf{H})} = \Expect\left[N_r \Log\left(1+\frac{P^{\text{Total}}}{N_T \sigma_n}\lambda \right) -
N_e \Log\left(1+\frac{P^{\text{Total}}}{N_T \sigma_n}\hat{\lambda} \right) \right]
\end{gather}
in which $\lambda$ and $\hat{\lambda}$ denote the eigenvalues of $\mathbf{H}\mathbf{H}^H$ and $\hat{\mathbf{H}}\hat{\mathbf{H}}^H$ respectively. Generally, finding capacity will require knowledge of the \emph{joint} AED $p_{\lambda}(\lambda,\hat{\lambda})$.

\end{enumerate}
\section{Random Matrix Theory and Free Probability Tools}

\subsubsection{Matrix AEDs}
For some matrices, we know the AED directly. For more general classes of matrices we find the AED using the known moments of the matrix to find the Stieltjes transform and then inverting the Stieltjes transform to find the true distribution. Generally, given the distributions of two matrices, the distribution for compositions of these matrices is difficult to find. 

\subsubsection{Free Probability}
By considering asymptotically large matrices as free random variables, the distribution resulting from the composition of free random variables can be found using the R-transform and S-transform to linearize addition and multiplication respectively.

\chapter{Previous Work}
First, two systems related to the IRS channel capacity problem are described and relevant details are highlighted. 
\section{Notes from Decode and Forward Relay Work in \cite{hadley2019capacity}}\label{relay}
\subsection{System Model}

\begin{itemize}
\item A decode and forward protocol is used at $R_L$ different relays between a transmitter and receiver. A matched filter precoding is applied at each transmitting relay such that the signal at the receiver is given by 
\begin{equation}
\mathbf{y} = \mathbf{H}_{Total} \mathbf{s} +  \mathbf{n} = \sum_{r=1}^{R_L} \mathbf{H}_{r}\mathbf{H}_{r}^{\dagger}\mathbf{s} +  \mathbf{n}
\end{equation}
for which the composite channel is given by 
\begin{equation}
\mathbf{H}_{Total} = \mathbf{H}_{1}\mathbf{H}_{1}^{H} +\cdots +\mathbf{H}_{R_L}\mathbf{H}_{R_L}^{H}.
\end{equation}
Assuming the covariance matrix $\mathbf{Q}_s = \mathbf{I}_{N_R}$, the ergodic capacity is given by 
\begin{equation}\label{ergodic_capacity}
E\left[\text{log}_2(\mathbf{I}_{N_R} + \frac{P}{\sigma_n}\mathbf{H}_{Total}\mathbf{H}_{Total}^H)\right].
\end{equation}
We consider $\mathbf{H}_{Total}\mathbf{H}_{Total}^H$ as a polynomial in the self-adjoint variables $\mathbf{H}_{i}\mathbf{H}_{i}^{H}$.
\end{itemize}
\subsection{Free Probability Analysis}
It is claimed in the final paragraph of the "theory" section of \cite{hadley2019capacity} that the R-transform and S-transform cannot be used to analyze more complex polynomials. This seems to contradict the methods used in the decontamination paper (TO DISCUSS). 


\section{Notes Pilot Decontamination Work}
\subsection{System Model}
Disregarding the motivation and context of this problem, we want to find the AED of the signal 
\begin{equation}
\mY = \mH\mX + \mH_I \mX_I + \mW
\end{equation}
with covariance matrix 
\begin{gather*}
\mQ_{YY} = (\mH\mX + \mH_I \mX_I + \mW)(\mH\mX + \mH_I \mX_I + \mW)^H = 
\\
\mH\mX\mX^H\mH^H + \mH\mX\mX_I^H\mH_I^H + \mH\mX\mW^H +\\ \mH_I \mX_I \mX^H\mH^H + \mH_I \mX_I\mX_I^H\mH_I^H + \mH_I \mX_I\mW^H + \mW\mX^H\mH^H + \mW\mX_I^H\mH_I^H + \mW\mH_I^H.
\end{gather*}
Clearly, the individual terms of this polynomial are not self-adjoint.
\subsection{Free Probability Analysis}\label{decont_fpt}
In order to analyzed the matrix polynomial above, the Stieltjes of the matrix 

\begin{equation}
\mathbf{D} = \sum_{k=1}^{K} \alpha_k \mathbf{B}_k \mathbf{C}_k
\end{equation}
with $\alpha_k \in \mathbb{R}$ is considered. Given the S-transform of $\mathbf{B}_k $ and $\mathbf{C}_k$ (\cite{nica2006lectures}) we can first use the product property of the S-transform to find the Stieltjes transform of each term in the sum $\mathbf{B}_k \mathbf{C}_k$. Given the Stieltjes transform of each term, we then find the R-transform of each term $\mathbf{B}_k \mathbf{C}_k$ and exploiting the additive property of the R-transform, we find the R-transform of $\mathbf{D}$. Inverting the R-transform to get the Stieltjes transform of $\mathbf{D}$ which we can evaluate to find the AED.
\par
Note that the resulting Stieltjes transform is given as a fixed point equation which is then evaluated for each value of of interest to find the corresponding value of the AED. 


\begin{itemize}
\item The IRS model needs to be checked for compatibility with \cite{nica2006lectures}.
\item Need to solve for a similar fixed point equation.
\item Evaluate the capacity by using the above distribution to perform an integration.
\end{itemize} 
\chapter{IRS Work}

\section{System Model}\label{system_model}

\begin{enumerate}
\item 
	We assume non-frequency selective channels using techniques like OFDM. (Is this always going to be reasonable? TO DISCUSS)
	
%\item 
%	We assume a transmitter without channel state information and thus power is distributed equally over all antennas such that
%	\begin{equation}
%	\mathbf{Q_{\mathbf{x}}} = \frac{P^{\text{Total}}}{N_T}
%	\end{equation}
%	in which $P^{\text{Total}}$ is the power available at the transmitter, and $N_T$ is the number of antennas at the transmitter. 
\item 
	We assume a transmitter with channel state information and beamforming capability such that the transmitted signal is given by 
	\begin{equation}
	\mathbf{x} = \mathbf{F}\mathbf{s}^0
	\end{equation}
	in  which $\mathbf{F}$ is a beamforming matrix and $\mathbf{s}^0$ is the normalized vector of transmitted symbols such that $\mathbf{Q_{\mathbf{s}^0,\mathbf{s}^0}} = \mathbf{I}_{N_t}$. 
	
\item Intelligent Reflective Surfaces (IRS) are placed between transmitter and receiver. The IRS is made of individual elements each of which can reflect the received signal with a unique, chosen phase shift.

%\item Individual users transmit simultaneously to the first set of IRS's. This transmit signal hops through an arbitrary number of IRS's. The final set of IRS's reflect the signal to a set of independent users. Possible? (For half duplex need to make sure number of relay sets makes sense)

\item 
The signal $\mathbf{x}$ is transmitted through $N$ sets of IRS's. The set $n, \forall n = [1 \cdots N]$, has $L^n$ reflecting surfaces. Each reflecting surface in set $n$ is indexed by $R^n_{i}, \forall i = [1\cdots L^n]$ and has $K_{i}^{n}$ reflecting elements.
\item
The set of $L_n$ individual IRS in set $n$ may be spaced differently and thus the attenuation coefficient for each IRS is given by $\alpha_i^n$.
\item 
 The  signal received at element $k$ of IRS $R_i^n$ is 
\begin{equation}
s^{n}_{i,k} = \sum_{j = 1}^{L^{n-1}} \alpha_i^n (\mathbf{h}^{n}_{j,i,k})^T \boldsymbol{\Theta}^{n-1}_{j}\mathbf{s}^{n-1}_{j}
\end{equation}
with $\boldsymbol{\Theta}^{n-1}_{j}$ as the phase shifts applied by IRS $R_j^{n-1}$ and $\mathbf{h}^{n-1}_{j,i,k}$ as the vector of channel coefficients between $R_j^{n-1}$ and element $k$ of $R_i^n$.

\item
	To reduce notation in the following, the attenuation component,$\alpha_i^n$ is included directly in the channel such that $ \mathbf{h}^{n}_{j,i,k} = \alpha_i^n \mathbf{h}^{n}_{j,i,k}$
\item
The \emph{vector} of received signals for all elements at $R^n_{i}$, is given by
\begin{equation}\label{received_vector}
\mathbf{s}^{n}_{i} = \sum_{j = 1}^{L^{n-1}} \mathbf{H}^{n}_{j,i}\boldsymbol{\Theta}^{n-1}_{j}\mathbf{s}^{n-1}_{j}.
\end{equation}
in which $\mathbf{H}^{n}_{j,i} = [\mathbf{h}^{n}_{j,i,1} \cdots \mathbf{h}^{n}_{j,i,K_{j}^{n}}]^T$.



\item For IRS $R^n_{i}$ the received signal vector is given in a recursive form as
\begin{equation}\label{general_received}
\mathbf{s}^{n}_{i} = \sum_{j = 1}^{L^{n-1}}  \mathbf{H}^{n}_{j,i}\boldsymbol{\Theta}^{n-1}_{j} \left[
\sum_{k = 1}^{L^{n-2}}  \mathbf{H}^{n-1}_{k,i}\boldsymbol{\Theta}^{n-2}_{k}\mathbf{s}^{n-2}_{k} \right].
\end{equation}

\item 
The received signal vector for the first set of IRSs, $R^n_{i}$ $\forall i = [1 \cdots L^{1}]$ is given by 
\begin{equation}
\mathbf{s}^{1}_{i} = \mathbf{H}^{1}_{i}\mathbf{F}\mathbf{s}^0
\end{equation}
\item 
	Finally, using $\mathbf{G}^b$ and $\mathbf{G}^e$ as the channel matrices between the last set of IRSs and a legitimate users and an eavesdropper respectively, the received signals for the legitimate user is given by 
	\begin{equation}\label{legit_received}
\mathbf{r}_{b} = \sum_{j = 1}^{L^{N}}\mathbf{G}^{b}\boldsymbol{\Theta}^{N}_{j}\mathbf{s}^{N}_{j}
\end{equation}
	and the received signal for the eavesdropper is given by 
	\begin{equation}\label{tap_received}
\mathbf{r}_{e} = \sum_{j = 1}^{L^{N}} \mathbf{G}^{e}\boldsymbol{\Theta}^{N}_{j}\mathbf{s}^{N}_{j}
\end{equation}
\item
	Note that this model assumes sufficient spacing between set $n$ and set $n+2$ such that we consider the signal reflected in set $n$ only at set $n+1$ and not for any $n<k$.

\end{enumerate}
To better analyze types of components within the polynomial characterizing the resulting composite channel, we consider only the legitimate receiver channel for specific cases of this general system model. Before these examples, we first consider the optimization of the phase-shifting array at each IRS.

\subsection{Optimization at IRS}
We assume the transmitter uses equal power distribution for the reasons shown in \ref{capacity}. With full CSI at each IRS (detail what this means), one choice for optimization of the phase-shift array $\boldsymbol{\Theta}$ is given by 

    \begin{equation}
    \begin{array}{ll}
    \underset{\boldsymbol{\Theta}}{\text{maximize }}   & \Expect\left[\Log\left(|\mathbf{I}_{N_R}+\frac{P_{\text{Total}}}{N_T \sigma_n}\mathbf{H}_{Total}\mathbf{H}_{Total}^H|\right)\right]
    \\
    \mbox{subject to } & tr\left(
    \left(\boldsymbol{\Theta}\boldsymbol{\Theta}^H - \mathbf{I}  \right)
    \left(\boldsymbol{\Theta}\boldsymbol{\Theta}^H - \mathbf{I}  \right)^H
    \right) = 0
    \end{array}
    \label{general_irs_opt}
    \end{equation}

in which $\boldsymbol{\Theta}$ represents the choice of all $\boldsymbol{\Theta}^n_i$ and $\mathbf{H}_{Total}$ is the composite channel from transmitter to receiver. The constraint enforces that $\boldsymbol{\Theta}^n_i$ is diagonal and that each element on the diagonal has unit modulus. 
Noting that elements the matrix  $\mathbf{H}_{Total}\mathbf{H}_{Total}^H$ will generally have terms that are both convex and concave w.r.t. the diagonal elements of $\boldsymbol{\Theta}^n_i$, this optimization problem is generally non-convex. A similar version of this problem is presented and analyzed in \cite{wu2019intelligent}.
\par
We now considering this problem in the case by case basis in the following examples. 


\subsection{Case in which $L_n=1$ and $N=1$}
We now consider the case with only a single IRS in each hop between the transmitter and receiver. 
Equation \ref{legit_received} now becomes 
	\begin{equation}
\mathbf{r}_{b} =  \mathbf{G}^{b}\boldsymbol{\Theta}^{1}\mathbf{H}^{1}\mathbf{F}\mathbf{s}^0
\end{equation}
Assuming no CSI at the transmitter, the composite channel covariance matrix is given by 
\begin{equation}
\mathbf{Q}_{bb} = [\mathbf{G}^{b}\boldsymbol{\Theta}^{1}\mathbf{H}^{1}][\mathbf{G}^{b}\boldsymbol{\Theta}^{1*}\mathbf{H}^{1}]^H
\end{equation}
This is similar to the example given in \cite[Section 4.10]{muller2013applications}. As we are interested in the distribution of the non-zero eigenvalues of this matrix, we can instead consider the distribution of 
\begin{equation}
\tilde{\mathbf{Q}}_{bb} = [\mathbf{G}^{b}\boldsymbol{\Theta}^{1} \boldsymbol{\Theta}^{1*}\mathbf{G}^{bH}]\mathbf{H}^{1}\mathbf{H}^{1H}
\end{equation}

Noting that $\boldsymbol{\Theta}^{1}\boldsymbol{\Theta}^{1*} = \mathbf{I}$ this becomes

\begin{equation}
\tilde{\mathbf{Q}}_{bb} = [\mathbf{G}^{b}\mathbf{G}^{bH}]\mathbf{H}^{1}\mathbf{H}^{1H}
\end{equation}

In order to use the same methods for finding the AED as \cite[Section 4.10]{muller2013applications}, it must first be shown that $\left(\left( \mathbf{G}^{b}\mathbf{G}^{b^\dagger }\right)
,\left( \mathbf{H}^{1}\mathbf{H}^{1^\dagger} \right)
\right)$
form a free family. 
\par
TODO need to investigate how the final AED is found in this work.

\subsection{Case in which $L_n=1$}
We now increase the generalization to a system with $N$ IRS hops but still a \emph{single} IRS in each hop between the transmitter and receiver. 
Equation \ref{general_received} now becomes 
\begin{equation}\label{}
\mathbf{r}_{b} =  \mathbf{G}^{b}\boldsymbol{\Theta}^{N}\mathbf{s}^{N}
\end{equation}
in which $\mathbf{s}^{N}$ is defined in Equation \ref{general_received}.

In this case, the composite channel is given by 
\begin{equation}
\mathbf{H}_{Total} = \mathbf{G}^{b}\boldsymbol{\Theta}^{N}\mathbf{H}^{N} \cdots \boldsymbol{\Theta}^{1}\mathbf{H}^{1}
\end{equation}
with covariance matrix 
\begin{equation}
\mathbf{Q}_{bb} = \mathbf{H}_{Total}\mathbf{H}_{Total}^H = [\mathbf{G}^{b}\boldsymbol{\Theta}^{N}\mathbf{H}^{N} \cdots \boldsymbol{\Theta}^{1}\mathbf{H}^{1}][\mathbf{G}^{b}\boldsymbol{\Theta}^{N}\mathbf{H}^{N} \cdots \boldsymbol{\Theta}^{1}\mathbf{H}^{1}]^H
\end{equation}
Using the rotational property of the trace as in the previous case
We can equivalently look for the AED of the matrix 
\begin{equation}
\tilde{\mathbf{Q}}_{bb} = [\mathbf{G}^{b}\boldsymbol{\Theta}^{N}\mathbf{H}^{N} \cdots \boldsymbol{\Theta}^{1}][\mathbf{G}^{b}\boldsymbol{\Theta}^{N}\mathbf{H}^{N} \cdots \boldsymbol{\Theta}^{1}]\mathbf{H}^{1}\mathbf{H}^{1H}].
\end{equation}
Iterating this procedure and noting that $\boldsymbol{\Theta}^{n}\boldsymbol{\Theta}^{n*} = \mathbf{I}$ $\forall n \in [1\cdots N]$, this is exactly the case from \cite[Section 4.10]{muller2013applications}.

\subsection{Case in which $N=1$}
We now consider the case in which there is one IRS hop between the transmitter and receiver (i.e $N=1$) but with an  arbitrarily number of surfaces ($L_n$). 
Equation \ref{legit_received} now becomes 
\begin{equation}
\mathbf{r}_{b} = \sum_{j = 1}^{L^{1}} \mathbf{G}^{b}_{j}\boldsymbol{\Theta}^{1}_{j}\mathbf{H}^{1}_{j}\mathbf{F}\mathbf{s}^0
\end{equation}
with composite channel
\begin{equation}
\mathbf{H}_{Total} = \sum_{j = 1}^{L^{1}} \mathbf{G}^{b}_{j}\boldsymbol{\Theta}^{N}_{j}\mathbf{H}^{1}_{j}\mathbf{F}
\end{equation}
and covariance matrix

\begin{equation}
\mathbf{Q}_{bb} = [\mathbf{G}^{b}_{1}\boldsymbol{\Theta}^{N}_{1}\mathbf{H}^{1}_{1}\mathbf{F} + 
\cdots + \mathbf{G}^{b}_{N}\boldsymbol{\Theta}^{N}_{L^{1}}\mathbf{H}^{1}_{L^{1}}\mathbf{F}]
[\mathbf{G}^{b}_{1}\boldsymbol{\Theta}^{N}_{1}\mathbf{H}^{1}_{1}\mathbf{F} + 
\cdots + \mathbf{G}^{b}_{N}\boldsymbol{\Theta}^{N}_{L^{1}}\mathbf{H}^{1}_{L^{1}}\mathbf{F}]^H
\end{equation}

This sum looks similar to the polynomial considered in Section \ref{decont_fpt}, however each term of the polynomial contains more than two variables. If we can iteratively apply the S-transform rule then perhaps we can deal with this in the same way (TO DISCUSS). 
\par
Unlike the model with amplify and forward relays discussed in section \ref{relay} and \cite{hadley2019capacity}, the individual variables in each term of the 
polynomial, $\mathbf{G}^{b}_{1}\boldsymbol{\Theta}^{N}_{1}\mathbf{H}^{1}_{1}\mathbf{F}$ cannot be self-adjoint
 assuming the channel matrices $\mathbf{H}^{1}_{j} \neq \mathbf{H}^{1}_{i} $ $\forall j \neq i$. 
 This prevents using the methods proposed in \cite{belinschi2017analytic} for finding the AED of this polynomial.
 
\section{System Model With Secrecy Capacity}
The model from Section \ref{system_model} can be easily extended to the case in which an eavesdropper may be located at any point along the transmission path. We denote the received signal at the eavesdropper by

\begin{equation}
\hat{\mathbf{s}}^{n}_{i} = \sum_{j = 1}^{L^{n-1}} \hat{\mathbf{H}}^{n-1}_{j,i}\boldsymbol{\Theta}^{n-1}_{j} \mathbf{s}^{n-1}_{j}.
\end{equation}
Note that the term $\mathbf{s}^{n-1}_{j}$ is unchanged from \ref{received_vector} because up until the final hop, the path for the eavesdropper and legitimate receiver are the same. 

* We should only need to known the difference between the two composite channels to find the secrecy capacity. Even if we don't know the individual ones?
\section{Free Probability Analysis}

\chapter{Next Steps}
\section{Independent TODO items}
\begin{itemize}
\item 
	Show why the channels are free as assumed in decontamination work.
\end{itemize}
\section{Questions}
\begin{itemize}
\item 
	Just having the IRS present should give us a boost, do we need to actually consider selecting the phases?
	
\item 
	Confirm that the one example on how to pull out the hermetian matrices is a result of the trace property.
	
\item 
	Examples of secrecy codes apart from the one presented in the original wire-tap paper
	
\item 
	How to actually evaluate a pdf given a fixed point equation for the stieltjes transform and an expression for the
	AED in terms of the Steitjes transform. 
	
\item 
	Can we iterate the process used in the decontamination work (i.e) use the S-Transform on terms with $>2$
	 matrices (free variables).
\end{itemize}

\bibliography{bibliography}
\end{document}
