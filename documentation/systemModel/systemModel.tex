\documentclass[12pt,a4paper]{report}
\usepackage[utf8]{inputenc}
\usepackage{amsmath}
\usepackage{amsfonts}
\usepackage{amssymb}
\usepackage{graphicx, neuralnetwork,tikz}
\usepackage{float}
\usetikzlibrary{positioning}


\bibliographystyle{alpha}
\graphicspath{ {./figures/} }
\tikzstyle{process} = [rectangle, minimum width=3cm, minimum height=1cm, text centered, draw=black]
\tikzstyle{arrow} = [thick,->,>=stealth]

\tikzstyle{state}=[shape=circle,draw=blue!30,fill=blue!10]
\tikzstyle{observation}=[shape=rectangle,draw=orange!30,fill=orange!10]
\tikzstyle{lightedge}=[<-, dashed]
\tikzstyle{mainstate}=[state, thick]
\tikzstyle{mainedge}=[<-, thick]
\tikzstyle{block} = [draw,rectangle,thick,minimum height=2em,minimum width=2em]
\tikzstyle{sum} = [draw,circle,inner sep=0mm,minimum size=2mm]
\tikzstyle{connector} = [->,thick]
\tikzstyle{line} = [thick]
\tikzstyle{branch} = [circle,inner sep=0pt,minimum size=1mm,fill=black,draw=black]
\tikzstyle{guide} = []
\tikzstyle{snakeline} = [connector, decorate, decoration={pre length=0.2cm,
                         post length=0.2cm, snake, amplitude=.4mm,
                         segment length=2mm},thick, magenta, ->]




\title{Free Probability Analysis of Capacity in Relay Networks}
\author{Peter Hartig}

\begin{document}
\maketitle
\chapter{Introduction}
The goal of this work is to develop a framework to analyze the capacity of communication networks that incorporate intelligent reflective surfaces. 
\par
Questions
\begin{enumerate}
%\item Why does this not fit into the current tools we have for analyzing MIMO channels? Aspects of this that we currently lack. 
\item Are there similar results for amplify and forward relay networks? If so can we use the same tools?
\item For deterministic channels, do we know the capacity for this type of irs or amplify and forward relaying? TODO
\item Why do we need free probability and why aren't results to normal RMT sufficient. TODO
\item Explain why it is reasonable to consider a single coefficient of channel and no multipath. TODO
\item Can phases be design such that a specific receiver detects a single transmitted symbol?
\item Is it reasonable to assume that an IRS will know sufficient CSI?
\item Need to show if other user interference can be cancelled using IRS? yes for the case of complex 
\item Is it reasonable to think of complex as a two channel real MIMO system?
\end{enumerate}
\chapter{System Model}
\section{Lucinda's System Model}
\begin{itemize}
\item Decode and forward relaying so at each relay, the original signal is perfectly detected and then retransmitted after a precoding. 
\item Point to point with all relays having the same number of antennas and receiver usering having at least as many antennas as the transmitting user.
\item Assumes a single hop in the system. TODO check if it is common for IRS systems to have multiple hops
\item Allows for multiple relays to be in the single hop. L total relays
\item Assumes Source knowns no CSI
\item Assumes relays have perfect CSI of all channels. TOD Discuss if this is this realistic for such large systems and if not, if we can add uncertainty to model easily.
\item Not power allocation to the different antennas is performed?
\item 
\end{itemize}
My TODO
\begin{enumerate}
	\item
		Really have nice formalization of the final channel whose AED is desired.
	\item
		Then start building up with SISO one hop first then relay and finally up to the point where some form of FPT is
		required.
	\item
		Look at algo from Lucinda Paper
	\item
		Look at the other article Saba sent
	\item
		Start looking at how to incorporate ideas from the pilot decontamination into this
\end{enumerate}
\begin{itemize}
\item Why is the assumption of asymmetric important and what exactly does asymmetric mean? Means that the attenuation coefficient may not be the same in the link from relay to detector. This makes finding the distribution hard for some reason detailed in lucinda's paper.
\item Can any of this be used in a MIMO broadcast setting where no processing can be done at the receiver?
\item Need to look at optimal matched filtering for precoding. Will not be possible for IRS and also might not even assumed that this
	CSI would be known. Indeed it won't be able to do this precoding, so what do we lose?
\item Need to look at how to detect at the final end point for amplify and forward methods and IRS methods in general.
\item TODO derive standard ergodic capacity of MIMO channel. How the standard form with the determinant decomposes into the eigenvalue product. Also look at LogDet in problem 5 of Tutorial 1 of RMT
\item For Marcenco Pasteur, why is it reasonable to assumed variance is one over NR
\item IRS wouldnt have any impact in the case of a guassian channel unless it knows the incoming phases. The phase of a gaussian is uniform so adding a shift won't change this. 
\item Generally will not be able to optimize the secrecy capacity of the phase matrix because finding the CSI to eavedroppers would be hard. This would also impact the resulting estimate for the SNR based upon any noise injected into the system.

%\item 
\end{itemize}

\section{My System Model}
\begin{enumerate}
\item IRS: A surface made of individual elements each of which reflect the single signal received by the surface with a unique phase shift.
\item Most general: Individual users transmit simultaneously to the first set of IRS's. These relays hop through an arbitrary number of IRS's. The final set of IRS's reflect the signal out to a set of independent users. Possible? (For half duplex need to make sure number of relay sets makes sense)
\item How do we describe the signal coming off of an IRS? Large diagonal matrix with each reflecting element represented as one elements (one phase shift) along the diagonal. 
\item
Each element of the IRS can apply a phase shift to the incoming signal.
\item 
A set of $T$ transmitters is assumed to transmit $T$ statistically independent symbols in time period T1. This signal is then transmitted through $N$ sets of IRS's, each with $R_{n,i}$, $i \in [1... \; R^n_L]$ reflecting surfaces. The signal received at element $k$ of IRS $(n,i)$ is 
\begin{equation}
s^{n}_{i,k} = \sum_{j = 1}^{R^{n-1}_L} (\mathbf{h}^{n-1}_{j,i,k})^T \boldsymbol{\Theta}^{n-1}_{j}\mathbf{s}^{n-1}_{j}
\end{equation}
with $\mathbf{h}^{n-1}_{j,i,k}$ as the vector of channel coefficients between the $j$th IRS in the $(n-1)$st set of IRS's and $k$th element of the $i$th IRS in the $n$th set of IRS's.

\item
The vector of received signals for all elements at IRS $(n,i)$ is therefore
\begin{equation}
\mathbf{s}^{n}_{i} = \sum_{j = 1}^{R^{n-1}_L} (\mathbf{H}^{n-1}_{j,i})^T\boldsymbol{\Theta}^{n-1}_{j}\mathbf{s}^{n-1}_{j}.
\end{equation}

with $\mathbf{s}^{n}_{i} \in \mathbb{C}^{n_L}$

\end{enumerate}

\bibliography{bibliography}
\end{document}
