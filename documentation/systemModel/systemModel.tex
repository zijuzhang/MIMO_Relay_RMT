\documentclass[12pt,a4paper]{report}
\usepackage[utf8]{inputenc}
\usepackage{amsmath}
\usepackage{amsfonts}
\usepackage{amssymb}
\usepackage[colorlinks=true,linkcolor=blue]{hyperref}
\usepackage{graphicx,tikz}
\usepackage{float}
\usetikzlibrary{positioning}
\input defs.tex
\bibliographystyle{ieeetr}
\graphicspath{ {./figures/} }

\title{Free Probability Analysis of Capacity in Relay Networks}
\author{Peter Hartig}

\begin{document}
\maketitle
\tableofcontents
\chapter{Introduction}
The goal of this work is to consider the intelligent reflective surface (IRS) channel capacity and secrecy capacity using free probability theory. Specifically, this system is compared to two similar problems which have been shown to admit analysis using free probability. 
\par

\chapter{Problem Background}

\section{Information Theory Tools}

\begin{enumerate}
\item For a random MIMO communication channel and CSI known at the transmitter and Gaussian distributed transmit signal $\mathbf{x}$, the capacity ($\mathcal{C}$) of a point-to-point channel $\mathbf{H}$ is given by
\begin{equation}\label{capacity}
\Capacity{(\mathbf{H})} = \Expect\left[\underset{tr(\mathbf{Q}_{\mathbf{xx}}) < P_{Total}}{\mathrm{max}} \;\Log\left(|\mathbf{I}_{N_R}+\frac{1}{\sigma_n}\mathbf{H}\mathbf{Q}_{\mathbf{xx}}\mathbf{H}^H|\right)\right].
\end{equation}
*Question: Without an analytic solution to the waterfilling problem used to choose $\mathbf{Q}_{\mathbf{xx}}$, how can we take an expectation? Answer: need to do a Monte Carlo simulation (in MIMO book) so probably will not be able to include the beamformer in any free probability solution.

\item
In the case in which no CSI is known at the transmitter and the power at the transmitter, $P_{\text{Total}}$, is distributed equally across transmit antennas Equation \ref{capacity} becomes
\begin{equation}
\Capacity{(\mathbf{H})} = \Expect\left[\Log\left(|\mathbf{I}_{N_R}+\frac{P_{\text{Total}}}{N_T \sigma_n}\mathbf{H}\mathbf{H}^H|\right)\right].
\end{equation} 
Assuming asymptotically large antenna arrays, and given the distribution for the eigenvalues of $\mathbf{H}\mathbf{H}^H$ which we denote by $p_{\lambda}(\lambda)$, Equation \ref{capacity} can be rewritten as 
\begin{equation}
\Capacity{(\mathbf{H})} = N_r \Expect\left[\Log\left(1+\frac{P^{\text{Total}}}{N_T \sigma_n}\lambda\right)\right].
\end{equation}
 in which the expectation is taken over $p_{\lambda}(\lambda)$.

\item The secrecy capacity of the general wire-tapped communication channel with perfect knowledge of the channels to both the legitimate receiver, $\mathbf{H} \in \mathbb{C}_{N_r \times N_t}$, and eavesdropper, $\hat{\mathbf{H}}\in \mathbb{C}_{N_e \times N_t}$, is given by 

\begin{gather}\label{secrecy_capacity}
\Capacity{(\mathbf{H})} = \Expect\left[\underset{tr{\mathbf{Q}_{xx}} < P_{Total}}{\mathrm{max}} \;\left[\Log\left(|\mathbf{I}_{N_R}+\frac{1}{\sigma_n}\mathbf{H}\mathbf{Q}_{xx}\mathbf{H}^H|\right) -
\Log\left(|\mathbf{I}_{N_R}+\frac{1}{\sigma_n}\hat{\mathbf{H}}\mathbf{Q}_{xx}\hat{\mathbf{H}}^H|\right)
\right] \right].
\end{gather}
Again assuming $P_{\text{Total}}$ is distributed equally across transmit antennas such that $\mathbf{Q}_{xx} = \frac{P_{\text{Total}}}{N_t}\mathbf{I}_{N_t}$, and that both the legitimate user and eavesdropper have sufficient antennas such that $N_r > N_t$ and $N_e >N_t$,  this can be written as 

\begin{gather}\label{secrecy_capacity}
\Capacity{(\mathbf{H})} = \Expect\left[N_r \Log\left(1+\frac{P^{\text{Total}}}{N_T \sigma_n}\lambda \right) -
N_e \Log\left(1+\frac{P^{\text{Total}}}{N_T \sigma_n}\hat{\lambda} \right) \right]
\end{gather}
in which $\lambda$ and $\hat{\lambda}$ denote the eigenvalues of $\mathbf{H}\mathbf{H}^H$ and $\hat{\mathbf{H}}\hat{\mathbf{H}}^H$ respectively. Generally, finding capacity will require knowledge of the joint asymptotic eigenvalue distribution $p_{\lambda}(\lambda,\hat{\lambda})$.

\end{enumerate}
\section{Random Matrix Theory and Free Probability Tools}

\subsubsection{Matrix AEDs}
For some matrices, we know the AED directly. For more general classes of matrices we find the AED using the known moments of the matrix to find the Stieltjes transform and then inverting the Stieltjes transform to find the true distribution. Generally, given the distributions of two matrices, the distribution for compositions of these matrices is difficult to find. 

\subsubsection{Free Probability}
In case of asymptotically large matrices, we can consider the matrices and free variables and use the theories of free probability. The distribution resulting from the composition of free variables can be found using the R-transform and S-transform to linearize addition and multiplication respectively.

\subsubsection{Other Tools from Free Probability}
Here the current tools available from free probability analysis are reviewed in order to clarify the types of polynomials (composite channels) which can be analyzed.

\section{Itelligent Reflective Surfaces}
In the following work, the MIMO channel with IRSs is considered. 


\chapter{Comparison to Previous Work}
First, the two systems to which the IRS channel is compared are first given and relevant details are highlighted. 
\section{Notes from Decode and Forward Relay Work in \cite{hadley2019capacity}}
\subsection{System Model}

\begin{itemize}
\item Assuming decode and forwarding at relays and a matched filter precoding at the transmitter, the resulting received signal is given by 
\begin{equation}
\mathbf{y} = \mathbf{H}_{Total} \mathbf{s} +  \mathbf{n} = \sum_{r=1}^{R_L} \mathbf{H}_{r}\mathbf{H}_{r}^{\dagger}\mathbf{s} +  \mathbf{n}
\end{equation}
for which the composite channel is given by 
\begin{equation}
\mathbf{H}_{Total} = \mathbf{H}_{1}\mathbf{H}_{1}^{H} +\cdots +\mathbf{H}_{R_L}\mathbf{H}_{R_L}^{H}.
\end{equation}
Assuming the covariance matrix $\mathbf{Q}_s = \mathbf{I}_{N_R}$, the ergodic capacity is given by 
\begin{equation}\label{ergodic_capacity}
E\left[\text{log}_2(\mathbf{I}_{N_R} + \frac{P}{\sigma_n}\mathbf{H}_{Total}\mathbf{H}_{Total}^H)\right].
\end{equation}
We consider $\mathbf{H}_{Total}\mathbf{H}_{Total}^H$ as a polynomial in the self-adjoint variables $\mathbf{H}_{i}\mathbf{H}_{i}^{H}$.
\end{itemize}
\subsection{Free Probability Analysis}
The use of additive free convolution in

\section{Notes Pilot Decontamination Work}
\subsection{System Model}
Here we consider the signal 
\begin{equation}
\mY = \mH\mX + \mH_I \mX_I + \mW
\end{equation}
with covariance matrix 
\begin{gather*}
\mQ_{YY} = (\mH\mX + \mH_I \mX_I + \mW)(\mH\mX + \mH_I \mX_I + \mW)^H = 
\\
\mH\mX\mX^H\mH^H + \mH\mX\mX_I^H\mH_I^H + \mH\mX\mW^H +\\ \mH_I \mX_I \mX^H\mH^H + \mH_I \mX_I\mX_I^H\mH_I^H + \mH_I \mX_I\mW^H + \mW\mX^H\mH^H + \mW\mX_I^H\mH_I^H + \mW\mH_I^H.
\end{gather*}
Clearly, the individual terms of this polynomial are not self-adjoint.
\subsection{Free Probability Analysis}
Need to figure out how the actual steiltjes transform of this polynomial is found.
- There is a fixed point equation for finding this transform.

- Ultimately we want to have a form of the Steiltjes transform for the entire channel AED using a fixed point equation.
\chapter{IRS Work}

\section{System Model}\label{system_model}



\begin{enumerate}
\item 
	We assume non-frequency selective channels using techniques like OFDM. (Is this always going to be reasonable?)
	
%\item 
%	We assume a transmitter without channel state information and thus power is distributed equally over all antennas such that
%	\begin{equation}
%	\mathbf{Q_{\mathbf{x}}} = \frac{P^{\text{Total}}}{N_T}
%	\end{equation}
%	in which $P^{\text{Total}}$ is the power available at the transmitter, and $N_T$ is the number of antennas at the transmitter. 
\item 
	We assume a transmitter with channel state information and beamforming capability such that the transmitted signal is given by 
	\begin{equation}
	\mathbf{x} = \mathbf{F}\mathbf{s}^0
	\end{equation}
	in  which $\mathbf{U}$ is a beamforming matrix, $\mathbf{s}^0$ is the normalized vector of transmitted symbols such that $\mathbf{Q_{\mathbf{s}^0,\mathbf{s}^0}} = \mathbf{I}_{N_R}$. 
	
\item Intelligent Reflective Surfaces (IRS) are placed between transmitter and receiver. The IRS is made of individual elements each of which can reflect the received signal with a unique, chosen phase shift.

%\item Individual users transmit simultaneously to the first set of IRS's. This transmit signal hops through an arbitrary number of IRS's. The final set of IRS's reflect the signal to a set of independent users. Possible? (For half duplex need to make sure number of relay sets makes sense)

\item 
The signal $\mathbf{x}$ is transmitted through $N$ sets of IRS's. The set $n, \forall n = [1 \cdots N]$, has $L^n$ reflecting surfaces. Each reflecting surface in set $n$ is indexed by $R^n_{i}, \forall i = [1\cdots L^n]$ and has $K_{i}^{n}$ elements.
\item 
 The  signal received at element $k$ of IRS $R_i^n$ is 
\begin{equation}
s^{n}_{i,k} = \sum_{j = 1}^{L^{n-1}} (\mathbf{h}^{n}_{j,i,k})^T \boldsymbol{\Theta}^{n-1}_{j}\mathbf{s}^{n-1}_{j}
\end{equation}
with $\boldsymbol{\Theta}^{n-1}_{j}$ as the phases shifts applied by $R_j^{n-1}$ and $\mathbf{h}^{n-1}_{j,i,k}$ as the vector of channel coefficients between $R_j^{n-1}$ and element $k$ of $R_i^n$.

\item
The \emph{vector} of received signals for all elements at $R^n_{i}$, is given by
\begin{equation}\label{received_vector}
\mathbf{s}^{n}_{i} = \sum_{j = 1}^{L^{n-1}} \mathbf{H}^{n}_{j,i}\boldsymbol{\Theta}^{n-1}_{j}\mathbf{s}^{n-1}_{j}.
\end{equation}
in which $\mathbf{H}^{n}_{j,i} = [\mathbf{h}^{n}_{j,i,1} \cdots \mathbf{h}^{n}_{j,i,K_{j}^{n}}]^T$.



\item For IRS $i$ in set $n$, $R^n_{i}$, the received signal vector is given in a recursive form as
\begin{equation}\label{general_received}
\mathbf{s}^{n}_{i} = \sum_{j = 1}^{L^{n-1}} \mathbf{H}^{n}_{j,i}\boldsymbol{\Theta}^{n-1}_{j} \left[
\sum_{k = 1}^{L^{n-2}} \mathbf{H}^{n-1}_{k,i}\boldsymbol{\Theta}^{n-2}_{k}\mathbf{s}^{n-2}_{k} \right].
\end{equation}
The received signal vector for the first set of IRSs, $R^n_{i}$ $\forall i = [1 \cdots L^{1}]$ is given by 
\begin{equation}
\mathbf{s}^{1}_{i} = \mathbf{H}^{1}_{i}\mathbf{F}\mathbf{s}^0
\end{equation}
\item 
	Finally, using $\mathbf{G}_b$ and $\mathbf{G}_e$ as the channel matrices between the last set of IRSs and a legitimate users and an eavesdropper respectively, the received signals for the legitimate user is given by 
	\begin{equation}\label{legit_received}
\mathbf{r}_{b} = \sum_{j = 1}^{L^{N}} \mathbf{G}_{b}\boldsymbol{\Theta}^{N}_{j}\mathbf{s}^{N}_{j}
\end{equation}
	and the received signal for the eavesdropper is given by 
	\begin{equation}\label{tap_received}
\mathbf{r}_{e} = \sum_{j = 1}^{L^{N}} \mathbf{G}_{e}\boldsymbol{\Theta}^{N}_{j}\mathbf{s}^{N}_{j}
\end{equation}
\item
	Note that this model assumes sufficient spacing between set $n$ and set $n+2$ such that we consider the signal reflected in set $n$ only at set $n+1$ and not for any $n<k$.
	
\item 
	Note that by allowing for $L^{n} >1$ (more than one IRS in each step of the path) highlights the fact that the channels to each IRS may have a unique attenuation coefficient. If this were not the case, this system model could be equivalently modeled by a \emph{single} large IRS in each step of the path. 

\end{enumerate}
To better analyze types of components within the polynomial characterizing the resulting, composite channel we consider only the legitimate receiver channel for specific cases of this general system model. Before these examples, we first consider the optimization of the phase-shifting array at each IRS.

\subsection{Optimization at IRS}
Assuming full CSI at each IRS (detail what this means), one choice for optimization of the phase-shift array $\boldsymbol{\Theta}$ is given by 

    \begin{equation}
    \begin{array}{ll}
    \underset{\boldsymbol{\Theta}}{\text{maximize }}   & \Expect\left[\Log\left(|\mathbf{I}_{N_R}+\frac{P_{\text{Total}}}{N_T \sigma_n}\mathbf{H}_{Total}\mathbf{H}_{Total}^H|\right)\right]
    \\
    \mbox{subject to } & tr\left(
    \left(\boldsymbol{\Theta}\boldsymbol{\Theta}^H - \mathbf{I}  \right)
    \left(\boldsymbol{\Theta}\boldsymbol{\Theta}^H - \mathbf{I}  \right)^H
    \right) = 0
    \end{array}
    \label{general_irs_opt}
    \end{equation}

in which $\boldsymbol{\Theta}$ represents the choice of all $\boldsymbol{\Theta}^n_i$ and $\mathbf{H}_{Total}$ is the composite channel from transmitter to receiver. The constraint enforces that $\boldsymbol{\Theta}^n_i$ is diagonal and that each element on the diagonal has unit modulus. 
Noting that terms in the matrix  $\mathbf{H}_{Total}\mathbf{H}_{Total}^H$ will generally have terms that are both convex and concave in the diagonal elements of $\boldsymbol{\Theta}^n_i$, this optimization problem is generally non-convex. A similar version of this problem is presented and analyzed in \cite{wu2019intelligent}.
\par
We now considering this problem in the case by case basis in the following examples. 


\subsection{Case in which $L_n=1$ and $N=1$}
We now consider the case with only a single IRS in each hop between the transmitter and receiver. 
Equation \ref{legit_received} now becomes 
	\begin{equation}
\mathbf{r}_{b} =  \mathbf{G}_{b}\boldsymbol{\Theta}^{1}\mathbf{H}^{1}\mathbf{F}\mathbf{s}^0
\end{equation}
Assuming no CSI at the transmitter, the composite channel covariance matrix is given by 
\begin{equation}
\mathbf{Q}_{bb} = [\mathbf{G}_{b}\boldsymbol{\Theta}^{1}\mathbf{H}^{1}][\mathbf{G}_{b}\boldsymbol{\Theta}^{1*}\mathbf{H}^{1}]^H
\end{equation}
This is similar to the example given in \cite[Section 4.10]{muller2013applications}. As we are interested in the distribution of the non-zero eigenvalues of this matrix, we can instead consider the distribution of 
\begin{equation}
\tilde{\mathbf{Q}}_{bb} = [\mathbf{G}_{b}\boldsymbol{\Theta}^{1}\boldsymbol{\Theta}^{1*}\mathbf{G}_{b}^H]\mathbf{H}^{1}\mathbf{H}^{1H}
\end{equation}

Noting that $\boldsymbol{\Theta}^{1}\boldsymbol{\Theta}^{1*} = \mathbf{I}$ this becomes

\begin{equation}
\tilde{\mathbf{Q}}_{bb} = [\mathbf{G}_{b}\mathbf{G}_{b}^H]\mathbf{H}^{1}\mathbf{H}^{1H}
\end{equation}

In order to use the same methods for finding the AED as \cite[Section 4.10]{muller2013applications}, it must first be shown that $\left(\left( \mathbf{G}_{b},\mathbf{G}_{b}^\dagger \right)
,\left( \mathbf{H}^{1},\mathbf{H}^{1^\dagger} \right)
\right)$
form a free family. 
\par
Referring to Section \ref{irs_opt}, we find that given the assumed complete knowledge of the CSI at base stations, 

\subsection{Case in which $L_n=1$}
We now generalize to a system with $N$ IRS hops but still a \emph{single} IRS in each hop between the transmitter and receiver. 
Equation \ref{general_received} now becomes 
\begin{equation}\label{}
\mathbf{r}_{b} =  \mathbf{G}_{b}\boldsymbol{\Theta}^{N}\mathbf{s}^{N}
\end{equation}
in which $\mathbf{s}^{N}$ is defined in Equation \ref{general_received}.

In this case, the composite channel is given by 
\begin{equation}
\mathbf{H}_{Total} = \mathbf{G}_{b}\boldsymbol{\Theta}^{N}\mathbf{H}^{N} \cdots \boldsymbol{\Theta}^{1}\mathbf{H}^{1}
\end{equation}
with covariance matrix 
\begin{equation}
\mathbf{Q}_{bb} = \mathbf{H}_{Total}\mathbf{H}_{Total}^H = [\mathbf{G}_{b}\boldsymbol{\Theta}^{N}\mathbf{H}^{N} \cdots \boldsymbol{\Theta}^{1}\mathbf{H}^{1}][\mathbf{G}_{b}\boldsymbol{\Theta}^{N}\mathbf{H}^{N} \cdots \boldsymbol{\Theta}^{1}\mathbf{H}^{1}]^H
\end{equation}
Using the rotational property of the trace as in the previous case
We can equivalently look for the AED of the matrix 
\begin{equation}
\tilde{\mathbf{Q}}_{bb} = [\mathbf{G}_{b}\boldsymbol{\Theta}^{N}\mathbf{H}^{N} \cdots \boldsymbol{\Theta}^{1}][\mathbf{G}_{b}\boldsymbol{\Theta}^{N}\mathbf{H}^{N} \cdots \boldsymbol{\Theta}^{1}]\mathbf{H}^{1}\mathbf{H}^{1H}].
\end{equation}
Iterating this procedure and noting that $\boldsymbol{\Theta}^{n}\boldsymbol{\Theta}^{n*} = \mathbf{I}$ $\forall n \in [1\cdots N]$, this is exactly the case from \cite[Section 4.10]{muller2013applications}.

\subsection{Case in which $N=1$}
We now consider the case in which there is one IRS hop between the transmitter and receiver with arbitrarily many IRSs. 
Equation \ref{legit_received} now becomes 
\begin{equation}
\mathbf{r}_{b} = \sum_{j = 1}^{L^{1}} \mathbf{G}_{b}\boldsymbol{\Theta}^{1}_{j}\mathbf{H}^{1}_{j}\mathbf{F}\mathbf{s}^0
\end{equation}
with composite channel
\begin{equation}
\mathbf{H}_{Total} = \sum_{j = 1}^{L^{1}} \mathbf{G}_{b}\boldsymbol{\Theta}^{N}_{j}\mathbf{H}^{1}_{j}
\end{equation}
and covariance matrix

\begin{equation}
\mathbf{Q}_{bb} = [\mathbf{G}_{b}\boldsymbol{\Theta}^{N}_{1}\mathbf{H}^{1}_{1} + \cdots + \mathbf{G}_{b}\boldsymbol{\Theta}^{N}_{L^{1}}\mathbf{H}^{1}_{L^{1}}]
[\mathbf{G}_{b}\boldsymbol{\Theta}^{N}_{1}\mathbf{H}^{1}_{1} + \cdots + \mathbf{G}_{b}\boldsymbol{\Theta}^{N}_{L^{1}}\mathbf{H}^{1}_{L^{1}}]^H
\end{equation}
\section{System Model With Secrecy Capacity}
The model from Section \ref{system_model} can be easily extended to the case in which an eavesdropper may be located at any point along the transmission path. We denote the received signal at the eavesdropper by

\begin{equation}
\hat{\mathbf{s}}^{n}_{i} = \sum_{j = 1}^{L^{n-1}} \hat{\mathbf{H}}^{n-1}_{j,i}\boldsymbol{\Theta}^{n-1}_{j} \mathbf{s}^{n-1}_{j}.
\end{equation}
Note that the term $\mathbf{s}^{n-1}_{j}$ is unchanged from \ref{received_vector} because up until the final hop, the path for the eavesdropper and legitimate receiver are the same. 

* We should only need to known the difference between the two composite channels to find the secrecy capacity. Even if we don't know the individual ones?
\section{Free Probability Analysis}

\chapter{TODO}
\section{My TODO}
\begin{itemize}
\item 
	Show why the channels are free. 
\item
	Use method from mueller paper to solve further.
\end{itemize}
\section{Questions}
\begin{itemize}
\item 
	Just having the IRS present should give us a boost, do we need to actually consider selecting the phases?
	
\item 
	Confirm that the one example on how to pull out the hermetian matrices is a result of the trace property.
	
\end{itemize}

\bibliography{bibliography}
\end{document}
