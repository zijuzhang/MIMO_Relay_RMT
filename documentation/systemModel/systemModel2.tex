\documentclass[12pt,a4paper]{report}
\usepackage[utf8]{inputenc}
\usepackage{amsmath}
\usepackage{amsfonts}
\usepackage{amssymb}
\usepackage[colorlinks=true,linkcolor=blue]{hyperref}
\usepackage{graphicx,tikz}
\usepackage{float}
\usetikzlibrary{positioning}
\input defs.tex
\bibliographystyle{ieeetr}
\graphicspath{ {./figures/} }

\title{Progress Report}
\author{Peter Hartig}

\begin{document}
\maketitle
\tableofcontents


\section{Summary of Work}
Below is a summary of my work up to now.
\section{System Model}
I consider a flat-fading MIMO broadcast channel with a linear precoding at the transmitter. 
Using a basic model with an IRS component and line of sight (LOS) component gives the received signal 
	 \begin{equation}
		\mathbf{r} = \mathbf{H}_{Total}\mathbf{F}\mathbf{x}.
	\end{equation}
	The total channel, is given by
	\begin{equation}
	\mathbf{H}_{Total} =  \mathbf{H}_{2}\boldsymbol{\Phi}\mathbf{H}_{1} + \mathbf{G}
	\end{equation}
	with both the IRS component $ \mathbf{H}_{2}\boldsymbol{\Phi}\mathbf{H}_{1}$ and 
	the LOS component $\mathbf{G}$. To simplify notation I will assume 
	$\mathbf{H}_{1} \in \mathbb{C}_{N \times N}$ ...

\section{Channel Capacity}
First the capacity of the total channel $\mathbf{H}_{Total}$ is considered. 
Following the methods from (non-hermetian), we can find the AED of the channel 
covariance matrix, $\mathbf{H}_{Total}\mathbf{H}_{Total}^H$ using the asymptotic distribution of the "symmetrized singular values" of $\mathbf{H}_{Total}$.
One problem I am currently working on is that the fixed point equation  provided in (non-hermetian) assumes that the
 elements of $ \mathbf{H}_{1}$ and  $\mathbf{H}_{2}$ have variance $\frac{1}{RS}$. Typically we want these
  matrices to have elements with variance $\frac{1}{R}$.
  If I were to use the equation from (non-hermetian), the IRS component would have no impact on the capacity of system. In order to resolve this, I will need to re-derive this which requires using subordination methods. 

\subsection{Canceling Phases}
Additive free convolution of the R-Transform holds for symmetrized singular value distributions. Therefore we can consider $R_{\mathbf{H}_{Total}} = R_{\mathbf{H}_{2}\boldsymbol{\Phi}\mathbf{H}_{1}} + R_{\mathbf{G}}$. In order to find the individual terms $R_{\mathbf{H}_{2}\boldsymbol{\Phi}\mathbf{H}_{1}}$ and $R_{\mathbf{G}}$ we will use the property 
\begin{equation}
\tilde{G}_{B}(s) = sG_{BB^{\dagger}}(s^2).
\end{equation}
For the term $R_{\mathbf{H}_{2}\boldsymbol{\Phi}\mathbf{H}_{1}}$ we begin by finding 
$G_{\mathbf{H}_{2}\boldsymbol{\Phi}\mathbf{H}_{1}[\mathbf{H}_{2}\boldsymbol{\Phi}\mathbf{H}_{1}]^H}(s)$ 
Using the method shown in (vector channel paper), the rotational property of the S-Transform
\begin{equation}
S_{C_N}(z) = \frac{z+1}{z+\chi_N} S_{\tilde{C}_N}(\frac{z}{\chi_N}).
\end{equation}
is used with 
\begin{equation}
\tilde{C}_N = \boldsymbol{\Phi}^H\mathbf{H}_{2}^H\mathbf{H}_{2}\boldsymbol{\Phi}\mathbf{H}_{1}\mathbf{H}_{1}^H
\end{equation}.
Applying the rotational property again to the term 
\begin{equation}
\boldsymbol{\Phi}^H\mathbf{H}_{2}^H\mathbf{H}_{2}\boldsymbol{\Phi}
\end{equation}
we get 
\begin{equation}
\boldsymbol{\Phi}\boldsymbol{\Phi}^H\mathbf{H}_{2}^H\mathbf{H}_{2}
\end{equation}
with $\boldsymbol{\Phi}\boldsymbol{\Phi}^H = \mathbf{I}$.
From this, we can conclude that the present of the phase-shift matrix $\boldsymbol{\Phi}$ does not influence the
AED of the channel covariance matrix and thus does not affect the channel capacity. 

\section{Optimization}



\section{Simulation Results}

\subsection{Evaluating Impact of Cross-terms on AED}
Here we wanted to see if the final AED is influenced by the cross terms in the polynomial describing the channel
covariance matrix given by $\mathbf{H}_{Total}\mathbf{H}_{Total}^H$

\begin{figure}[H]
	\includegraphics[width=\textwidth]{results/cross}
\caption{Eigenvalue histogram for N=500. Cross terms represent the eigenvalues of the matrix sum $(\mathbf{H}_{2}\boldsymbol{\Phi}\mathbf{H}_{1}\mathbf{G}^H + \mathbf{G}[\mathbf{H}_{2}\boldsymbol{\Phi}\mathbf{H}_{1}]^H)$ and the non-cross terms are the eigenvalues of the matrix sum $(\mathbf{H}_{2}\boldsymbol{\Phi}\mathbf{H}_{1}[\mathbf{H}_{2}\boldsymbol{\Phi}\mathbf{H}_{1}]^H + \mathbf{G}\mathbf{G}^H)$}
\label{irs_figure}
\end{figure}

\subsection{Impact of Phases on total Capacity}
It is hypothesized that in the asymptotic limit, the phases will not impact the capacity. In order to check this numerically I evaluated the capacity while holding the channel values constant for different values for the IRS phases. Here statistics are shown for 50 random realizations of the phases at the IRS. 
Phases, $e^{j\phi}$, are generated using a uniform distribution of $\phi \in \text{Uniform}[0,2\pi]$ and.
\begin{itemize}
\item 
Capacity variance:
 8.968116152773685e-06
\item 
Capacity average:
 1.9860953144228657
\item 
Capacity min:
 1.9783515328782062
\item 
Capacity max:
 1.993849732554225
\end{itemize}







	

	
	



\bibliography{bibliography}
\end{document}
