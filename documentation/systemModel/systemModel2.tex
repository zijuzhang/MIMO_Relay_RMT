\documentclass[12pt,a4paper]{report}
\usepackage[utf8]{inputenc}
\usepackage{amsmath}
\usepackage{amsfonts}
\usepackage{amssymb}
\usepackage[colorlinks=true,linkcolor=blue]{hyperref}
\usepackage{graphicx,tikz}
\usepackage{float}
\usetikzlibrary{positioning}
\input defs.tex
\bibliographystyle{ieeetr}
\graphicspath{ {./figures/} }

\title{Progress Report}
\author{Peter Hartig}

\begin{document}
\maketitle
\tableofcontents


\section{Summary of Work}
Below is a summary of my work up to now.
\section{System Model}
I consider a flat-fading MIMO broadcast channel with a linear precoding at the transmitter. 
Using a basic model with an IRS component and line of sight (LOS) component gives the received signal 
	 \begin{equation}\label{system_model}
		\mathbf{r} = \mathbf{H}_{Total}\mathbf{F}\mathbf{x}.
	\end{equation}
	The total channel, is given by
	\begin{equation}
	\mathbf{H}_{Total} =  \mathbf{H}_{2}\boldsymbol{\Phi}\mathbf{H}_{1} + \mathbf{G}
	\end{equation}
	with both the IRS component $ \mathbf{H}_{2}\boldsymbol{\Phi}\mathbf{H}_{1}$ and 
	the LOS component $\mathbf{G}$. To simplify notation I will assume 
	$\mathbf{H}_{1},\mathbf{H}_{2}, \mathbf{G} \in \mathbb{C}_{N \times N}$

\section{Channel Capacity}
First the capacity of the channel $\mathbf{H}_{Total}$ is considered without precoding at the transmitter. 
Following the methods from (non-hermetian), we can find the AED of the channel 
covariance matrix, $\mathbf{H}_{Total}\mathbf{H}_{Total}^H$ using the asymptotic distribution of the "symmetrized singular values" of $\mathbf{H}_{Total}$.
One problem I am currently working on is that the fixed point equation  provided in (non-hermetian) assumes that the
 elements of $ \mathbf{H}_{1}$ and  $\mathbf{H}_{2}$ have variance $\frac{1}{RS}$. Typically we want these
  matrices to have elements with variance $\frac{1}{R}$.
  If I were to use the equation from (non-hermetian), the IRS component would have no impact on the capacity of system. In order to resolve this, I will need to re-derive this which requires using subordination methods. 

\subsection{Canceling Phases}
Additive free convolution of the R-Transform holds for symmetrized singular value distributions. Therefore we can consider $R_{\mathbf{H}_{Total}}(w) = R_{\mathbf{H}_{2}\boldsymbol{\Phi}\mathbf{H}_{1}}(w) + R_{\mathbf{G}}(w)$. In order to find the individual terms $R_{\mathbf{H}_{2}\boldsymbol{\Phi}\mathbf{H}_{1}}(w)$ and $R_{\mathbf{G}}(w)$, we use the property 
\begin{equation}\label{svd_aed_property}
\tilde{G}_{B}(s) = sG_{BB^{\dagger}}(s^2).
\end{equation}
For the term $R_{\mathbf{H}_{2}\boldsymbol{\Phi}\mathbf{H}_{1}}$ we begin by finding 
$G_{\mathbf{H}_{2}\boldsymbol{\Phi}\mathbf{H}_{1}[\mathbf{H}_{2}\boldsymbol{\Phi}\mathbf{H}_{1}]^H}(s)$ in order to use property \eqref{svd_aed_property} to find $G_{\mathbf{H}_{2}\boldsymbol{\Phi}\mathbf{H}_{1}}(s)$
\par
To find $G_{\mathbf{H}_{2}\boldsymbol{\Phi}\mathbf{H}_{1}[\mathbf{H}_{2}\boldsymbol{\Phi}\mathbf{H}_{1}]^H}(s)$ we define
\begin{equation}
C_N = \mathbf{H}_{2}\boldsymbol{\Phi}\mathbf{H}_{1}[\mathbf{H}_{2}\boldsymbol{\Phi}\mathbf{H}_{1}]^H
\end{equation}
and the rotated version 
\begin{equation}
\tilde{C}_N = \boldsymbol{\Phi}^H\mathbf{H}_{2}^H\mathbf{H}_{2}\boldsymbol{\Phi}\mathbf{H}_{1}\mathbf{H}_{1}^H
\end{equation} in order to use the rotational property of the S-Transform shown in (vector channel paper)
\begin{equation}\label{rotation_property}
S_{C_N}(z) = \frac{z+1}{z+\chi_N} S_{\tilde{C}_N}(\frac{z}{\chi_N}).
\end{equation}
in which $\chi_N$ is the aspect ratio of the channel matrices. Here we assume $\chi_N = 1$
Applying the rotational property again to the term 
\begin{equation}
\boldsymbol{\Phi}^H\mathbf{H}_{2}^H\mathbf{H}_{2}\boldsymbol{\Phi}
\end{equation}
we get 
\begin{equation}
\boldsymbol{\Phi}\boldsymbol{\Phi}^H\mathbf{H}_{2}^H\mathbf{H}_{2}
\end{equation}
with $\boldsymbol{\Phi}\boldsymbol{\Phi}^H = \mathbf{I}$.
From this, we can conclude that the phase-shift matrix $\boldsymbol{\Phi}$ does not influence the
AED of the channel covariance matrix and thus does not affect the channel capacity. 

\section{Optimization}
Now we consider how to choose the transmit covariance matrix $\mathbf{Q}_x$ and $\boldsymbol{\Phi}$ in order to optimize an objective.
First, We will optimize the precoder and IRS phases to maximize the channel capacity.
		    \begin{equation}
    \begin{array}{ll}
    \underset{\boldsymbol{\Phi}, \mathbf{Q}_x}{\text{maximize }}   & \Expect\left[\Log\left(|\mathbf{I}_{N_R}+\frac{P_{\text{Total}}}{N_T \sigma_n}[\mathbf{H}_{2}\boldsymbol{\Phi}\mathbf{H}_{1} + \mathbf{G}]\mathbf{Q}_x[\mathbf{H}_{2}\boldsymbol{\Phi}\mathbf{H}_{1} + \mathbf{G}]^H|\right)\right]
    \\
    \mbox{subject to } & \|\phi_i\|^2 = 1
    \\
    & tr(\mathbf{Q}_x) \leq P_{\text{Total}}
%        \\
%    & \tilde{\mathbf{f}}_i \mathbf{h}_j = 0, \; i\neq j, \; \forall i,j \in N
    \end{array}
    \label{general_irs_opt}
    \end{equation} 

If the proposal above is true and the phase matrices do no affect the capacity in the asymptotic case, then it would seem that standard water filling could be performed in order to choose the covariance matrix $\mathbf{Q}_x$.

\section{Simulation Results}

\subsection{Evaluating Impact of Cross-terms on AED}
Here we wanted to see if the final AED is influenced by the cross terms in the polynomial describing the channel
covariance matrix given by $\mathbf{H}_{Total}\mathbf{H}_{Total}^H$. The case in which elements of the channel matrices are zero-mean with variance  $\frac{1}{N}$ is shown in \ref{cross}

\begin{figure}[H]
	\includegraphics[width=\textwidth]{results/cross}
\caption{Eigenvalue histogram for N=500. Cross terms represent the eigenvalues of the matrix sum $(\mathbf{H}_{2}\boldsymbol{\Phi}\mathbf{H}_{1}\mathbf{G}^H + \mathbf{G}[\mathbf{H}_{2}\boldsymbol{\Phi}\mathbf{H}_{1}]^H)$ and the non-cross terms are the eigenvalues of the matrix sum $(\mathbf{H}_{2}\boldsymbol{\Phi}\mathbf{H}_{1}[\mathbf{H}_{2}\boldsymbol{\Phi}\mathbf{H}_{1}]^H + \mathbf{G}\mathbf{G}^H)$}
\label{cross}
\end{figure}

If the elements of the channel matrix are, however, not zero-mean these cross terms will impact the channel AED. 
This scenario is shown in \ref{cross_non_zero_mean}.

\begin{figure}[H]
	\includegraphics[width=\textwidth]{results/cross_non_zero_mean}
\caption{Eigenvalue histogram for N=500 with non-zero mean elements.}
\label{cross_non_zero_mean}
\end{figure}

\subsection{Impact of Phases on total Capacity}
It is hypothesized that in the asymptotic limit, the phases will not impact the capacity. In order to check this numerically I evaluated the capacity while holding the channel values constant for different values for the IRS phases. Here statistics are shown for 100 random realizations of the phases at the IRS. 
Phases, $e^{j\phi}$, are generated using a uniform distribution of $\phi \in \text{Uniform}[0,2\pi]$. This is clearly not rigorous but perhaps provides some insight. 
$N_t = N_r = M = 500$
\begin{itemize}
\item
Capacity variance:
 1.1512987368371963e-05
 \item
Capacity average:
 1.990962024050068
 \item
Capacity min:
 1.9818993465470727
 \item
Capacity max:
 1.9986482518130326
\end{itemize}

If, however, the aspect ratio of the channel matrices are not 1, the results below reflect that the phases play a more important role in determining the capacity. This supports the terms of the in equation \eqref{rotation_property} highlighted in red below. 

\begin{equation}\label{rotation_property}
S_{C_N}(z) = \frac{z+1}{z+ \textcolor{red}{\chi_N}} S_{\tilde{C}_N}(\frac{z}{\textcolor{red}{\chi_N}}).
\end{equation}

As indicated in the simulation with $N_t = N_r = 4$ and $M = 100$

\begin{itemize}
\item 
Capacity variance:
 0.04863896876118487
 \item
Capacity average:
 1.695035095091801
 \item
Capacity min:
 1.2342437149649095
 \item
Capacity max:
 2.320914548447325
\end{itemize}






	

	
	



\bibliography{bibliography}
\end{document}
