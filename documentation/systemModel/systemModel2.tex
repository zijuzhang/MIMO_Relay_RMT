\documentclass[12pt,a4paper]{report}
\usepackage[utf8]{inputenc}
\usepackage{amsmath}
\usepackage{amsfonts}
\usepackage{amssymb}
\usepackage[colorlinks=true,linkcolor=blue]{hyperref}
\usepackage{graphicx,tikz}
\usepackage{float}
\usetikzlibrary{positioning}
\input defs.tex
\bibliographystyle{ieeetr}
\graphicspath{ {./figures/} }

\title{Progress Report}
\author{Peter Hartig}

\begin{document}
\maketitle
\tableofcontents


\section{Summary of Work}
Below is a summary of my work up to now.
\section{System Model}
I consider a flat-fading MIMO channel with an IRS component and a line of sight (LOS) component. In order to analyze the capacity, I consider the channel without a transmitter or receiver filter.

\section{Channel Capacity}
The capacity of the channel from the system 

	 \begin{equation}\label{system_model}
		\mathbf{r} = \mathbf{H}_{Total}\mathbf{x}.
	\end{equation} is considered. 
	The total channel, is given by
	\begin{equation}
	\mathbf{H}_{Total} =  \underbrace{\mathbf{H}_{2}\boldsymbol{\Phi}\mathbf{H}_{1}}_{\text{IRS component}} + \underbrace{\mathbf{G}}_{\text{LOS component}}.
	\end{equation}
	with 
	$\mathbf{H}_{1}\in \mathbb{C}_{S \times T},\mathbf{H}_{2} \in \mathbb{C}_{R \times S}, \mathbf{G} \in \mathbb{C}_{R \times T}$ and $\boldsymbol{\Phi}$ as a diagonal matrix with elements $e^{j\phi_i}$.

Following the approach from \cite{muller2012channel}, we can find the AED of the channel 
covariance matrix, $\mathbf{H}_{Total}\mathbf{H}_{Total}^H$, using the asymptotic distribution of the "symmetrized singular values" of $\mathbf{H}_{Total}$.\cite{muller2012channel} provides a fixed point equation for the Stieltjes transform of this system. One problem, however, is that the system in this formulation considers
 elements of $ \mathbf{H}_{1}$ and  $\mathbf{H}_{2}$ to have variance $\frac{1}{RS}$. We want these
  matrices to have elements with variance $\frac{1}{s}$ and $\frac{1}{R}$ respectively.
  If I were to use the equation from \cite{muller2012channel}, the IRS component of the channel would have no impact on the capacity of system because all of the eigenvalues would be so small. In order to resolve this, I will need to re-derive the fixed point equation which requires using subordination methods. I am currently working this.
 \par
 Some numerical results investigating the eigenvalues of the terms from the covariance matrix polynomial are show in 
 section \ref{cross_terms}.

\subsection{Canceling Phases}
While deriving the fixed point equation for the AED, one step includes using additive free convolution to find the R-Transform for symmetrized singular value distributions using the property 
\begin{equation}
\tilde{R}_{\mathbf{H}_{Total}}(w) = \tilde{R}_{\mathbf{H}_{2}\boldsymbol{\Phi}\mathbf{H}_{1}}(w) + \tilde{R}_{\mathbf{G}}(w).
\end{equation} In order to find the individual terms $R_{\mathbf{H}_{2}\boldsymbol{\Phi}\mathbf{H}_{1}}(w)$ and $R_{\mathbf{G}}(w)$, we use the property relating the symmetrized singular value distribution to the eigenvalue distribution.
\begin{equation}\label{svd_aed_property}
\tilde{G}_{B}(s) = sG_{BB^{\dagger}}(s^2).
\end{equation}
We begin by finding the stieltjes transform of the eigenvalue distributions of the individual component,
$G_{\mathbf{H}_{2}\boldsymbol{\Phi}\mathbf{H}_{1}[\mathbf{H}_{2}\boldsymbol{\Phi}\mathbf{H}_{1}]^H}(s)$, in order to use property \eqref{svd_aed_property} to find $\tilde{G}_{\mathbf{H}_{2}\boldsymbol{\Phi}\mathbf{H}_{1}}(s)$ (the symmetrized singular valued distribution).
\par
To find $G_{\mathbf{H}_{2}\boldsymbol{\Phi}\mathbf{H}_{1}[\mathbf{H}_{2}\boldsymbol{\Phi}\mathbf{H}_{1}]^H}(s)$ we define
\begin{equation}
C_2 = \mathbf{H}_{2}\boldsymbol{\Phi}\mathbf{H}_{1}[\mathbf{H}_{2}\boldsymbol{\Phi}\mathbf{H}_{1}]^H
\end{equation}
and a rotated version 
\begin{equation}
\tilde{C}_2 = \boldsymbol{\Phi}^H\mathbf{H}_{2}^H\mathbf{H}_{2}\boldsymbol{\Phi}\mathbf{H}_{1}\mathbf{H}_{1}^H
\end{equation}. 
These allow use of the rotational property of the S-Transform from \cite{muller2002asymptotic}
\begin{equation}\label{rotation_property}
S_{C_2}(z) = \frac{z+1}{z+\chi_2} S_{\tilde{C}_2}(\frac{z}{\chi_2}).
\end{equation}
in which $\chi_N$ is the aspect ratio of the channel matrices. In the given model $\chi_2 = \frac{S}{R}$ and
$\chi_2 = \frac{S}{R}$.
Applying the rotational property again to the term 
\begin{equation}
C_{1} = \boldsymbol{\Phi}^H\mathbf{H}_{2}^H\mathbf{H}_{2}\boldsymbol{\Phi}
\end{equation}
we get 
\begin{equation}
\tilde{C}_{1} = \boldsymbol{\Phi}\boldsymbol{\Phi}^H\mathbf{H}_{2}^H\mathbf{H}_{2}
\end{equation}
with $\boldsymbol{\Phi}\boldsymbol{\Phi}^H = \mathbf{I}$.

From this we can conclude that for $R,S,T \rightarrow \infty $ the only impact of the IRS comes from the aspect ratio
$\chi_N$ but the actual phases of the IRS do not impact the final AED.
The results shown in \cite{zhang2019capacity} do not reflect that the phases play no role in capacity because 
the matrix dimensions chosen, $R = 4, S = 100 , T = 4$ are not applicable to asymptotic analysis. 
These results can be seen in 
\section{Optimization}
For the system
\begin{equation}
		\mathbf{r} = \mathbf{H}_{Total}\mathbf{x},
	\end{equation}
	with  $R,S,T \rightarrow \infty $,
we now consider
how to choose the transmit covariance matrix $\mathbf{Q}_x$ and $\boldsymbol{\Phi}$ in order to optimize an objective.
First, We will optimize the precoder and IRS phases to maximize the channel capacity.
		    \begin{equation}
    \begin{array}{ll}
    \underset{\boldsymbol{\Phi}, \mathbf{Q}_x}{\text{maximize }}   & \Expect\left[\Log\left(|\mathbf{I}_{N_R}+\frac{1}{\sigma_n}[\mathbf{H}_{2}\boldsymbol{\Phi}\mathbf{H}_{1} + \mathbf{G}]\mathbf{Q}_x[\mathbf{H}_{2}\boldsymbol{\Phi}\mathbf{H}_{1} + \mathbf{G}]^H|\right)\right]
    \\
    \mbox{subject to } & \|\phi_i\|^2 = 1
    \\
    & tr(\mathbf{Q}_x) \leq P_{\text{Total}}
%        \\
%    & \tilde{\mathbf{f}}_i \mathbf{h}_j = 0, \; i\neq j, \; \forall i,j \in N
    \end{array}
    \label{general_irs_opt}
    \end{equation} 
    
    
%We assume that  $R,S,T \rightarrow \inf $. In cases when  $\chi_N = 1$ it is shown that the phases do 
%not impact the capacity and thus standard water-filling can be used. If, however, $\chi_N \neq 1$, 
%Then the problem is non-convex and many approaches are possible. 

\section{Simulation Results}

\subsection{Evaluating Impact of Cross-terms on AED}\label{cross_terms}
The polynomial describing the channel
covariance matrix given by 
\begin{equation}
\mathbf{H}_{Total}\mathbf{H}_{Total}^H = 
\underbrace{
\mathbf{H}_{2}\boldsymbol{\Phi}\mathbf{H}_{1}\mathbf{G}^H + \mathbf{G}\mathbf{G}^H}_{\text{cross terms}}
+ 
\underbrace{
\mathbf{H}_{2}\boldsymbol{\Phi}\mathbf{H}_{1}[\mathbf{H}_{2}\boldsymbol{\Phi}\mathbf{H}_{1}]^H + \mathbf{G}[\mathbf{H}_{2}\boldsymbol{\Phi}\mathbf{H}_{1}]^H}_{\text{non-cross terms}}.
\end{equation} 
Here we wanted to see if the final AED is influenced by the cross terms. The case in which elements of the channel matrices are zero-mean with variance  $\frac{1}{N}$ is shown in \ref{cross}. In this case, the cross terms appear to have symmetry such that they will not influence the final AED.

\begin{figure}[H]
	\includegraphics[width=\textwidth]{results/cross}
\caption{Eigenvalue histogram for N=500. Cross terms represent the eigenvalues of the matrix sum $(\mathbf{H}_{2}\boldsymbol{\Phi}\mathbf{H}_{1}\mathbf{G}^H + \mathbf{G}[\mathbf{H}_{2}\boldsymbol{\Phi}\mathbf{H}_{1}]^H)$ and the non-cross terms are the eigenvalues of the matrix sum $(\mathbf{H}_{2}\boldsymbol{\Phi}\mathbf{H}_{1}[\mathbf{H}_{2}\boldsymbol{\Phi}\mathbf{H}_{1}]^H + \mathbf{G}\mathbf{G}^H)$}
\label{cross}
\end{figure}

If the elements of the channel matrix are, however, not zero-mean the cross terms will impact the channel AED. 
This scenario is shown in \ref{cross_non_zero_mean}.

\begin{figure}[H]
	\includegraphics[width=\textwidth]{results/cross_non_zero_mean}
\caption{Eigenvalue histogram for N=500 with non-zero mean elements.}
\label{cross_non_zero_mean}
\end{figure}

\subsection{Impact of Phases on total Capacity}\label{phase_impact}
It is hypothesized that in the asymptotic limit, if $\chi_n = 1$ the phases will not impact the capacity. In order to check this numerically I evaluated the capacity while holding the channel values constant for different values of the IRS phases. Here statistics are shown for 100 random realizations of the phases at the IRS. 
Phases, $e^{j\phi}$, are generated randomly over a uniform distribution $\phi \in [0,2\pi]$. This is clearly not rigorous but perhaps provides some insight. 
$T = R = S = 500$
\begin{itemize}
\item
Capacity variance:
 1.1512987368371963e-05
 \item
Capacity average:
 1.990962024050068
 \item
Capacity min:
 1.9818993465470727
 \item
Capacity max:
 1.9986482518130326
\end{itemize}
Similarly, the results for $T = 500$, $ R = 500$ and $S = 1000$ are shown 
\begin{itemize}
\item
Capacity variance:
 1.730056273847992e-05
 \item 
Capacity average:
 1.9937729377849243
 \item
Capacity min:
 1.9855835490537985
 \item
Capacity max:
 2.0025075539557737
\end{itemize}

%\textcolor{red}{\chi_N}}

In the non-asymptotic case with $T = R = 4$ and $M = 100$, the results below indicate that the phases play a more important role in determining the capacity. 

\begin{itemize}
\item 
Capacity variance:
 0.04863896876118487
 \item
Capacity average:
 1.695035095091801
 \item
Capacity min:
 1.2342437149649095
 \item
Capacity max:
 2.320914548447325
\end{itemize}





\bibliography{bibliography}
\end{document}
